% Documentation technique d'OBM : module Agenda
% ALIACOM Mehdi Rande
% $Id$

\clearpage
\section{L'agenda partagé}

L'\agenda \obm  est un agenda partagé permettant d'inserer, modifier, 
consulter ou supprimer des évenements pour un ou plusieurs utilisateurs 
simultanés.

\subsection{Organisation de la base de données}

Le \calendar utilise 5 tables :
\begin{itemize}
 \item CalendarEvent
 \item CalendarException
 \item CalendarCategory
 \item EntityEvent
 \item EntityRight
\end{itemize}

\subsubsection{CalendarCategory}
Cette table est utilisée pour sotcker les catégories des événements.\\

\begin{tabular}{|p{3cm}|c|p{5.4cm}|p{2.6cm}|}
\hline
\textbf{Champs} & \textbf{Type} & \textbf{Description} & \textbf{Commentaire} \\
\hline
\_id & int 8 & Identifiant & Clé primaire \\
\hline
\_timeupdate & timestamp 14 & Date de mise à jour & \\
\hline
\_timecreate & timestamp 14 & Date de création & \\
\hline
\_userupdate & int 8 & Id du modificateur & \\
\hline
\_usercreate & int 8 & Id du créateur & \\
\hline
\_label & varchar 128 &  Label de la catégorie & \\
\hline
\end{tabular}

\subsubsection{CalendarEvent}
Cette table stocke toute la description d'un évenement ainsi que ces caractéristiques.\\

\begin{tabular}{|p{3cm}|c|p{5.4cm}|p{2.6cm}|}
\hline
\textbf{Champs} & \textbf{Type} & \textbf{Description} & \textbf{Commentaire} \\
\hline
\_id & int 8 & Identifiant &  Clé primaire \\
\hline
\_timeupdate & timestamp 14 & Date de mise à jour & \\
\hline
\_timecreate & timestamp 14 & Date de création & \\
\hline
\_userupdate & int 8 & Id du modificateur &  Clé etrangère\\
\hline
\_usercreate & int 8 & Id du créateur & Clé etrangère\\
\hline
\_owner & int 8 & Id du propriétaire & Clé etrangère\\
\hline
\_title & varchar 255 & Titre & \\
\hline
\_location & varchar 100 & Lieu & \\
\hline
\_description & text & Description & \\
\hline
\_category\_id & int 8 & Id de la catégorie & Clé etrangère\\
\hline
\_priority & int 2 & Priorité : <1> Basse <2> Normal <3> Haute & \\
\hline 
\_privacy & int 2 & Privé : <0> Non <1> Oui & \\
\hline
\_date & timestamp & Date de début de la première occurence de l'événement & \\
\hline
\_duration & int 8 & Durée de l'événement en secondes & \\
\hline
\_allday & int 1 & Evénement sur toute la journée <0> Non <1> Oui & \\
\hline
\_repeatkind & varchar 20 & Type de répétition & \\
\hline
\_repeatfrequence & int 3 & Fréquence de répétition & \\
\hline
\_repeatdays & varchar 7 & Jours de répétion pour les répétition de type
hebdomadaire & \\
\hline
\_endrepeat & timestamp & Date de fin de répétition & \\
\hline
\end{tabular}


\subsubsection{CalendarException}

Cette table stocke les exceptions des événements périodiques.\\

\begin{tabular}{|p{3cm}|c|p{5.4cm}|p{2.6cm}|}
\hline
\textbf{Champs} & \textbf{Type} & \textbf{Description} & \textbf{Commentaire} \\
\hline
\_timeupdate & timestamp 14 & Date de mise à jour & \\
\hline
\_timecreate & timestamp 14 & Date de création & \\
\hline
\_userupdate & int 8 & Id du modificateur & Clé etrangère\\
\hline
\_usercreate & int 8 & Id du créateur & Clé etrangère\\
\hline
\_event\_id & int 8 & Id de l'événement & Clé etrangère - Clé primaire \\
\hline
\_date & timestamp & Date de l'exception & \\
\hline
\end{tabular}


\subsubsection{EventEntity}

Cette table stocke toutes les relations entre les événements et les entités (utilisateurs, groupes, resources,...).\\


\begin{tabular}{|p{3cm}|c|p{5.4cm}|p{2.6cm}|}
\hline
\textbf{Champs} & \textbf{Type} & \textbf{Description} & \textbf{Commentaire} \\
\hline
\_timeupdate & timestamp 14 & Date de mise à jour & \\
\hline
\_timecreate & timestamp 14 & Date de création & \\
\hline
\_userupdate & int 8 & Id du modificateur & Clé etrangère\\
\hline
\_usercreate & int 8 & Id du créateur & Clé etrangère\\
\hline
\_event\_id & int 8 & Id de l'événement & Clé etrangère - Clé primaire \\
\hline
\_entity\_id & int 8 & Id de l'entité & Clé etrangère - Clé primaire \\
\hline
\_entity & varchar 32 & nom de l'entité & Clé primaire - 'user', 'resource',... \\
\hline
\_state & char 1 & Etat de l'événément : <R> Refusé, <W> En attente, et <A> 
Accepté  & \\
\hline 
\_required & int 1 & Détermine si l'utilisateur est nécessaire pour l'événement
: <0> Non <1> Oui & \\
\hline
\end{tabular}


\subsubsection{EntityRight}

Table générique de gestion des droits individuels ou ACL.
L'entité est l'objet sur lequel on fixe les droits (un calendrier, une ressource,...).
Le consommateur est l'objet à qui on donne les droits (utilisateur ou groupe).

Voir section \ref{acl}.\\

\begin{tabular}{|p{3cm}|c|p{5.4cm}|p{2.6cm}|}
\hline
\textbf{Champs} & \textbf{Type} & \textbf{Description} & \textbf{Commentaire} \\
\hline
\_entity &varchar 32 & type de l'entité & Clé primaire - 'calendar', 'resource'\\
\hline
\_entity\_id & int 8 & Id de l'entité & Clé primaire\\
\hline
\_consumer & varchar 32 & type du consommateur & Clé primaire - 'user, 'group\\
\hline
\_consumer\_id & int 8 & Id du consommateur & Clé primaire\\
\hline 
\_read & int 1 & Droit de lecture <0> Non <1> Oui  & \\
\hline
\_write & int 1 & Droit d'ecriture <0> Non <1> Oui & \\
\hline
\end{tabular}


\subsection{Actions et droits}

Voici la liste des actions du module \agenda, avec le droit d'accès requis 
ainsi qu'une description sommaire de chacune d'entre elles.\\

\begin{tabular}{|l|c|p{9.5cm}|}
 \hline
 \textbf{Intitulé} & \textbf{Droit} & \textbf{Description} \\
 \hline
  index & read & Accueil \\ 
 \hline
  decision & write & Définit l'état d'une liste d'événements \\
 \hline
  view\_year & read & Vue annuelle \\
 \hline
  view\_month & read & Vue mensuelle \\
 \hline
  view\_week & read & Vue hebdomadaire \\
 \hline
  view\_day & read & Vue quotidienne \\
 \hline
  new & write & Formulaire de nouvel événement \\
 \hline
  insert & write & Insertion de l'événement \\
 \hline
  detail\_consult & read & Consultation d'un événement \\
 \hline
  check\_delete & admin & Confirmation de suppression d'événement \\
 \hline
  delete & write & Suppression d'un événement \\
 \hline
  detailupdate & write & Formulaire de mise à jour d'un événement \\
 \hline
  update & write & Mise à jour d'un événement \\
 \hline
  update\_decision & write & Mise à jour de l'état d'un événements \\
 \hline
  rights\_admin & write & Formulaire de mise à jour des droits \\
 \hline
  rights\_update &  write & Mise à jour des droits \\
 \hline
  new\_meeting & write & Formulaire de nouvelle réunion \\
 \hline
  perform\_meeting & write & Vue réunion \\
 \hline
\end{tabular}


\subsection{Paramètres}

Différents paramètres concernant les utilisateurs, groupes et leur stockage dans le tableau de paramètres :\\

\begin{tabular}{|l|p{3cm}|p{8cm}|}
 \hline
 \textbf{Haschage \$agenda} & \textbf{Paramètre} & \textbf{Description} \\
 \hline
  [``user\_id''] & param\_user & Utilisateur spécifié lors de l'acceptation d'événement (déléguation) \\ 
 \hline
  [``group\_view''] & param\_group & Groupe sélectionné pour l'affichage du calendrier d'un groupe \\ 
 \hline
  [``new\_group''] & new\_group & Indicateur de sélection d'un nouveau groupe\\
 \hline
  [``sel\_group\_id''] & sel\_group\_id & Groupes sélectionnés (formulaire rdv) \\ 
 \hline
  [``sel\_user\_id''] & sel\_user\_id ou sel\_ent[``user''] & Utilisateurs sélectionnés \\ 
 \hline
  [``sel\_resource\_id''] & sel\_resource\_id ou sel\_ent[``resource] & Ressources sélectionnées \\ 
 \hline
\end{tabular}


\subsection{Sélection et stockage des utilisateurs, resources, groupes}

Lors de la sélection de plusieurs utilisateurs ou entités, ceux-ci sont gardés en session afin de maintenir la sélection en changeant de vue ou en se déplaçant dans le calendrier.
De même quand un événement est créé pour plusieurs utilisateurs, la vue passe automatiquement en vue multi-utilisateurs avec ces utilisateurs sélectionnés.

Cependant les données stockées en session ne sont pas systématiquement les données sélectionnées; exemple :
\begin{itemize}
\item Quand un groupe est sélectionné, la vue passe automatiquement en vue membres du groupe, et le choix des utilisateurs est limité au groupe (restreint à ceux qui ont donné les droits en lecture).
\item La sélection est limitée par défaut au nombre d'utilisateurs affichables simultanément (par défaut 6).
\end{itemize}
\vspace{0.3cm}

Règles utilisées pour la sélection et le stockage en session des entités :\\

Les entités sélectionnées sont stockées dans le tableau \variable{\$cal\_entity\_id} qui est stocké en session à la fin de l'exécution (car la sélection peut être modifiée par les traitements, comme la restriction au nombre d'utilisateurs affichables simultanément) et qui comprend plusieurs entrées :
\begin{itemize}
\item[-] [user] : tableau d'id d'utilisateurs
\item[-] [resource] : tableau d'id de ressources
\item[-] [group] : tableau d'id de groupes
\item[-] [group\_view] : groupe sélectionné pour l'affichage des utilisateurs
\end{itemize}
\vspace{0.3cm}

Le tableau \variable{\$cal\_entity\_id} est copié dans le hashage global \variable{\$agenda[entity]} afin de le mettre directement à disposition des différentes fonctions.\\

\#nb correspond au nombre maximal de calendriers affichables simultanément.\\

Si aucune entité n'est sélectionnée (utilisateur, groupe, ressource) l'utilisateur connecté est automatiquement sélectionné.\\

\begin{longtable}{|p{4cm}|p{10cm}|}
\hline
\textbf{Action} & \textbf{Sélection} \\

\hline
Affichage d'une vue
&
\begin{itemize}
\item Session inchangée (listes déjà été limitées à \#nb)
\item Affichage calendriers  : contenu de [user], [resource]
\item Group sélectionné : inchangé
\end{itemize}
\\ 

\hline
Sélection de 1 ou plusieurs utilisateurs ou ressources en mode vue
&
\begin{itemize}
\item[-] [user] = sélection d'utilisateurs limitée à \#nb
\item[-] [group] = inchangé
\item[-] [resource] = sélection de ressources limitée à \#nb - \#nb\_users
\item Affichage calendriers : contenu de [user] [resource]
\item Groupe sélectionné : aucun 
\end{itemize}
\\ 

\hline
Sélection de 1 groupe mode vue
&
\begin{itemize}
\item[-] [user] = utilisateurs du groupe sélectionné limité à \#nb
\item[-] [group] = groupe sélectionné
\item[-] [resource] = vide
\item Affichage calendriers : contenu de [user]
\item Group sélectionné : le groupe sélectionné
\end{itemize}
\\ 

\hline
Fenêtre nouveau rdv
&
\begin{itemize}
\item Session inchangée (listes déjà été limitées à \#nb)
\item Utilisateurs pré-sélectionnés : contenu de [user] 
\item Groupes pré-sélectionnés : contenu de [group] 
\item Ressources pré-sélectionnées : contenu de [resource]
\end{itemize}
\\ 

\hline
Création rdv : Sélection de 1 ou plusieurs utilisateurs, groupes, ressources
&
\begin{itemize}
\item[Note] Les vues limitent les sélections, non les insertions
\item[-] [user] = utilisateurs sélectionnés, limité à \#nb
\item[-] [group] = groupes sélectionnés
\item[-] [resource] = ressources sélectionnées limité à \#nb - \#nb\_users
\item Affichage calendriers  : contenu de [user], [resource] 
\item Group sélectionné : aucun
\end{itemize}
\\

\hline
Fenêtre nouvelle réunion
&
\begin{itemize}
\item Utilisateurs pré-sélectionnés : contenu de [user] 
\item Groupes pré-sélectionnés : contenu de [group] 
\item Ressources pré-sélectionnées : contenu de [resource] 
\end{itemize}
\\ 

\hline
Recherche crénneaux disponibles
&
\begin{itemize}
\item[-] [user] = utilisateurs sélectionnés
\item[-] [group] = groupes sélectionnés
\item[-] [resource] = ressources sélectionnées
\end{itemize}
\\
\hline
\end{longtable}



\subsubsection{Remarques}
 Par défaut l'action \textbf{index} affiche la vue hebdomadaire de la semaine
 courante. Cependant si l'utilisateur courant ou bien un utilisateur qui lui a 
 cédé les droits d'ecriture sur son agenda à des événements en attente cette
 action affichera la liste des événements en attente.

\subsection{Principe de fonctionnement d'affichage d'une vue}
 Les etapes de construction de la vue sont :
 \begin{itemize}
  \item{Récupération des informations en base de données}
  \item{Traitement des données et construction du modèle}
  \item{Traitement du modèle et construction de la vue}
  \item{Affichage de la vue}
 \end{itemize}

\subsubsection{Structure du modèle}
 Le modèle de donnée est un des rare éléments à utiliser objets au sein d'un
 module. Ceci vient du fait que l'approche objet est particulierement adapté aux
 traitement à effectuer.

 Voici les différents éléments de ce modèle :
\paragraph{L'objet Event}
 L'objet Event représente un événement au sens abstrait.
 C'est à dire que quelque soit le nombre d'occurence d'un événement (dans le cas
 d'un événement répétitif) il n'y aura qu'un seul objet Event.

 \begin{tabular}{|l|p{9.5cm}|}
  \hline
   \textbf{Attribut} & \textbf{Description} \\
  \hline
   id & Identifiant \\
  \hline
   duration & Durée en secondes \\
  \hline
   title & Titre \\
  \hline
   category & Label de la catégorie \\
  \hline
   privacy & Si événement privé \\
  \hline
   description & Description de l'événement \\
  \hline
 \end{tabular}

\paragraph{L'objet Day}
 L'objet Day represente un jour. Il stockera toutes les informations relatives à
 ce jour : date, événements, ... 
 \\
 \begin{tabular}{|l|p{9.5cm}|}
  \hline
   \textbf{Attribut} & \textbf{Description} \\
  \hline
   day & Date du jour \\
  \hline
   events & Tableau d'objet Events lié avec des id utilisateurs et une heure de
   début \\
  \hline
   title & Titre \\
  \hline
 \end{tabular}
 \\
 \\
 \\
 \begin{tabular}{|l|c|p{9.5cm}|}
  \hline
   \textbf{Fonction} & \textbf{Paramètres} & \textbf{Description} \\
  \hline
   is\_same\_day & date & Return true si la date passée en parametre est la meme
   que celle de l'objet\\
  \hline
   add\_even & event - begin\_date - uid & Ajoute un événement au tableau
   \textbf{events}, le lie avec l'heure de début et l'uid passé en paramètre. Si
   l'événement été déjà stocké, l'événement est juste lié à l'uid.\\
  \hline
   get\_events & uid & Retourne tout les événement du tableau \textbf{events} 
   de l'utilisateurs dont l'uid est passé en paramètre.\\
  \hline
  \hline
   have\_events\_between & start - end & Retourne tout les événement du tableau \textbf{events} 
   qui se débute entre l'heure \textbf{start} et l'heure \textbf{end} \\
  \hline
   get\_events\_between & start - end - uid & Retourne tout les événement du tableau \textbf{events} 
   qui se débute entre l'heure \textbf{start} et l'heure \textbf{end} pour
   l'utilisateur \textbf{uid} \\
  \hline
 \end{tabular}
  \textbf{Remarques :} \\
  Les fonctions \textbf{have\_events\_between} et \textbf{get\_events\_between}
  retourn \textbf{NULL} si aucun événements ne débute durant l'interval de
  temps et que aucun événements ne se déroule durant l'interval de
  temps, \textbf{-1} si aucun événements ne débute durant l'interval de
  temps mais que des événements sont déjà en cours.
  
 \subsubsection{Affichage de la vue}
   La fonction qui doit construire est afficher la vue reçoit donc un tableau
   d'objets \textbf{Day} qui correspondent aux jours à afficher.
   Pour construire la vue, la fonction interroge juste les objets \textbf{Day}
   du tableau pour savoir si un jour donné contient des informations à afficher,
   et si oui quelles sont ces information.

