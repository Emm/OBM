% Documentation technique d'OBM : module Project
% ALIACOM Pierre Baudracco
% $Id$


\clearpage
\section{Project}
révision : \obm 1.2.4

Le module \project \obm.

\subsection{Organisation de la base de données}

Le module \project utilise 5 tables :
\begin{itemize}
 \item Project
 \item ProjectTask
 \item ProjectRefTask
 \item ProjectUser
 \item ProjectStat
\end{itemize}

\subsection{Project}
Table principale des informations d'un projet.\\

\begin{tabular}{|p{3cm}|c|p{5.4cm}|p{2.6cm}|}
\hline
\textbf{Champs} & \textbf{Type} & \textbf{Description} & \textbf{Commentaire} \\
\hline
\_id & int 8 & Identifiant & Clé primaire \\
\hline
\_timeupdate & timestamp 14 & Date de mise à jour & \\
\hline
\_timecreate & timestamp 14 & Date de création & \\
\hline
\_userupdate & int 8 & Id du modificateur & \\
\hline
\_usercreate & int 8 & Id du créateur & \\
\hline
\_name & varchar 128 & Nom du projet & \\
\hline
\_shortname & varchar 10 & Nom court du projet & \\
\hline
\_tasktype\_id & int 8 & Type de tâche du projet & \\
\hline
\_company\_id & int 8 & Société contractante & NULL (projets internes)\\
\hline
\_deal\_id & int 8 & Affaire d'origine & NULL (projets internes)\\
\hline
\_soldtime & int 8 & Durée (jours) vendue & \\
\hline
\_estimatedtime & int 8 & Durée (jours) estimée & \\
\hline
\_datebegin & date & Date de début de projet & \\
\hline
\_dateend & date & Date de fin de projet & \\
\hline
\_archive & int 2 & Indicateur d'archivage & (1 = 0ui)\\
\hline
\_comment & text (64k) & Commentaire &\\
\hline
\end{tabular}


\subsection{ProjectTask}
Table des informations des tâches des projets.\\

\begin{tabular}{|p{3cm}|c|p{5.4cm}|p{2.6cm}|}
\hline
\textbf{Champs} & \textbf{Type} & \textbf{Description} & \textbf{Commentaire} \\
\hline
\_id & int 8 & Identifiant & Clé primaire \\
\hline
\_project\_id & int 8 & Projet de la tâche & \\
\hline
\_timeupdate & timestamp 14 & Date de mise à jour & \\
\hline
\_timecreate & timestamp 14 & Date de création & \\
\hline
\_userupdate & int 8 & Id du modificateur & \\
\hline
\_usercreate & int 8 & Id du createur & \\
\hline
\_label & varchar 128 & Label de la tâche & \\
\hline
\_parenttask\_id & int 8 & Tâche mère (ProjectTask) & 0 si racine \\
\hline
\_rank & int 8 & Ordre d'affichage de la tâche & \\
\hline
\end{tabular}


\subsection{ProjectRefTask}
Table des informations des tâches de référence.\\

\begin{tabular}{|p{3cm}|c|p{5.4cm}|p{2.6cm}|}
\hline
\textbf{Champs} & \textbf{Type} & \textbf{Description} & \textbf{Commentaire} \\
\hline
\_id & int 8 & Identifiant & Clé primaire \\
\hline
\_timeupdate & timestamp 14 & Date de mise à jour & \\
\hline
\_timecreate & timestamp 14 & Date de création & \\
\hline
\_userupdate & int 8 & Id du modificateur & \\
\hline
\_usercreate & int 8 & Id du createur & \\
\hline
\_tasktype\_id & int 8 & Type de tâche de la tâche de référence & \\
\hline
\_label & varchar 128 & Label de la tâche de référence & \\
\hline
\end{tabular}


\subsection{ProjectUser}
Table d'affectation (avec informations associées) d'utilisateurs aux tâches d'un projet.

\begin{tabular}{|p{3cm}|c|p{5.4cm}|p{2.6cm}|}
\hline
\textbf{Champs} & \textbf{Type} & \textbf{Description} & \textbf{Commentaire} \\
\hline
\_id & int 8 & Identifiant & Clé primaire \\
\hline
\_project\_id & int 8 & Projet & \\
\hline
\_user\_id & int 8 & Utilisateur & \\
\hline
\_projecttask\_id & int 8 & Tâche & \\
\hline
\_timeupdate & timestamp 14 & Date de mise à jour & \\
\hline
\_timecreate & timestamp 14 & Date de création & \\
\hline
\_userupdate & int 8 & Id du modificateur & \\
\hline
\_usercreate & int 8 & Id du createur & \\
\hline
\_projectedtime & int 8 & Durée prévue & \\
\hline
\_missingtime & int 8 & Durée restante estimée & Par Chef P \\
\hline
\_validity & timestamp 14 & Date d'estimation d'avancement & \\
\hline
\_soldprice & int 8 & Prix de vente de l'utilisateur & \\
\hline
\_manager & int 1 & Statut de l'utilisateur dans le projet (si tâche est nulle) & 1 = CP \\
\hline
\end{tabular}

\subsubsection{Remarques}

\paragraph{projectuser\_validity} : Date de la dernière modification de l'avancement qui permet de vérifier la crédibilité des informations disponibles sur l'avancement du projet (non implémenté pour l'instant).
\paragraph{projectuser\_soldprice} : Tarif journalier de cet utilisateur sur cette tâche. Permet une analyse des coûts plus détaillée pour le projet (non implémenté pour l'instant).
\paragraph{projectuser\_manager} : Indique si la personne est chef de projet. Influe sur les droits de modification.\\


ProjectUser est en fait la table de liaison entre Tâches et Membres d'un projet.
L'ajout d'un membre au projet insère un tuple dans cette table avec une tâche nulle. Ainsi, la \textbf{liste des membres d'un projet} est directement tirée de cette table.

L'information de statut d'un utilisateur (Chef de projet,...) doit être lue dans cet enregistrement dont la tâche est nulle.

\subsection{ProjectStat}
Table d'historique des informations relatives à l'avancement des projet.\\

\begin{tabular}{|p{3cm}|c|p{5.4cm}|p{2.6cm}|}
\hline
\textbf{Champs} & \textbf{Type} & \textbf{Description} & \textbf{Commentaire} \\
\hline
\_project\_id & int 8 & Projet & \\
\hline
\_usercreate & int 8 & Id du créateur & \\
\hline
\_date & timestamp 14 & Date d'estimation d'avancement & \\
\hline
\_useddays & int 8 & Durée passée (jours) sur le projet & \\
\hline
\_remainingdays & int 8 & Durée restante estimée & Par Chef P \\
\hline
\end{tabular}


\subsection{Actions et droits}

Voici la liste des actions du module \project, avec le droit d'accès requis ainsi qu'une description sommaire de chacune d'entre elles.\\

\begin{tabular}{|l|c|p{9.5cm}|}
 \hline
 \textbf{Intitulé} & \textbf{Droit} & \textbf{Description} \\
 \hline
 \hline
  index & read & (Défaut) formulaire de recherche de projets. \\ 
 \hline
  search & read & Résultat de recherche de projets. \\
 \hline
  new & write & Formulaire de création d'un projet. \\
 \hline
  detailconsult & read & Fiche détail d'un projet. \\
 \hline
  detailupdate & write & Formulaire de modification d'un projet. \\
 \hline
  insert & write & Insertion d'un projet. \\
 \hline
  update & write & Mise à jour du projet. \\
 \hline
  check\_delete & write & Vérification avant suppression du projet. \\
 \hline
  delete & write & Suppression du projet. \\
 \hline
  task & write & Liste des tâches définies et formulaire de nouvelle tâche. \\
 \hline
  task\_add & write & Ajout d'une tâche au projet. \\
 \hline
  task\_update & write & Modification d'une tâche. \\
 \hline
  task\_del & write & Suppression de tâches. \\
 \hline
  dashboard & read & consultation du tableau de bord. \\
 \hline
  planning & read & consultation du planning. \\
 \hline
  member & write & Liste des participants au projet. \\
 \hline
  sel\_member & write & Appel externe au module \user pour l'ajout de participants).\\
 \hline
  member\_add & write & Ajout d'un participant au projet. \\
 \hline
  member\_del & write & Suppression de participants au projet. \\
 \hline
  member\_update & write & Mise à jour d'un participant (statut, chef de projet). \\
 \hline
  allocate & write & Formulaire d'affectation des participants aux tâches. \\
 \hline
  allocate\_update & write & Mise à jour de l'affectations des participants. \\
 \hline
  advance & write & Formulaire d'avancement d'un projet. \\
 \hline
  advance\_update & write & Mise à jour de l'avancement du projet. \\
 \hline
  display & read & Ecran de modification des préférences d'affichage. \\
 \hline
  dispref\_display & read & Modifie l'affichage d'un élément. \\
 \hline
  dispref\_level & read & Modifie l'ordre d'affichage d'un élément. \\
 \hline
  document\_add & write & Ajout de liens vers des documents. \\
 \hline
\end{tabular}
