% Documentation technique d'OBM : Module sauvegarde
% ALIACOM Pierre Baudracco
% $Id$

\section{Gestion des sauvegardes}

Depuis la version 1.0, OBM int�gre un module graphique de sauvegarde des donn�es.

La sauvegarde g�n�re une archive des donn�es de la base de donn�es et stocke cette archive dans un d�p�t de sauvegardes.\\

Les sauvegardes ne sont pas stock�es en base de donn�es mais uniquement sur disque pour des questions de coh�rence.
Les informations de sauvegarde sont contenues dans le nom du fichier de sauvegarde (date, version).
Le module \backup lit donc ses informations depuis le syst�me de fichiers (liste des sauvegardes,..).


\subsection{Les options de param�trage}

\begin{tabular}{|p{6cm}|p{8cm}|}
\hline
D�p�t des sauvegardes & \$cbackup\_path \\
\hline
Nom des fichiers de sauvegarde (fix�) & obmdb-AAAAMMJJ-HHMMSS-version.dump \\
\hline
\end{tabular}


\subsection{L'impl�mentation selon la base de donn�es}

\subsubsection{MySQL}

\begin{tabular}{|p{4cm}|p{10cm}|}
\hline
Sauvegarde & mysqldump -u \textit{user} -p\textit{password} \textit{db} > fichier\\
\hline
\multirow{3}{2cm}{Restauration} & mysql -u \textit{user} -p\textit{password} -e DROP DATABASE IF EXISTS \textit{db} \\ 
\cline{2-2}
& mysql -u \textit{user} -p\textit{password} -e CREATE DATABASE \textit{db} \\
\cline{2-2}
& mysql -u \textit{user} -p\textit{password} \textit{db} < fichier\\
\hline
\end{tabular}


\subsubsection{PostgreSQL}

Avec Postgres il n'est pas possible de d�truire une base de donn�es ayant des connexions ouvertes (donc par exemple celle utilis�e par \obm).
Il n'est pas possible depuis \obm de fermer toutes les connexions actives (il pourrait etre possible que l'utilisateur connect� n'ait plus de connexions, mais impossible d'agir sur celles des autres utilisateurs).

La restauration s'effectue donc en supprimant d'abord toutes les donn�es de toutes les tables (pas de suppression / cr�ation de base de donn�es).\\


\begin{tabular}{|p{4cm}|p{10cm}|}
\hline
Sauvegarde & pg\_dump -U \textit{user} \textit{db} -f fichier\\
\hline
\multirow{2}{2cm}{Restauration} & Effacement de toutes les donn�es des tables d'\obm \\ 
\cline{2-2}
& psql -U \textit{user} \textit{db} < fichier \\
\hline
\end{tabular}
