% Documentation technique d'OBM : Module sauvegarde
% ALIACOM Pierre Baudracco
% $Id$

\section{Gestion des sauvegardes}

Depuis la version 1.0, OBM intègre un module graphique de sauvegarde des données.

La sauvegarde génère une archive des données de la base de données et stocke cette archive dans un dépôt de sauvegardes.\\

Les sauvegardes ne sont pas stockées en base de données mais uniquement sur disque pour des questions de cohérence.
Les informations de sauvegarde sont contenues dans le nom du fichier de sauvegarde (date, version).
Le module \backup lit donc ses informations depuis le système de fichiers (liste des sauvegardes,..).


\subsection{Les options de paramétrage}

\begin{tabular}{|p{6cm}|p{8cm}|}
\hline
Dépôt des sauvegardes & \$cbackup\_path \\
\hline
Nom des fichiers de sauvegarde (fixé) & obmdb-AAAAMMJJ-HHMMSS-version.dump \\
\hline
\end{tabular}


\subsection{L'implémentation selon la base de données}

\subsubsection{MySQL}

\begin{tabular}{|p{4cm}|p{10cm}|}
\hline
Sauvegarde & mysqldump -u \textit{user} -p\textit{password} \textit{db} > fichier\\
\hline
\multirow{3}{2cm}{Restauration} & mysql -u \textit{user} -p\textit{password} -e DROP DATABASE IF EXISTS \textit{db} \\ 
\cline{2-2}
& mysql -u \textit{user} -p\textit{password} -e CREATE DATABASE \textit{db} \\
\cline{2-2}
& mysql -u \textit{user} -p\textit{password} \textit{db} < fichier\\
\hline
\end{tabular}


\subsubsection{PostgreSQL}

Avec Postgres il n'est pas possible de détruire une base de données ayant des connexions ouvertes (donc par exemple celle utilisée par \obm).
Il n'est pas possible depuis \obm de fermer toutes les connexions actives (il pourrait etre possible que l'utilisateur connecté n'ait plus de connexions, mais impossible d'agir sur celles des autres utilisateurs).

La restauration s'effectue donc en supprimant d'abord toutes les données de toutes les tables (pas de suppression / création de base de données).\\


\begin{tabular}{|p{4cm}|p{10cm}|}
\hline
Sauvegarde & pg\_dump -U \textit{user} \textit{db} -f fichier\\
\hline
\multirow{2}{2cm}{Restauration} & Effacement de toutes les données des tables d'\obm \\ 
\cline{2-2}
& psql -U \textit{user} \textit{db} < fichier \\
\hline
\end{tabular}
