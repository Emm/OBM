% Documentation technique d'OBM : module Document
% ALIACOM Pierre Baudracco
% $Id$


\clearpage
\section{Document}

Le module \doc \obm.

\subsection{Organisation de la base de données}

Le module \doc utilise 5 tables :
\begin{itemize}
 \item Document
 \item DocumentEntity
 \item DocumentMimeType
 \item DocumentCategory1
 \item DocumentCategory2
\end{itemize}

\subsection{Document}
Table principale des informations d'un document.\\

\begin{tabular}{|p{3cm}|c|p{5.4cm}|p{2.6cm}|}
\hline
\textbf{Champs} & \textbf{Type} & \textbf{Description} & \textbf{Commentaire} \\
\hline
\_id & int 8 & Identifiant & Clé primaire \\
\hline
\_timeupdate & timestamp 14 & Date de mise à jour & \\
\hline
\_timecreate & timestamp 14 & Date de création & \\
\hline
\_userupdate & int 8 & Id du modificateur & \\
\hline
\_usercreate & int 8 & Id du créateur & \\
\hline
\_title & varchar 255 & Titre du document & Description\\
\hline
\_name & varchar 255 & Nom du document & Nom du fichier\\
\hline
\_kind & int 8 & Type de document & (répertoire, fichier, lien)\\
\hline
\_mymetype\_id & int 8 & Type Mime du fichier & \\
\hline
\_category1\_id & int 8 & Catégorie 1 &\\
\hline
\_category2\_id & int 8 & Catégorie 2 &\\
\hline
\_privacy & int 1 & Visibilité du Document & Voir aussi \_acl\\
\hline
\_size & int 15 & Taille du document & en octets \\
\hline
\_author & varchar 255 & Auteur du document & \\
\hline
\_path & text (64k) & Chemin du document &\\
\hline
\_acl & text (64k) & Liste de contrôle d'accés &\\
\hline
\end{tabular}

\subsubsection{Le champ kind (type de document)}

\begin{tabular}{|c|c|l|}
\hline
\textbf{variable} & \textbf{valeur} & \textbf{Description}\\
\hline
\$cdoc\_kind\_dir & 0 & Répertoire \\
\hline
\$cdoc\_kind\_file & 1 & Fichier \\
\hline
\$cdoc\_kind\_link & 2 & Lien (http, https) \\
\hline
\end{tabular}


\subsection{DocumentEntity}

Table de liaison entre les documents et différentes entités d'\obm.\\

\begin{tabular}{|p{3cm}|c|p{5.4cm}|p{2.6cm}|}
\hline
\textbf{Champs} & \textbf{Type} & \textbf{Description} & \textbf{Commentaire} \\
\hline
\_document\_id & int 8 & Identifiant du document & \\
\hline
\_entity\_id & int 8 & Identifiant de l'entité & \\
\hline
\_entity & varchar 255 & Entité (company, contact,...) & \\
\hline
\end{tabular}


\subsection{DocumentMimeType}
Table des informations de types MIME.\\

\begin{tabular}{|p{3cm}|c|p{5.4cm}|p{2.6cm}|}
\hline
\textbf{Champs} & \textbf{Type} & \textbf{Description} & \textbf{Commentaire} \\
\hline
\_id & int 8 & Identifiant & Clé primaire \\
\hline
\_timeupdate & timestamp 14 & Date de mise à jour & \\
\hline
\_timecreate & timestamp 14 & Date de création & \\
\hline
\_userupdate & int 8 & Id du modificateur & \\
\hline
\_usercreate & int 8 & Id du créateur & \\
\hline
\_label & varchar 255 & Label du type MIME & \\
\hline
\_extension & varchar 10 & Extension du type (jpg, gif, txt,..) & \\
\hline
\_mime & varchar 255 & Type MIME \\
\hline
\end{tabular}


\subsection{DocumentCategory1}
Table des informations de catégorisation des documents :  category 1.
Lien monovalué.\\

\begin{tabular}{|p{3cm}|c|p{5.4cm}|p{2.6cm}|}
\hline
\textbf{Champs} & \textbf{Type} & \textbf{Description} & \textbf{Commentaire} \\
\hline
\_id & int 8 & Identifiant & Clé primaire \\
\hline
\_timeupdate & timestamp 14 & Date de mise à jour & \\
\hline
\_timecreate & timestamp 14 & Date de création & \\
\hline
\_userupdate & int 8 & Id du modificateur & \\
\hline
\_usercreate & int 8 & Id du créateur & \\
\hline
\_label & varchar 255 & Label de la catégorie & \\
\hline
\end{tabular}


\subsection{DocumentCategory2}
Table des informations de catégorisation des documents :  category 2.
Lien monovalué.\\

\begin{tabular}{|p{3cm}|c|p{5.4cm}|p{2.6cm}|}
\hline
\textbf{Champs} & \textbf{Type} & \textbf{Description} & \textbf{Commentaire} \\
\hline
\_id & int 8 & Identifiant & Clé primaire \\
\hline
\_timeupdate & timestamp 14 & Date de mise à jour & \\
\hline
\_timecreate & timestamp 14 & Date de création & \\
\hline
\_userupdate & int 8 & Id du modificateur & \\
\hline
\_usercreate & int 8 & Id du créateur & \\
\hline
\_label & varchar 255 & Label de la catégorie & \\
\hline
\end{tabular}


\subsection{Gestion du dépot de documents}

\subsubsection{Principe}

Les informations d'un document sont stockées en base de données.
Les documents eux-même sont stockés sur disque.\\

\obm gère de façon autonome le dépôt de documents; \obm gère automatiquement :
\begin{itemize}
\item L'arborescence physique des documents
\item Le nommage et l'emplacement des documents sur disque
\end{itemize}
\vspace{0.3cm}

Ceci permêt d'éviter :
\begin{itemize}
\item Les problèmes de sécurité d'accès aux fichiers du serveur 
\item Les problèmes de nommage des documents (jeux de caractères, caractères spéciaux,...) sur disque
\end{itemize}


\subsubsection{Implémentation et gestion par défaut}

La racine de l'arborescence des documents sur le serveur est spécifiée par le paramètre de configuration \variable{\$cdocument\_root}.\\

Par défaut \obm créé 9 répertoires nommés 1, 2, 3, 4, 5, 6, 7, 8 et 9 à la racine du dépôt de documents.

\paragraph{Le chemin d'un document} est calculé en fonction de l'id du document. Le dernier chiffre de l'identifiant du document définit le répertoire de stockage du document.
Ceci permet de répartir les documents dans plusieurs réperoires.

\paragraph{Le nom d'un document} est égal à son id. Ceci évite tout problème de nommage des fichiers.

\paragraph{Exemples} :
\begin{tabular}{|l|c|l|}
 \hline
 \textbf{Id Document} & \textbf{Chemin relatif} & \textbf{nom physique} \\
 \hline
  123 & 3 & \$cdocument\_root/3/123 \\ 
 \hline
  4 & 4 & \$cdocument\_root/4/4 \\ 
 \hline
  12156 & 6 & \$cdocument\_root/6/12156 \\ 
 \hline
\end{tabular}


\subsubsection{Outil de vérification de cohérence}

\obm fournit un outil de vérification de cohérence entre la base de données et les fichiers dans le dépôt de documents.\\

\textit{ADMINISTRATION -> Données -> action : show\_data, module document}.\\

Cet Outil référence :
\begin{itemize}
\item Les fichiers enregistrés en base de données et non présents sur disque
\item Les fichiers présents sur disque sans correspondance en base de données
\end{itemize}


\subsubsection{Modification du stockage des documents}

\obm permet une modification simple de la gestion du dépôt des documents. Une fonction dédiée est utilisée pour calculer le chemin d'un fichier en fonction de son Id. Cette fonction est aisément modifiable.\\

\shadowbox{
\begin{minipage}{13cm}
\begin{verbatim}
function get_document_disk_path($id) {
\end{verbatim}
\end{minipage}
}
Fonction de calcul du chemin physique réel d'un document sur disque.\\

Il est par exemple possible de définir 99 répertoires initiaux (à la place des 9) pour mieux supporter de gros volumes de documents.


\subsubsection{Gestion des liens}

Les liens sont stockés uniquement en base de données sans stockage physique sur disque.


\subsection{Gestion des types MIME}

A la création ou modification d'un document, le type MIME peut être sélectionné ou laissé par défaut.
Si le type MIME est laissé par défaut, \obm essaie de déterminer automatiquement le type MIME associé au fichier.

Ceci s'effectue en fonction de l'extention du fichier par exemple.\\

\shadowbox{
\begin{minipage}{13cm}
\begin{verbatim}
function get_auto_mime_type($document) {
\end{verbatim}
\end{minipage}
}
Fonction de détermination automatique du type MIME d'un document.\\

Si aucun type MIME n'est déterminé (en fonction des types MIME enregistrés en base), le type MIME par défaut est sélectionné selon le paramètre de configuration \variable{\$default\_mime}.


\subsection{Paramètres de configuration du module \doc}

\begin{tabular}{|l|l|l|}
\hline
\textbf{Paramètre} & \textbf{par défaut} & \textbf{Description}\\
\hline
\$cdocument\_root & /var/www/obmdocuments & Racine du dépôt de document sur disque \\
\hline
\$default\_mime & application/octet-stream & Type MIME par défaut d'un fichier quand non déterminé \\
\hline
\end{tabular}


\subsection{Actions et droits}

Voici la liste des actions du module \project, avec le droit d'accès requis ainsi qu'une description sommaire de chacune d'entre elles.\\

\begin{tabular}{|l|c|p{9.5cm}|}
 \hline
 \textbf{Intitulé} & \textbf{Droit} & \textbf{Description} \\
 \hline
 \hline
  index & read & (Défaut) formulaire de recherche de documents. \\ 
 \hline
  search & read & Résultat de recherche de documents. \\
 \hline
  new & write & Formulaire de création d'un document. \\
 \hline
  new\_dir & write & Formulaire de création d'un répertoire. \\
 \hline
  tree & read & Visualisation en arborescence. \\
 \hline
  detailconsult & read & Fiche détail d'un document. \\
 \hline
  detailupdate & write & Formulaire de modification d'un document. \\
 \hline
  insert & write & Insertion d'un document. \\
 \hline
  insert\_dir & write & Insertion d'un répertoire. \\
 \hline
  update & write & Mise à jour du document. \\
 \hline
  check\_delete & write & Vérification avant suppression du document. \\
 \hline
  delete & write & Suppression du document. \\
 \hline
  dir\_check\_delete & write & Vérification avant suppression du répertoire. \\
 \hline
  dir\_delete & write & Suppression du répertoire. \\
 \hline
  admin & write & Ecran d'administartion (gestion des catégories). \\
 \hline
  cat1\_insert & write\_admin & Insertion de la catégorie 1. \\
 \hline
  cat1\_update & write\_admin & Modification de la catégorie 1. \\
 \hline
  cat1\_checklink & write\_admin & Vérification des liens de la catégorie 1. \\
 \hline
  cat1\_delete & write\_admin & Suppression de la catégorie 1. \\
 \hline
  cat2\_insert & write\_admin & Insertion de la catégorie 2. \\
 \hline
  cat2\_update & write\_admin & Modification de la catégorie 2. \\
 \hline
  cat2\_checklink & write\_admin & Vérification des liens de la catégorie 2. \\
 \hline
  cat2\_delete & write\_admin & Suppression de la catégorie 2. \\
 \hline
  mime\_insert & write\_admin & Insertion du type MIME. \\
 \hline
  mime\_update & write\_admin & Modification du type MIME. \\
 \hline
  mime\_checklink & write\_admin & Vérification des liens du type MIME. \\
 \hline
  mime\_delete & write\_admin & Suppression du type MIME. \\
 \hline
  display & read & Ecran de modification des préférences d'affichage. \\
 \hline
  dispref\_display & read & Modifie l'affichage d'un élément. \\
 \hline
  dispref\_level & read & Modifie l'ordre d'affichage d'un élément. \\
 \hline
 \hline
  ext\_get\_path & read & Appel externe pour sélection d'un chemin. \\
 \hline
  ext\_get\_ids & read & Appel externe pour sélection de documents. \\
 \hline
\end{tabular}
