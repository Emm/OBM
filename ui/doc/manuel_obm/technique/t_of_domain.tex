% Documentation technique d'OBM : Gestion du multi-domaine
% AliaSource Pierre Baudracco
% $Id$


\subsection{Gestion du multi-domaine}
\label{of_domain}

\obm intègre la fonctionnalité multi-domaine depuis la version 2.0.
Le multi-domaine permet de gérer dans une même instance d'\obm plusieurs espaces de données totalement cloisonnés.\\

Le multi-domaine peut être activé ou désactivé. Il est intégré de la sorte qu'il n'y ait aucun impact pour les sites n'utilisant pas la fonctionnalité.\\

Les fonctionalités de multi-domaine sont fournies par le fichier \fichier{of/of\_query.inc}.

\subsubsection{Multi-domaine, données et utilisateurs}

Chaque données dans \obm est soit globale, soit appartient à un domaine unique.\\

Dans un fonctionnement mono-domaine, les données de domaine ne sont pas utilisées et sont renseignées soit à 0 soit à null (non renseignées).\\

Dans un fonctionnement multi-domaine, le \textbf{domaine 0} est un domaine particulier.\\

Pour les objets pouvant être globaux (définis dans le domaine 0) il doit y avoir unicité de la clé (login pour user,...) dans l'espace domaine local + domaine global.
La seule possibilité de doublon est pour 2 domaines locaux.\\ 

Pour les hôtes, l'adresse IP doit être unique tous domaines confondus.

\begin{tabular}{|p{4cm}|p{10cm}|}
\hline
\textbf{Donnée} & \textbf{Description}\\
\hline
Utilisateur domaine 0 & C'est un super-administrateur, il :
\begin{itemize}
\item peut gérer les domaines
\item peut gérer les utilisateurs de tous les domaines
\item peut voir toutes les données de tous les domaines
\item peut gérer les données transverses de référence (d'id 0)
\item au niveau LDAP il est créé dans tous les domaines (bals,..)
\end{itemize}
\vspace{0.3cm}

Pour la création de données en base : quand il se loggue il choisit un domaine :
\begin{itemize}
\item domaine global : il ne peut créer que les données qui permettent de sélectionner un domaine :
  \begin{itemize}
  \item user
  \item host
  \item mailshare
  \end{itemize}
\item domaine spécifique : il agit comme un utilisateur du domaine donc limité aux données du domaine.
\end{itemize}

\\
\hline
Utilisateur domaine != 0 & C'est un utilisateur de son domaine :
\begin{itemize}
\item il est limité (consultation / modification) à son domaine
\item toute donnée créée appartient à son domaine
\end{itemize}
\\
\hline
Entité domaine 0 & C'est une entité transverse, elle :
\begin{itemize}
\item est visible dans tous les domaines
\item n'est gérée que par un super-administrateur
\end{itemize}
\\
\hline
Entité domaine != 0 & C'est une entité de son domaine :
\begin{itemize}
\item elle est accessible par les utilisateurs du domaine (selon droits) et super-administrateurs
\item elle est modifiable par les utilisateurs du domaine (selon droits)
\end{itemize}
\\
\hline
\end{tabular}
\vspace{0.3cm}

Il n'est pas possible à un super-administrateur d'insérer des données (autres que référence) ou d'avoir des entités de domaine 0 pour éviter les cas de données de domaines différents liées entre-elles (ex: contact d'un domaine ou public lié à une société de domaine différent).\\

L'utilisation d'OBM-LDAP, OBM-MAIL ou OBM-SAMBA (dès que l'automate entre en jeu) nécessite la définition d'un domaine, même si un seul domaine est utilisé.\\

Au niveau des hôtes :
- pour la messagerie :
Il est nécessaire de choisir un hôte comme serveur de messagerie.
La liste des hôtes proposée est l'aggrégation des hôtes du domaine OBM et des hôtes du domaine global (cas serveur hébèrge plusieurs domaines).

- pour samba
Une machine ne peut appartenir qu'à un seul domaine windows.
Donc les hôtes déclarés dans le domaine global ne pourront pas être déclarés comme une machine windows.


\subsubsection{Table Domain et stockage dans les données}


\begin{tabular}{|p{3cm}|c|p{8cm}|}
\hline
\textbf{Champs} & \textbf{Type} & \textbf{Description} \\
\hline
\_id & int 8 & Id du domaine \\
\hline
\_label & varchar 32 & Label du domaine \\
\hline
\_description & varchar 255 & Description du domaine \\
\hline
\_domain\_name & varchar 128 & Nom de domaine principal (messagerie) \\
\hline
\_alias & text & alias de noms de domaines (messagerie) \\
\hline
\end{tabular}
\vspace{0.3cm}

Les champs \textbf{domain\_name} et \textbf{\_alias} n'apparaissent que dans le cadre D'OBM-LDAP, OBM-Mail ou OBM-Samba.
Le champ domain\_name n'est pas modifiable via l'interface (car nécessite une intervention sur la conf des services ldap, cyrus).\\



Chaque table d'entité (hors tables de liaison) doit contenir un champ référençant le domaine :\\

\begin{tabular}{|p{3cm}|c|p{4cm}|}
\hline
\textbf{Champs} & \textbf{Type} & \textbf{Description}\\
\hline
\_domain\_id & int 8 & Id du domaine de l'entité\\
\hline
\end{tabular}


\subsubsection{Implémentation : API fonctions publiques}

\paragraph{Fonctions d'affichage} : \\

\shadowbox{
\begin{minipage}{14cm}
\begin{verbatim}
of_domain_get_list() {
\end{verbatim}
\end{minipage}
}
Fonction qui retourne la liste des domaines dans un tableau. Le domaine ``0'' global est ajouté.\\


\shadowbox{
\begin{minipage}{14cm}
\begin{verbatim}
of_domain_get_domain_infos($d_id) {
\end{verbatim}
\end{minipage}
}
Fonction qui retourne les informations du domaine selon son id\\


\shadowbox{
\begin{minipage}{14cm}
\begin{verbatim}
of_domain_get_entity_domain_id($id, $entityu, $ptable="") {
\end{verbatim}
\end{minipage}
}

\begin{tabular}{|p{3cm}|p{10cm}|}
\hline
\textbf{Paramètres} & \textbf{Description}\\
\hline
\multicolumn{2}{|c|}{Fonction qui récupère le domaine id d'une entité}\\
\hline
\$id & id de l'entité\\
\hline
\$entityu & entité (majuscule pour nom de table) (ex: ``Company'', ``contact'',..)\\
\hline
\$ptable & nom de table (si différent de \$entityu) (ex: ``UGroup'')\\
\hline
\end{tabular}
\vspace{0.4cm}


\shadowbox{
\begin{minipage}{14cm}
\begin{verbatim}
of_domain_dis_select($domain_id) {
\end{verbatim}
\end{minipage}
}
Affiche le selecteur de domaines
