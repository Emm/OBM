% Documentation technique d'OBM : utilisation des css
% ALIACOM Medhi Rande
% $Id$

\section{Les feuilles de style - CSS}
\label{display}

Le code produit par OBM est et doit être conforme XHTML 1.0 Transitionnal.
Les feuilles de style (CSS) sont utilisées selon le standard CSS2. (1 ?).\\

OBM défini des zones graphiques (bandeau de gauche, bandeau de droite,...) ainsi que des objects graphiques (block des sections, fiche de détail, block des dernières entités visitées,...). Une fonction générale d'affichage \textbf{display\_page()}, spécifique à chaque thème indique la répartition des objets dans les zones.

\subsection{Les zones graphiques définies dans le thème standard}

L'affichage d'OBM est effectué dans des zones définies.
Le thème standard défini les zones :\\

\begin{tabular}{|c|l|}
\hline
\textbf{Zone} & \textbf{Description pour thème standard}\\
\hline
header & Bandeau du haut \\
\hline
leftpanel & Bandeau de gauche \\
\hline
detailpanel & Zone principale d'affichage, soit unique soit découpée en middlepanel et rightpanel\\
\hline
middlepanel & Zone centrale d'affichage quand un bandeau est présent à droite (rightpanel)\\
\hline
rightpanel & Bandeau de droite. Si non vide, alors affiché ainsi que middle panel\\
\hline
\end{tabular}

\subsection{Les objets graphiques définis}

L'affichage d'OBM est découpé selon les objets graphiques suivants, pour lesquels des styles sont définis :\\

\begin{tabular}{|c|c|p{8.8cm}|}
\hline
\textbf{Objet} & \textbf{Style} & \textbf{Description}\\
\hline
en-tête HTML & head & En-tête de définition du type de document, titre (<head>)\\
\hline
en-tête & header & Block d'en-tête contenant le logo, les infos de connexions\\
\hline
section & section & Block affichant les sections (onglets par exemple)\\
\hline
module & module & Block contenant les modules d'une section\\
\hline
action & action & Block contenant les actions disponibles du module\\
\hline
\multirow{4}{3cm}{bloc générique} & block & Bloc générique\\
\cline{2-3}
 & blockTitle & Titre du bloc \\
\cline{2-3}
 & blockItem & éléments du bloc \\
\hline
\multirow{4}{3cm}{dernière visite} & last & Bloc contenant les entités dernièrement visitées\\
\cline{2-3}
 & lastTitle & Titre du block \\
\cline{2-3}
 & lastItem & éléments de la liste des entités visitées \\
\hline
titre & title & Block contenant le titre de la page (ex: Société : Chercher)\\
\hline
message & msg & Block contenant la zone de message\\
\hline
\multirow{3}{3cm}{recherche} & search & Block contenant le formulaire de recherche\\
\cline{2-3}
 & searchForm & Donnée d'un champ de recherche \\
\cline{2-3}
 & searchLabel & Label d'un champ de recherche \\
\hline
\multirow{5}{3cm}{détail} & detail & Block contenant le détail d'une fiche d'un module\\
\cline{2-3}
 & detailHead & En-tête d'une fiche détail \\
\cline{2-3}
 & detailLabel & Label d'une fiche détail \\
\cline{2-3}
 & detailText & Donnée d'une fiche détail \\
\cline{2-3}
 & detailForm & Donnée d'un formulaire d'une fiche détail en modification \\
\hline
infos création & detailInfo & Utilisateurs et dates de création / modification\\
\hline
\multirow{2}{3cm}{boutons détail} & detailButton & Block contenant les boutons associés à une fiche détail\\
\cline{2-3}
 & detailButtons & Bouton d'une fiche détail \\
\hline
\multirow{3}{3cm}{link} & link & Block contenant les liens vers les entités associées\\
\cline{2-3}
 & linkTitle & Titre du block \\
\cline{2-3}
 & linkList & éléments de la liste des entités associées \\
\hline
\multirow{8}{3cm}{résultat de recherche} & result & Block contenant un résultat de recherche\\
\cline{2-3}
 & resultIndex & Block contenant l'index d'un résultat de recherche \\
\cline{2-3}
 & resultIndexIcon & icônes de l'index d'un résultat de recherche \\
\cline{2-3}
 & resultHead & Titre des colonnes de résultat \\
\cline{2-3}
 & resultText & Cellules du tableau de résultats \\
\cline{2-3}
 & resultTextC & Cellules centrées du tableau de résultats \\
\cline{2-3}
 & resultPref & table des préférences d'affichage d'un résultat de recherche \\
\cline{2-3}
 & resultPrefHide & Cellules des champs désactivés des préférences d'affichage \\
\hline
\multirow{4}{3cm}{formulaire d'admin} & admin & Block contenant un formulaire d'administration\\
\cline{2-3}
 & adminHead & titre du formulaire d'administration \\
\cline{2-3}
 & adminLabel & label d'un formulaire d'administration \\
\cline{2-3}
 & adminText & Donnée d'un formulaire d'administration \\
\hline
fin HTML & end & Cloture de la page\\
\hline
\end{tabular}

