% Documentation technique d'OBM : Appels de modules externes (ext_)
% ALIACOM Pierre Baudracco
% $Id$


\subsection{Appels de modules externes, popups}
\label{extmod}

L'appel de modules externes permet depuis un module d'interroger ou d'obtenir des informations d'un autre module.
Exemple : pour ajouter un utilisateur à un groupe, le module groupe fait appel au module utilisateur.\\

Afin de permettre une implémentation unique du module appelé qui sera utilisée par tous les modules le nécessitant, l'appel de modules externes est standardisé et répond aux objectifs suivants :\\
\begin{itemize}
\item Pas de dupplication de code inutile et lourde à maintenir
\item Le module appelé propose une interface indépendante du module appelant
\item Différents types d'informations peuvent être demandées (un Id, une liste d'Id)
\item Le module appelant doit récupérer les informations (par appel à une url ou dans des widgets)
\end{itemize}



\subsubsection{Description des différents appels externes}

Un appel externe est caractérisé par une action du module appelé.
3 types d'appels sont actuellement définis.
Pour chaque appel, divers paramètres sont à fournir.\\

\begin{tabular}{|p{3cm}|p{9.5cm}|}
\hline
\textbf{Action} & \textbf{Commentaire} \\
\hline
ext\_get\_ids & Affiche sans bandeau, en popup, la recherche des entités du module, permet d'effectuer des recherches et de renvoyer les entités sélectionnées par des cases à cocher à une url \\
\hline
ext\_get\_id & Affiche sans bandeau, en popup, la recherche des entités du module, permet d'effectuer des recherches et de renvoyer l'entité sélectionnée en cliquant dessus à une url ou dans un widget\\
\hline
ext\_get\_id\_url & Voir difference reelle avec ext\_get\_id ? uniquement param ext\_url ? \\
\hline
\end{tabular}


\subsubsection{Différenciation des types d'action (ext\_type)}

Afin de permettre aux fonctions d'affichage d'un module de différencier les affichages en fonction du type de retour (sélection d'une seule entrée, par lien, ou de plusieurs par case à cocher), les modules offrant les 2 possibilités doivent renseigner la variable \variable{ext\_type}.
Ceci s'effectue dans l'action en question, et doit être propagé dans les fonctions d'affichage (formulaire de recherche et affichage de liste de résultats).

\paragraph{Exemple} : le module contact offre des appels externes pour sélectionner un seul contact (sélection d'un contact pour un lead) ou plusieurs (cas de l'ajout de contacts à une liste).


\subsubsection{Spécifications de l'action ext\_get\_ids}

L'exemple donné est l'ajout d'utilisateurs à un groupe.
Le module appelant est \group, le module appelé est \user.
Depuis un groupe, on sélection l'ajout d'utilisateurs. Une fenêtre externe popup de recherche d'utilisateurs s'ouvre.\\

L'action ``Ajout d'utilisateurs'' doit donc être définie dans le module groupe comme un appel externe au module utilisateur.
Des paramètres doivent être passés.\\

\begin{tabular}{|p{2.5cm}|p{6cm}|p{4.5cm}|}
\hline
\textbf{Paramètre} & \textbf{Commentaire} & \textbf{Exemple} \\
\hline
\multicolumn{3}{|c|}{\textbf{Paramètres de configuration du module appelé}}\\
\hline
action & ext\_get\_ids & \\
\hline
popup & (0 | 1) affichage en popup (sans menu) & 1 \\
\hline
ext\_title & Titre affiché dans la fenêtre externe & Ajouter des utilisateurs\\
\hline
\multicolumn{3}{|c|}{\textbf{Paramètres : retour par url + action}}\\
\hline
ext\_action & Action appelée par la fenêtre externe & user\_add\\
\hline
ext\_target & Fenêtre cible de retour (target) & Groupe \\
\hline
ext\_url & Url appelée par la fenêtre externe & \$path/group/group\_index.php \\
\hline
ext\_id & Id de l'entité réceptionnant la réponse & Id du groupe origine \\
\hline
\multicolumn{3}{|c|}{\textbf{Retour par widget (select) : fonction javascript fill\_ext\_form() }}\\
\hline
ext\_widget & Indique le widget (select) devant sélectionner les données & \\
\hline
\multicolumn{3}{|c|}{\textbf{Retour par insertion d'éléments (div) : fonction javascript fill\_ext\_element() }}\\
\hline
ext\_element & Indique l'élément (widget) devant sélectionner les données & \\
\hline
\end{tabular}


\subsubsection{Spécifications de l'action ext\_get\_id}

L'exemple donné est la sélection d'une société lors de la création d'une affaire ou contact.
Le module appelant est \deal, le module appelé est \company.
Depuis le module \deal, on sélectionne la création d'une nouvelle affaire. Une fenêtre externe popup de recherche de société s'ouvre.\\

L'action ``Nouvelle'' doit donc être définie dans le module \deal comme un appel externe au module \company.
Des paramètres doivent être passés.\\

\begin{tabular}{|p{2.5cm}|p{6cm}|p{4.5cm}|}
\hline
\textbf{Paramètre} & \textbf{Commentaire} & \textbf{Exemple} \\
\hline
\multicolumn{3}{|c|}{\textbf{Paramètres de configuration du module appelé}}\\
\hline
action & ext\_get\_id & \\
\hline
popup & (0 | 1) affichage en popup (sans menu) & 1 \\
\hline
ext\_title & Titre affiché dans la fenêtre externe & Sélectionner une société\\
\hline
\multicolumn{3}{|c|}{\textbf{Retour par url : fonction javascript check\_get\_id\_url()}}\\
\hline
ext\_url & Url appelée par la fenêtre externe (l'url n'est pas fermée car elle recevra l'id de l'entité sélectionnée) & \$path/deal/deal\_index.php? action=new\&amp; param\_company= \\
\hline
\multicolumn{3}{|c|}{\textbf{Retour par widget : fonction javascript check\_get\_id()}}\\
\hline
ext\_widget & Indique le widget (hidden) devant recevoir l'Id & f\_contact.company\_new\_id \\
\hline
ext\_widget\_text & Indique le widget (textfield) devant recevoir le label ou nom correspondant à l'Id sélectionné & f\_contact.company\_new\_name \\
\hline
\end{tabular}


\subsubsection{Définition d'un appel à un module externe}

Ce peut être une action d'un menu (ex: ajouter utilisateur à groupe) ou un lien direct (ajouter une catégorie à une société depuis le formulaire de mise à jour société).

\paragraph{Appel par définition d'une action}
Exemple : Ajouter un utilisateur au groupe courant.\\

\shadowbox{
\begin{minipage}{14cm}
\begin{verbatim}
// Sel user add : Users selection
  $actions["GROUP"]["sel_user_add"] = array (
    'Name'     => $l_header_add_user,
    'Url'      => "$path/user/user_index.php?action=ext_get_ids&amp;popup=1
&amp;ext_title=".urlencode($l_add_user)."&amp;ext_action=user_add
&amp;ext_url=".urlencode($path."/group/group_index.php")."
&amp;ext_id=".$group["id"]."&amp;ext_target=$l_group",
    'Right'    => $cright_write,
    'Popup'    => 1,
    'Target'   => $l_group,
    'Condition'=> array ('detailconsult','user_add','user_del','group_add'...) 
                                    	  )
\end{verbatim}
\end{minipage}
}
\begin{itemize}
 \item \textbf{Popup} : indique si l'action doit ouvrir une fenêtre externe (1) ou se dérouler dans la fenêtre courante (défaut ou 0).\\
 \item \textbf{Target} : uniquement si Popup est positionné, indique le nom donné à la fenêtre source avant ouverture du Popup. Nécessaire dans le cas d'utilisation d'ext\_target pour identifier la fenêtre retour. Ces deux valeurs doivent être identiques.\\
\end{itemize}

\paragraph{Appel par lien direct}
Exemple : Ajouter des groupes au rendez-vous courant.\\

\shadowbox{
\begin{minipage}{14cm}
\begin{verbatim}
  $url2 = "$path/group/group_index.php?action=ext_get_ids&amp;popup=1
&amp;ext_widget=forms[0].elements[6]
&amp;ext_title=" . urlencode($l_agenda_select_group);
  ...
  $block = ``
     <a href=\"javascript: return false;\" onclick=\"popup('$url2',
''); return false;\">
<img src=\"/images/$set_theme/$ico_add_group\" /></a></td>
  ...``;
\end{verbatim}
\end{minipage}
}


\subsubsection{Modules implémentant des appels externes}

\begin{tabular}{|p{5cm}|c|c|}
\hline
\textbf{Module} & \textbf{Appel} & \textbf{Depuis version OBM} \\
\hline
\user & ext\_get\_ids & 0.7\\
\hline
\group & ext\_get\_ids & 0.8 \\
\hline
\resource & ext\_get\_ids & 1.0\\
\hline
\company & ext\_get\_id & 0.7 \\
\hline
\multirow{2}{5cm}{\contact} & ext\_get\_ids & 0.7 \\
 & ext\_get\_id & 2.1.0 \\
\hline
\deal & ext\_get\_id & 0.8.6 \\
\hline
\List & ext\_get\_id & 0.8 \\
\hline
\publication & ext\_get\_id & 0.8 \\
\hline
\project & ext\_get\_id & 0.8 \\
\hline
\contract & ext\_get\_id & 0.8 \\
\hline
\doc & ext\_get\_ids & 0.8 \\
\hline
\end{tabular}


\subsubsection{Implémentation dans un module}

\paragraph{Définition des actions}
Une action doit être déclarée pour être autorisée.
Il faut donc déclarer les actions externes.
Il n'y a pas besoin d'indiquer de titre, l'action ne disposant pas de menu (appel depuis un module externe).\\

\shadowbox{
\begin{minipage}{14cm}
\begin{verbatim}
// Ext get Ids : External User selection
  $actions["USER"]["ext_get_ids"] = array (
    'Right'    => $cright_read,
    'Condition'=> array ('none')
                                    );
\end{verbatim}
\end{minipage}
}



\paragraph{Affichage allégé (menu, liens)}
L'affichage des fenêtres externes, en popup, possède quelques particularités afin d'améliorer l'ergonomie d'utilisation :\\
\begin{itemize}
\item Pas de bandeaux ou menus
\item Pas de liens, données supplémentaires (formulaire de sélection,...)
\end{itemize}
\vspace{0.3cm}

Le bandeau ou menu général ne sera affiché que si le paramètre popup n'est pas positionné (module\_index.php).\\

\shadowbox{
\begin{minipage}{14cm}
\begin{verbatim}
if (! $obm_user["popup"]) {
  $display["header"] = display_menu($module);
}
\end{verbatim}
\end{minipage}
}
\vspace{0.3cm}

Les fonctions d'affichage utilisées par les appels externes sont les mêmes que les fonctions d'affichage classiques du module (ex: affichage de la liste des utilisateurs après une recherche).\\

Il faut donc que ces fonctions tiennent compte de ces contraintes. L'affichage sans liens est supporté par la classe d'affichage OBM\_DISPLAY.

De même, le code de génération du formulaire de récupération des données (ex: sélection des utilisateurs) est à créer quand nécessaire.\\

\shadowbox{
\begin{minipage}{14cm}
\begin{verbatim}
  $user_d = new OBM_DISPLAY("DATA", $pref_q, "user");
  if ($popup) {
    $user_d->display_link = false;
    $user_d->data_cb_text = "X";
    $user_d->data_idfield = "userobm_id";
    $user_d->data_cb_name = "data-u-";
    if ($ext_widget != "") {
      $user_d->data_form_head = "
      <form onsubmit=\"fill_ext_form(this); return false;\">";
    } else {
      $user_d->data_form_head = "
      <form target=\"$ext_target\" method=\"post\" action=\"$ext_url\">";
    }
    $user_d->data_form_end = "
      <p class=\"detailButton\">
        <p class=\"detailButtons\">
        <input type=\"submit\" value=\"$l_add\" />
        <input type=\"hidden\" name=\"ext_id\" value=\"$ext_id\" />
        <input type=\"hidden\" name=\"action\" value=\"$ext_action\" />
        </p>
      </p>
      </form>'';

    $display_popup_end = "
      <p>
      <a href=\"\" onclick='window.close();'>$l_close</a>
      </p>";
  }
\end{verbatim}
\end{minipage}
}


\paragraph{Transport des paramètres}

Lorsque une navigation est possible dans une fenêtre externe (ex: cas de recherche puis liste de résultat) les paramètres externes doivent être transmis afin d'être conservés.\\

\shadowbox{
\begin{minipage}{14cm}
\begin{verbatim}
  if ($popup) {
    $ext_action = $user["ext_action"];
    $ext_target = $user["ext_target"];
    $ext_url = $user["ext_url"];
    $ext_id = $user["ext_id"];
    $url_ext = "&amp;ext_action=$ext_action&amp;ext_url=$ext_url
&amp;ext_id=$ext_id&amp;ext_target=$ext_target";
  }

  $url = url_prepare("user_index.php?action=search&amp;tf_login=$login
&amp;tf_lastname=$lname&amp;sel_perms=$perms&amp;cba_archive=$archive$url_ext");
\end{verbatim}
\end{minipage}
}

\paragraph{Gestion du retour javascript}

Une fois la ou les entités sélectionnées dans la fenêtre externe, l'appel retour est réalisé soit par l'url et l'action indiquées, soit à l'aide de fonctions javascript, telles que référencées dans les tableaux de spécifications des actions ext\_get\_ids et ext\_get\_id.


\paragraph{action ext\_get\_ids, retour par widget}
Cette fonction est définie uniquement si le paramètre \$ext\_widget est rempli afin d'éviter des erreurs javascript.

\shadowbox{
\begin{minipage}{14cm}
\begin{verbatim}
if ($ext_widget != "") {
  $extra_js .= "

function fill_ext_form(int_form) {
   size = int_form.length;
   ext_field = window.opener.document.$ext_widget;
   for(i=0; i <size ; i++) {
     if(int_form.elements[i].type == 'checkbox'){
       if(int_form.elements[i].checked == true) {
	 ext_size = ext_field.length;
	 for(j=0; j< ext_size; j++) {
	   if('cb_g' + ext_field.options[j].value == int_form.elements[i].name) {
	     window.opener.document.$ext_widget.options[j].selected =true;
	   }
	 }
       }
     }
   }
}";
}
\end{verbatim}
\end{minipage}
}


\paragraph{action ext\_get\_id, retour par widget}
Cette fonction est définie uniquement si les paramètres \$ext\_widget sont remplis afin d'éviter des erreurs javascript.

\shadowbox{
\begin{minipage}{14cm}
\begin{verbatim}
if (($ext_widget != "") || ($ext_widget_text != "")) {
  $extra_js .= "

function check_get_id(valeur,text) {
  if ((valeur < 1) || (valeur == null)) {
    alert (\"$l_j_select_company\");
    return false;
  } else {
    window.opener.document.$ext_widget.value=valeur;
    window.opener.document.$ext_widget_text.value=text;
    window.close();
    return true;
  }
}";
}
\end{verbatim}
\end{minipage}
}


\paragraph{action ext\_get\_id, retour par url}
L'id de l'entité sélectionnée est ajoutée à l'url de retour.

\shadowbox{
\begin{minipage}{14cm}
\begin{verbatim}
function check_get_id_url(p_url, valeur) {
  if ((valeur < 1) || (valeur == null)) {
    alert (\"$l_j_select_company\");
    return false;
  } else {
    new_url = p_url + valeur;
    window.opener.location.href=new_url;
    window.close();
    return true;
  }
}

\end{verbatim}
\end{minipage}
}
