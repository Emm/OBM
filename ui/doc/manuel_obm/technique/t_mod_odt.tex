% Documentation technique d'OBM : Module Odt
% ALIACOM Vincent Coulette
% $Id$

\clearpage
\section{Odt}

Depuis la version 1.2, \obm intègre un module d'export openOffice.

\obm utilise actuellement la libraire tbsOOo, dérivée de tbs (TinyButStrong) pour l'export openOffice.
Le module \odt propose une couche d'abstraction d'utilisation de l'export openOffice.\\

Le module \odt est particulier car il n'apparait pas dans les menus de l'application mais est un service pour les modules devant faire appel à l'export odt.
Le module \odt est appelé via un formulaire, il se charge de traiter les données reçues et renvoie le document odt (ou sxw) intégrant les données.
Il peut aussi se contenter de sauvegardé le document exporté à l'adresse spécifiée.

Il n'effectue pas de gestion des droits ou de session.

\subsection{Paramètres et utilisation}

Pour l'instant, une seule action est implémentée, l'export simple.
Pour utiliser le module il suffit de l'appeler via un formulaire POST en fournissant les paramètres suivants : \\

\begin{tabular}{|p{3cm}|p{5cm}|p{5cm}}
\hline
\textbf{Paramètre} & \textbf{Description} & \textbf{Commentaire} \\
\hline
action & Action choisie (mode d'export) & Pour l'instant seule l'action export est implémentée (export basique) \\
\hline
template & Id du document utilisé comme modèle & Modèle sélectionné parmi les modèles par défaut ou les documents \obm \\
\hline
save\_path & Chemin de sauvegarde du document à exporter & Paramètre optionnel (si ce paramètre est laissé vide, le document est affiché) \\
\hline
data &  Données spécifiques du document à exporter & Formattage décrit dans la section suivante \\
\hline
\end{tabular}
\\

Il y a donc deux modes d'export : \\
\begin{itemize}
 \item si le paramètre save\_path n'est pas spécifié, alors le document exporté devient disponbible à l'écran sous la forme d'une boîte de dialogue ouvrir/enregistrer le document.
 \item sinon le document est sauvegardé à l'adresse indiquée (si les droits d'écriture sont valables pour cette adresse).
\end{itemize}

Enfin, le type mime du document est déterminé la type mime du document \obm utilisé : il est donc important de bien spécifier le type (odt ou  sxw) lors de l'upload d'un document. Le type par défaut est sxw (pour garantir la compatibilité avec openOffice 1.XX). \\

\subsection{Formattage des données}


La formattage des données doit correspondre au modèle utilisé : ainsi la fonction de formattage est généralement spécifique à chaque entité puisqu'on souhaite afficher des données différentes. \\

\subsubsection{Formattage du modèle openOffice}

Les données à afficher dans le modèle sont définies selon la syntaxe de tbs, dont le manuel se trouve à l'adresse \variable{http://www.tinybutstrong.com/manual.php} ainsi qu'à quelques spécifités de tbsOOo dont le manuel se trouve à l'adresse \variable{http://www.tinybutstrong.com/tbsooo.php}. \\

Les deux principales règles de syntaxe sont : 
\begin{itemize}
 \item pour une variable simple \\
\shadowbox{
\begin{minipage}{16cm}
\begin{verbatim}
  [var.nom_variable]
\end{verbatim}
\end{minipage}
}
 \item pour un bloc de variables (liste, ou série) \\
\shadowbox{
\begin{minipage}{16cm}
\begin{verbatim}
  [nom_bloc;block=begin]
      [nom_bloc.variable1]
      [nom_bloc.variable2]
      [.....]
  [nom_bloc;block=end]
\end{verbatim}
\end{minipage}
}
\end{itemize}

\subsubsection{Formattage des données (variables PHP)}

Les données suivent un formattage spécial pour pouvoir être exploitées de manière uniforme par le module \odt. \\
On distingue deux types de données : 
\begin{itemize}
 \item les variables simples, qui ne sont affichées qu'une fois : titres des rubriques du document, noms de la personne titulaire du document..... Elles sont stockées dans le tableau \variable{data\_vars}.
 \item les variables de block dont on ne connaît pas le nombre à l'avance : ce sont les variables correspondant à des listes d'éléments : par exemple, des références pour un CV (plusieurs variables par référence à chaque fois)... Les tableaux représentant un bloc de données sont stockées dans le tableau \variable{data\_blocks}.
\end{itemize}

Ainsi la structure de données à fournir au module odt est la suivante : \\
\shadowbox{
\begin{minipage}{16cm}
\begin{verbatim}
$data = { 
    data_vars => { variable_simple_1, variable_simple_2, ..., variable_simple_n }, 
    data_blocks => { 
           block_1 => { variable_block_1_1, variable_block_1_2, ...., variable_block_1_n },
           block_2 => { variable_block_2_1, variable_block_2_2, ...., variable_block_2_n },
           ......,
           block_2 => { variable_block_n_1, variable_block_n_2, ...., variable_block_n_n }
                   }
        }
\end{verbatim}
\end{minipage}
}
\\

Enfin le bloc de données doit être sérialisé, avec la fonction \fonction{serialize} puis encodé avec la fonction \fonction{urlencode}. Lors de la réception des données, le module odt effectue les opérations inverses.  \\



