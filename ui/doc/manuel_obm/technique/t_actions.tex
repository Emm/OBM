% Documentation technique d'OBM : Actions d'un module
% ALIACOM Pierre Baudracco
% $Id$


\subsection{Les actions d'un module}
\label{actions}

Les actions d'un module représentent les traitements implémentés par ce module. Chaque traitement (consultation, création, modification, recherche,...) est défini par une action.

Chaque action possède des propriétés qui vont définir son comportement (droits d'accès, contexte d'affichage,...).


\subsubsection{Gestion des actions par un module}

Les différentes actions proposées par un module sont définies dans le module (fichier \fichier{module\_index.php}.

\begin{itemize}
\item La fonction \fonction{get\_module\_action()} appelée en début de module renseigne le tableau global \variable{\$actions}.

\item La fonction \fonction{update\_module\_action()} appelée en fin de module, après les traitements, permet de mettre à jour les actions en fonction du contexte et des résultats du traitement.
\end{itemize}

\subsubsection{Utilité de \fonction{update\_module\_action()}}

\paragraph{Exemple :} pour l'action insertion :
\begin{itemize}
\item si l'action est un succès, une nouvelle entité est créée et le module l'affiche directement. Dans ce cas les actions ``modifier'' ou ``supprimer'', qui nécessitent d'être positionné sur une entité, peuvent être affichées
\item si l'action échoue, n'étant pas positionné sur une entité, ces actions ne doivent pas être affichées.
\end{itemize}

\paragraph{Note :} l'affichage du bandeau est effectué après l'appel à \fonction{update\_module\_action()}.


\subsubsection{Schéma de principe global de gestion des actions}

\shadowbox{
\begin{minipage}{13cm}
\begin{verbatim}
get_module_action();

// Traitements
// if ($action == ...)
// else if (...
// }
// Fin Traitements

update_module_action(); // maj des actions
$display[``header''] = display_menu($module) // affichage du bandeau
\end{verbatim}
\end{minipage}
}


\subsubsection{Propriétés d'une action}

\begin{tabular}{|p{2cm}|p{5cm}|p{6cm}|}
\hline
\textbf{Nom} & \textbf{description} & \textbf{Valeur} \\
\hline
Name & label associé & \$l\_header\_find \\ 
\hline
Url & url de l'action & \$path/company/company\_index.php?action=.. \\ 
\hline
Right & Droit nécessaire pour exécution & \$cright\_read, \$cright\_write,.. \\ 
\hline
Condition & Condition d'affichage dans menu & array : 'all', 'None', 'action' \\ 
\hline
\end{tabular}

