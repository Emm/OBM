% Documentation technique d'OBM : Actions d'un module
% ALIACOM Pierre Baudracco
% $Id$


\subsection{Les actions d'un module}
\label{actions}

Les actions d'un module repr�sentent les traitements impl�ment�s par ce module. Chaque traitement (consultation, cr�ation, modification, recherche,...) est d�fini par une action.

Chaque action poss�de des propri�t�s qui vont d�finir son comportement (droits d'acc�s, contexte d'affichage,...).


\subsubsection{Gestion des actions par un module}

Les diff�rentes actions propos�es par un module sont d�finies dans le module (fichier \fichier{module\_index.php}.

\begin{itemize}
\item La fonction \fonction{get\_module\_action()} appel�e en d�but de module renseigne le tableau global \variable{\$actions}.

\item La fonction \fonction{update\_module\_action()} appel�e en fin de module, apr�s les traitements, permet de mettre � jour les actions en fonction du contexte et des r�sultats du traitement.
\end{itemize}

\subsubsection{Utilit� de \fonction{update\_module\_action()}}

\paragraph{Exemple :} pour l'action insertion :
\begin{itemize}
\item si l'action est un succ�s, une nouvelle entit� est cr��e et le module l'affiche directement. Dans ce cas les actions ``modifier'' ou ``supprimer'', qui n�cessitent d'�tre positionn� sur une entit�, peuvent �tre affich�es
\item si l'action �choue, n'�tant pas positionn� sur une entit�, ces actions ne doivent pas �tre affich�es.
\end{itemize}

\paragraph{Note :} l'affichage du bandeau est effectu� apr�s l'appel � \fonction{update\_module\_action()}.


\subsubsection{Sch�ma de principe global de gestion des actions}

\shadowbox{
\begin{minipage}{13cm}
\begin{verbatim}
get_module_action();

// Traitements
// if ($action == ...)
// else if (...
// }
// Fin Traitements

update_module_action(); // maj des actions
$display[``header''] = display_menu($module) // affichage du bandeau
\end{verbatim}
\end{minipage}
}


\subsubsection{Propri�t�s d'une action}

\begin{tabular}{|p{2cm}|p{5cm}|p{6cm}|}
\hline
\textbf{Nom} & \textbf{description} & \textbf{Valeur} \\
\hline
Name & label associ� & \$l\_header\_find \\ 
\hline
Url & url de l'action & \$path/company/company\_index.php?action=.. \\ 
\hline
Right & Droit n�cessaire pour ex�cution & \$cright\_read, \$cright\_write,.. \\ 
\hline
Condition & Condition d'affichage dans menu & array : 'all', 'None', 'action' \\ 
\hline
\end{tabular}

