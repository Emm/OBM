% Documentation technique d'OBM : module To Do
% ALIASOURCE Nourdine Bouaghaz
% $Id$


\clearpage
\section{Todo}
révision : \obm 2.0

Le module \todo d'\obm.

\subsection{Organisation de la base de données}

Le module \todo utilise 1 table :
\begin{itemize}
 \item Todo
\end{itemize}

\subsection{Todo}
Table principale des informations d'une tâche planifiée .\\

\begin{tabular}{|p{3cm}|c|p{5.4cm}|p{2.6cm}|}
\hline
\textbf{Champs} & \textbf{Type} & \textbf{Description} & \textbf{Commentaire} \\
\hline
\_id & int 8 & Identifiant & Clé primaire \\
\hline
\_timeupdate & timestamp 14 & Date de mise à jour & \\
\hline
\_timecreate & timestamp 14 & Date de création & \\
\hline
\_userupdate & int 8 & Id du modificateur de la tâche & \\
\hline
\_usercreate & int 8 & Id du créateur de la tâche & \\
\hline
\_user & int 8 & Id du créateur de la tâche & \\
\hline
\_date & timestamp & Date de création de la tâche & \\
\hline
\_dateend & timestamp & Date de fin de la tâche planifiée & \\
\hline
\_deadline & timestamp & Date limite de fin de tâche & \\
\hline
\_title & varchar 80 & Titre de la tâche & (de 1 à 5)\\
\hline
\_priority & int 8 & Priorité de la tâche & (de 1 à 5)\\
\hline
\_status & varchar 32 & Etat de la tâche & \\
\hline
\_webpage & varchar 255 & Date de fin de projet & \\
\hline
\_content & text & Commentaire &\\
\hline
\end{tabular}


\subsection{Actions et droits}

Voici la liste des actions du module \todo, avec le droit d'accès requis ainsi qu'une description sommaire de chacune d'entre elles.\\

\begin{tabular}{|l|c|p{9.5cm}|}
 \hline
 \textbf{Intitulé} & \textbf{Droit} & \textbf{Description} \\
 \hline
  index & read & (Défaut) formulaire de recherche des tâches planifiées. \\ 
 \hline
  detailconsult & read & Fiche détail d'une tâche planifiée. \\
 \hline
  detailupdate & write & Formulaire de modification d'une tâche planifiée. \\
 \hline
  insert & write & Insertion d'une tâche planifiée. \\
 \hline
  update & write & Mise à jour d'une tâche planifiée. \\
 \hline
  delete & write & Suppression d'une tâche planifiée. \\
 \hline
  dispref\_display & read &  Modification de l'affichage d'un élément.\\
 \hline
  display & read & Ecran de modification des préférences d'affichage. \\
 \hline
  dispref\_level & read & Modification de l'ordre d'affichage d'un élément. \\
 \hline
  delete\_unique & write &  \\
 \hline
\end{tabular}
