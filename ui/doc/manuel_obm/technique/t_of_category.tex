% Documentation technique d'OBM : Gestion des catégories
% ALIACOM Pierre Baudracco
% $Id$


\subsection{Gestion des catégories}
\label{of_category}

révision : \obm 2.0.0

\obm gère des catégories monovaluées ou multivaluées (n-n) pour plusieurs modules : \company, \contact, \deal... comme les secteurs d'activité, les rôles de société, les types de société...

Toutes ces ``catégories'' fonctionnent selon le même principe et possèdent le même format de stockage.

\paragraph{Remarque :} les ``catégories'' traitées ici sont les catégories intégrées à \obm générique. Les fonctionnalités listées ont pour objectif d'aider à la mutualisation de code. Les ``catégories'' définies et spécifiques aux clients sont traitées dans la section \ref{of_userdata}.

Cette librairie propose une gestion générique des catégories avec liaisons 1-n -mode = mono) ou n-n (mode = multi) offrant :\\

\begin{itemize}
\item Gestion de catégories avec codes (1.2.10, 3.2,...) et labels
\item Fonctions de gestion des champs de recherche de catégorie
\item Fonctions de gestion des champs catégorie dans les formulaires
\item Affichage en mode arborescence, avec sélection simple ou multiple
\end{itemize}
\vspace{0.3cm}

Les fonctionalités de catégories sont fournies par le fichier \fichier{of/of\_category.inc}


\subsubsection{Types de catégories}

\obm distingue 2 types de catégories :\\
\begin{itemize}
\item \textbf{mono} : catégories avec liaisons 1-n. L'entité n'est liée qu'à 0 ou 1 des catégories
\item \textbf{multi} : catégories avec liaisons n-n. L'entité peut être liée à 0, 1  ou n catégories.
\end{itemize}

\subsubsection{Modèle de stockage générique}

Pour une catégorie ``mono'' d'un module, l'entité du module doit contenir un champ référençant la catégorie \variable{entite\_category\_id} (ex \variable{company\_type\_id})et 1 table doit être créée :\\

\begin{itemize}
\item ModuleCategory (ex: CompanyType)
\end{itemize}
\vspace{0.3cm}

Pour une catégorie ``multi'' d'un module, 2 tables doivent être créées :\\

\begin{itemize}
\item ModuleCategory (si besoin d'une table catégorie)
\item ModuleCategoryLink (ou table de liaison)
\end{itemize}
\vspace{0.3cm}

Table des informations d'une catégorie, ModuleCategoryN (ex: CompanyType) :\\

\begin{tabular}{|p{3cm}|c|p{4cm}|p{4cm}|}
\hline
\textbf{Champs} & \textbf{Type} & \textbf{Description} & \textbf{Commentaire} \\
\hline
\_id & int 8 & Id de la catégorie & \\
\hline
\_code & varchar 10 & Code de la catégorie & ex: 1.0, 1.1, 1.1.4 \\
\hline
\_label & varchar 100 & Label de la catégorie & \\
\hline
\end{tabular}
\vspace{0.3cm}

Table des informations de liaison d'une catégorie avec un module, ModuleCategoryNLink (ex: CompanyCategory1Link) :\\

\begin{tabular}{|p{3cm}|c|p{4cm}|}
\hline
\textbf{Champs} & \textbf{Type} & \textbf{Description}\\
\hline
\_category\_id & int 8 & Id de la catégorie\\
\hline
\_company\_id & int 8 & Id de l'entité du module\\
\hline
\end{tabular}


\subsubsection{Implémentation : API fonctions publiques}

\paragraph{Fonctions d'affichage} : \\

\shadowbox{
\begin{minipage}{14cm}
\begin{verbatim}
of_category_dis_search_select($entity, $cat, $cats, $cat_id,
$first_element="all",$sel_append="") {
\end{verbatim}
\end{minipage}
}

\begin{tabular}{|p{3cm}|p{10cm}|}
\hline
\textbf{Paramètres} & \textbf{Description}\\
\hline
\multicolumn{2}{|c|}{Fonction d'affichage du champ de recherche de la catégorie pour le formulaire de recherche}\\
\hline
\$entity & entité (ex: ``company'', ``contact'',..)\\
\hline
\$cat & catégorie (ex: ``catégorie1, ``type'',..)\\
\hline
\$cats & tableau des catégories à afficher\\
\hline
\$cat\_id & entrée sélectionnée\\
\hline
\$first\_element & scalaire ou tableau (none|all) d'éléments à ajouter au <select>\\
\hline
\$sel\_append & chaine à insérer dans la définition du <select> (ex: code javascript)\\
\hline
\hline
\textbf{Retour} & \textbf{Description}\\
\hline
String XHTML & Block de recherche <select> du champ catégorie pour l'entité\\
\hline
\end{tabular}
\vspace{0.4cm}


\shadowbox{
\begin{minipage}{14cm}
\begin{verbatim}
of_category_dis_entity_form($entity, $cat, $cats, $mode="multi", $cat_id="", $first="") {
\end{verbatim}
\end{minipage}
}

\begin{tabular}{|p{3cm}|p{10cm}|}
\hline
\textbf{Paramètres} & \textbf{Description}\\
\hline
\multicolumn{2}{|c|}{Fonction d'affichage du champ catégorie pour le formulaire du module \$entity}\\
\hline
\$entity & entité (ex: ``company'', ``contact'',..)\\
\hline
\$cat & catégorie (ex: ``catégorie1, ``type'',..)\\
\hline
\$cats & tableau des catégories à afficher\\
\hline
\$mode & ``mono'' ou ``multi'' pour indiquer le type de liaison vers l'entité\\
\hline
\$cat\_id & pour mode ``mono'' entrée catégorie sélectionnée\\
\hline
\$first & pour mode ``mono'' scalaire ou tableau (none|all) d'éléments à ajouter au <select>\\
\hline
\hline
\textbf{Retour} & \textbf{Description}\\
\hline
String XHTML & Ligne de tableau de formulaire : champ catégorie : select simple ou d'entité\\
\hline
\end{tabular}
\vspace{0.4cm}


\shadowbox{
\begin{minipage}{14cm}
\begin{verbatim}
of_category_dis_block_consult($entity, $cat, $cats, $mode="multi") {
\end{verbatim}
\end{minipage}
}

\begin{tabular}{|p{3cm}|p{10cm}|}
\hline
\textbf{Paramètres} & \textbf{Description}\\
\hline
\multicolumn{2}{|c|}{Fonction d'affichage du champ catégorie en consultation du module \$entity}\\
\hline
\$entity & entité (ex: ``company'', ``contact'',..)\\
\hline
\$cat & catégorie (ex: ``catégorie1, ``type'',..)\\
\hline
\$cats & tableau des catégories à afficher\\
\hline
\$mode & ``mono'' ou ``multi'' pour indiquer le type de liaison vers l'entité\\
\hline
\hline
\textbf{Retour} & \textbf{Description}\\
\hline
String XHTML & Ligne de tableau (mono) ou block de consultation (multi) du champ catégorie\\
\hline
\end{tabular}
\vspace{0.4cm}


\shadowbox{
\begin{minipage}{14cm}
\begin{verbatim}
of_category_dis_tree($entity, $cat, $params, $action) {
\end{verbatim}
\end{minipage}
}

\begin{tabular}{|p{3cm}|p{10cm}|}
\hline
\textbf{Paramètres} & \textbf{Description}\\
\hline
\multicolumn{2}{|p{13cm}|}{Affichage de l'arborescence de la catégorie ``\$cat'' pour l'entité (souvent module) \$entity.
Si l'action demandée est du type ext\_get\_cat[1-9]?\_ids, l'affichage propose une sélection par checkbox, sinon une sélection directe unique par lien.}\\
\hline
\$entity & entité (ex: ``company'', ``contact'',..)\\
\hline
\$cat & catégorie (ex: ``catégorie1, ``type'',..)\\
\hline
\$params & hachage des informations du module [popup], [ext\_title]\\
\hline
\$action & action du module : si de type "ext\_get\_category[1-9]?\_ids", affichage de sélection multiple (par checkbox), sinon sélection par lien simple\\
\hline
\hline
\textbf{Retour} & \textbf{Description}\\
\hline
String XHTML & Arborescence complète de la catégorie avec sélection\\
\hline
\end{tabular}
\vspace{0.4cm}


\shadowbox{
\begin{minipage}{14cm}
\begin{verbatim}
of_category_dis_admin_form($cat, $cats, $user=false) {
\end{verbatim}
\end{minipage}
}

\begin{tabular}{|p{3cm}|p{10cm}|}
\hline
\textbf{Paramètres} & \textbf{Description}\\
\hline
\multicolumn{2}{|p{13cm}|}{Affichage du formulaire de gestion de la catégorie. Donne les possibilités d'ajouter, de modifier ou de supprimer une entrée.}\\
\hline
\$cat & catégorie (ex: ``catégorie1, ``type'',..)\\
\hline
\$cats & tableau des catégories à afficher\\
\hline
\$user & (utilisation interne pour signaler une categorie utilisateur : true)\\
\hline
\hline
\textbf{Retour} & \textbf{Description}\\
\hline
String XHTML & Formulaire de gestio nde la catégorie\\
\hline
\end{tabular}
\vspace{0.4cm}


\paragraph{Fonctions bases de données} : \\

\shadowbox{
\begin{minipage}{14cm}
\begin{verbatim}
of_category_get_ordered($entity, $catu, $mode="multi", $entity_id=0) {
\end{verbatim}
\end{minipage}
}

\begin{tabular}{|p{3cm}|p{10cm}|}
\hline
\textbf{Paramètres} & \textbf{Description}\\
\hline
\multicolumn{2}{|c|}{Récupération de la liste des catégories ordonnées selon le code}\\
\hline
\$entity & entité (ex: ``company'', ``contact'', ``DealCompany''..)\\
\hline
\$catu & catégorie (ex: ``catégorie1, ``type'',..) can be uppercase\\
\hline
\$mode & ``mono'' ou ``multi'' pour indiquer le type de liaison vers l'entité\\
\hline
\$entity\_id & pour mode ``multi''. Si donné, uniquement entrées liées à cette entité\\
\hline
\hline
\textbf{Retour} & \textbf{Description}\\
\hline
Array & Tableau de catégories ordonnées selon le code : [id], [code], [label]\\
\hline
\end{tabular}
\vspace{0.4cm}

\shadowbox{
\begin{minipage}{14cm}
\begin{verbatim}
of_category_query_insert($entity, $catu, $params) {
\end{verbatim}
\end{minipage}
}

\shadowbox{
\begin{minipage}{14cm}
\begin{verbatim}
of_category_query_insert_link($entity, $catu, $entity_id, $cat_id) {
\end{verbatim}
\end{minipage}
}

\shadowbox{
\begin{minipage}{14cm}
\begin{verbatim}
of_category_query_update($entity, $catu, $params) {
\end{verbatim}
\end{minipage}
}

\shadowbox{
\begin{minipage}{14cm}
\begin{verbatim}
of_category_query_delete($entity, $catu, $params) {
\end{verbatim}
\end{minipage}
}

\shadowbox{
\begin{minipage}{14cm}
\begin{verbatim}
of_category_query_delete_link($entity, $cat, $entity_id) {
\end{verbatim}
\end{minipage}
}

\shadowbox{
\begin{minipage}{14cm}
\begin{verbatim}
of_category_query_links($entity, $cat, $p_id, $mode=''multi'') {
\end{verbatim}
\end{minipage}
}

\shadowbox{
\begin{minipage}{14cm}
\begin{verbatim}
of_category_get_entitycategories($entity, $cat, $entity_id, $mode=''multi'') {
\end{verbatim}
\end{minipage}
}

\shadowbox{
\begin{minipage}{14cm}
\begin{verbatim}
of_category_query_category_per_entity($entity, $cat, $mode=''multi'') {
\end{verbatim}
\end{minipage}
}
