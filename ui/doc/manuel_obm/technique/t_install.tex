\section{Installation pas à pas d'OBM 2.0.x}

Cette documentation est valide pour la version 2.0. Seules les
manipulations spécifiques à OBM sont décrites. Pour l'installation de
PHP, MySQL, PostgreSQL, Apache ou autres composants libres, des exemples sont proposés correspondant à la version Debian mais il est préférable de vous référer à la documentation de votre distribution.


\subsection{Hypothèses}

Cette documentation suppose que vous disposez d'un serveur Apache,
d'un serveur de BD MySQL ou PostgreSQL et de PHP correctement configurés. 

Sous Debian, les packages suivants doivent être installés :
\begin{description}
\item[packages php5 ] php5, php5-cgi, php5-mysql ou php5-pgsql (ou, pour php4 : php4, php4-cgi, php4-mysql ou php4-pgsql)
\item[packages Apache 2 ] apache2 et apache2-common (ou, pour Apache 1 : apache et apache-common)
\end{description}

Dans la suite du document, on suppose que la racine d'OBM se trouve
dans \fichier{/var/www/obm}.

\subsection{Télécharger les sources}

L'installation serait plus difficile sans...

Il faut donc télécharger les sources et les décompresser dans
\fichier{/var/www} (en fait dans le répertoire qui contiendra la
racine de votre OBM).

\shadowbox{
  \begin{minipage}{13cm}
\begin{verbatim}
wget http://obm.aliacom.fr/obm-archives/obm-1.2.2.tar.gz
tar -C /var/www -xzvf obm-1.2.2.tar.gz
cd /var/www
ln -s obm-1.2.2 obm
\end{verbatim}
  \end{minipage}
}

Changer le propriétaire des fichiers sources d'OBM (nécessaire pour le
safe\_mode activé dans le module documents). 

\shadowbox{
  \begin{minipage}{13cm}
\begin{verbatim}
cd /var/www/obm
chown -R www-data:www-data *  (sur Debian)
chown -R apache:apache *  (sur Redhat)
\end{verbatim}
  \end{minipage}
}

\subsection{configuration de PHP}

Fichier de configuration de PHP : 

\begin{description}
\item[sur Debian] \fichier{/etc/php5/apache2/php.ini} (ou \fichier{/etc/php4/apache/php.ini} pour php4 et Apache 1)
\item[sur RedHat] \fichier{/etc/php.ini}
\end{description}

La version ligne de commande de PHP doit également être configurée
(certains outils d'administration ou certains scripts d'installation
sont aussi utilisables dans ce mode de fonctionnement).

\begin{description}
\item[sur debian] \fichier{/etc/php5/cgi/php.ini}, ou 
\fichier{/etc/php5/cli/php.ini} sur les version récentes (respectivement \fichier{/etc/php4/cgi/php.ini}, ou \fichier{/etc/php4/cli/php.ini} pour php 4).
\end{description}


La directive include\_path doit être renseignée ainsi (ou contenir au moins ces chemins) : 

\shadowbox{
  \begin{minipage}{13cm}
\begin{verbatim}
include_path = ".:/var/www/obm"
\end{verbatim}
  \end{minipage}
}

Depuis la version 0.7.3, la directive short\_open\_tag peut être
positionnée à Off. Pour les versions antérieures, elle doit toutefois
être positionnée à On.

\shadowbox{
  \begin{minipage}{13cm}
\begin{verbatim}
short_open_tag = On (can be switched to Off since OBM 0.7.3)
\end{verbatim}
  \end{minipage}
}

Les variables globales doivent être automatiquement enregistrées : 

\shadowbox{
  \begin{minipage}{13cm}
\begin{verbatim}
register_globals = On
\end{verbatim}
  \end{minipage}
}

La directive Magic\_quote doit être à On :

\shadowbox{
  \begin{minipage}{13cm}
\begin{verbatim}
magic_quotes_gpc = On
\end{verbatim}
  \end{minipage}
}

De plus, selon  le propriétaire des fichiers, il faut aussi activer le
safe\_mode (mais OBM peut fonctionner sans également) : 

\shadowbox{
  \begin{minipage}{13cm}
\begin{verbatim}
safe_mode = On
\end{verbatim}
  \end{minipage}
}

\subsubsection{Vérification du support Mysql (ou PostgreSQL) par PHP}

Selon le serveur de base de données utilisé, une de ces deux lignes
doit être présente dans votre \fichier{php.ini} : 

\shadowbox{
  \begin{minipage}{13cm}
\begin{verbatim}
extension=mysql.so 
extension=pgsql.so
\end{verbatim}
  \end{minipage}
}

Cette modification doit être faite pour le module Apache
\emph{et} pour le mode ligne de commande s'ils utilisent des fichiers
\fichier{php.ini} distincts.

\subsection{Configuration d'Apache et de PHP}

Les sources d'OBM sont dans le répertoire \fichier{php}, les fichiers
à inclure sont dans le répertoire \fichier{obminclude}. Depuis la
version 0.5.2, le répertoire \fichier{obminclude} est en dehors de la
racine des documents pour des considérations de sécurité.

Pour l'application web OBM, toutes les configurations Apache et PHP se
font dans le fichier \fichier{httpd.conf} d'Apache.

\subsubsection*{Configurer Apache pour le traitement des fichiers PHP}

Attention, les différents chemins et fichiers peuvent varier entre les distributions et les versions d'apache (apache 1 et 2).

\paragraph{Définition du type de fichier .php}
Éditer le fichier \fichier{apache2.conf} (ou \fichier{httpd.conf} pour Apache 1) et
ajouter :

\shadowbox{
  \begin{minipage}{13cm}
\begin{verbatim}
AddType application/x-httpd-php .php
\end{verbatim}
  \end{minipage}
}

\paragraph{Inclure le module php dans Apache}
Éditer le fichier \fichier{mods-available/php5.load} (ou \fichier{modules.conf} pour Apache 1) et
ajouter :

\shadowbox{
  \begin{minipage}{13cm}
\begin{verbatim}
LoadModule php5_module /usr/lib/apache2/modules/libphp5.so
\end{verbatim}
  \end{minipage}
}

ou (pour php4 et Apache 1) :

\shadowbox{
  \begin{minipage}{13cm}
\begin{verbatim}
LoadModule php4_module /usr/lib/apache/1.3/libphp4.so
\end{verbatim}
  \end{minipage}
}

Sous RedHat 8, avec Apache2, vérifier que le
\fichier{/etc/httpd/conf.d/php.conf} est correct.

Nous allons  configurer un \emph{virtual host} pour gérer une instance
d'OBM (vous pouvez avoir plusieurs \emph{virtuals hosts} sur un seul
serveur). 

Créer un fichier \fichier{obm} dans le répertoire \fichier{/etc/apache2/sites-available}
(ou modifier le fichier \fichier{httpd.conf}, ans la section \emph{virtual host}, pour 
Apache 1).

A l'intérieur de ce fichier, ajoutez les balises :

\shadowbox{
  \begin{minipage}{13cm}
\begin{verbatim}
<VirtualHost alias_ou_url_ou_ip_obm>
</VirtualHost>
\end{verbatim}
  \end{minipage}
}

Les modifications suivantes sont à apporter entre ces deux balises.

Positionner le \emph{Document Root} à /var/www/obm/php.

\shadowbox{
  \begin{minipage}{13cm}
\begin{verbatim}
DocumentRoot /var/www/obm/php
\end{verbatim}
  \end{minipage}
}

\subsubsection*{Répertoires d'include d'OBM}

Le nom du répertoire d'include (\fichier{obminclude} par défaut) est
maintenant une variable d'environnement pour permettre à plusieurs
instances d'OBM basées sur le même répertoire de source de tourner
simultanément. Seul ce répertoire est à modifier sur chaque instance
car il contient les réglages spécifiques (thèmes, bases de données,
langues) pour chaque instance.

Cette variable est donc positionnée dans le fichier du virtual host
d'obm \fichier{sites-available/obm} (ou dans le fichier \fichier{httpd.conf}
pour Apache 1).

Le module \fichier{env} est nécessaire pour positionner cette variable.
Sous debian, ce module est déjà compilé dans le noyau de apache 2. Il est
possible de le vérifier avec la commande \variable{apache2 -l} (dans la
liste des modules renvoyée, il doit y avoir mod\_env.c).

Si le module \fichier{mod\_env} n'est pas compilé dans la version installée
d'apache, il faut le charger. Sous apache 2, vous pouvez utilisez la commande
\variable{a2enmod env}. Sinon, dans le fichier \fichier{apache2.conf} (ou
\fichier{httpd.conf}) :

\shadowbox{
  \begin{minipage}{13cm}
\begin{verbatim}
LoadModule env_module /usr/lib/apache2/modules/mod_env.so (apache 2 sous Debian)
LoadModule env_module /usr/lib/apache/1.3/mod_env.so (apache 1.3 sous Debian)
LoadModule env_module modules/mod_env.so (apache 2 sous Redhat)
\end{verbatim}
  \end{minipage}
}

Puis renseigner la variable OBM\_INCLUDE\_VAR avec le nom du
répertoire d'include : 

\shadowbox{
  \begin{minipage}{13cm}
\begin{verbatim}
Setenv OBM_INCLUDE_VAR obminclude
\end{verbatim}
  \end{minipage}
}

Le chemin d'accès au répertoire d'include d'OBM doit être donné
(D'ailleurs, le répertoire \fichier{obminclude} peut être déplacé) :  

Renseigner la variable include\_path avec le chemin vers le répertoire
d'include : 

\shadowbox{
  \begin{minipage}{13cm}
\begin{verbatim}
php\_value include\_path ".:/var/www/obm"
\end{verbatim}
  \end{minipage}
}


\subsubsection*{Alias pour les images}

Il faut installer un alias images vers le répertoire de themes. Depuis 
la version 2.0, cet alias doit pointer vers le chemin 
\fichier{/var/www/obm/ressources} : 

\shadowbox{
  \begin{minipage}{13cm}
\begin{verbatim}
Alias /images /var/www/obm/ressources
\end{verbatim}
  \end{minipage}
}

Pour les versions précédentes d'obm, l'alias est le suivant :

\shadowbox{
  \begin{minipage}{13cm}
\begin{verbatim}
Alias /images /var/www/obm/obminclude/themes
\end{verbatim}
  \end{minipage}
}

\subsubsection*{Directory Index}

Il faut enfin spécifier le fichier par défaut : 

\shadowbox{
  \begin{minipage}{13cm}
\begin{verbatim}
DirectoryIndex obm.php
\end{verbatim}
  \end{minipage}
}

Nous recommandons \textbf{chaudement} d'interdire l'accès direct aux fichiers
\fichier{.inc} :

\shadowbox{
  \begin{minipage}{13cm}
\begin{verbatim}
<Files ~ "\.inc$">
   Order allow,deny
   Deny from all
</Files>
\end{verbatim}
  \end{minipage}
}


\subsubsection*{Section virtual host d'exemple complète}

Vérifier que votre IP soit définie comme un \emph{named virtual
  host} et insérer cette section :

\shadowbox{
  \begin{minipage}{13cm}
\begin{verbatim}
NameVirtualHost 192.168.1.5

<VirtualHost 192.168.1.5>
 ServerAdmin root@localhost
 DocumentRoot /var/www/obm/php
 ServerName obm
 ErrorLog /var/log/apache/obm-error.log
 CustomLog /var/log/apache/obm-access.log common
 SetEnv OBM_INCLUDE_VAR obminclude
 Alias /images /var/www/obm/ressources
 DirectoryIndex obm.php
</VirtualHost>
\end{verbatim}
  \end{minipage}
}


\subsection{Configuration d'\obm}

Depuis la version 2.0, la configuration d'\obm se trouve dans les 
fichiers \fichier{conf/obm\_conf.ini} et \fichier{conf/obm\_conf.inc} 
(Pour les versions précédentes, il s'agit du fichier 
\fichier{obminclude/obm\_conf.inc}).

Pour créer ces fichiers, copier \fichier{conf/obm\_conf.ini.sample} et 
\fichier{conf/obm\_conf.inc.sample} vers dans les nouveaux fichiers 
\fichier{conf/obm\_conf.ini} et \fichier{conf/obm\_conf.inc}, puis éditer 
ces derniers.


\subsubsection{Configuration initiale pour la base de données OBM}

Éditer le fichier \fichier{obm\_conf.ini} dans \fichier{conf} et

\begin{itemize}
\item choisir la base de données à utiliser ; 
\item la déclarer ;
\item déclarer le nom d'utilisateur et le mot de passe à utiliser.
\item déclarer l'url d'accès à OBM (\$cgp\_host).
\end{itemize}

\shadowbox{
  \begin{minipage}{13cm}
\begin{verbatim}
; Information sur les bases de données
host = localhost
dbtype = MYSQL
db = obm
user = obm
; Le mot de passe DOIT etre entre "
password = "aliacom"

\end{verbatim}
  \end{minipage}
}

Pour modifier des préférences globales (comme cgp\_mail\_enabled par 
exemple) et déclarer le dépôt de documents (voir \ref{install_doc}), 
il faut éditer le fichier \fichier{obm\_conf.inc}.

\shadowbox{
  \begin{minipage}{13cm}
\begin{verbatim}

$cgp_host = "http://obm/";
...
$cdocument_root = "/var/www/obmdocuments/";
...
// is Mail enabled ? (agenda)
$cgp_mail_enabled = false;
\end{verbatim}
  \end{minipage}
}


\subsubsection{Configurer le dépôt de documents}
\label{install_doc}

Éditer le fichier \fichier{obm\_conf.inc} dans \fichier{conf} (ou 
dans \fichier{obminclude} pour les versions antérieures à la 2.0).

\shadowbox{
  \begin{minipage}{13cm}
\begin{verbatim}
$cdocument_root = "/var/www/obmdocuments";
\end{verbatim}
  \end{minipage}
}


Créer le répertoire et le faire appartenir à l'utilisateur exécutant le serveur apache (www-data sous Debian).

\shadowbox{
  \begin{minipage}{13cm}
\begin{verbatim}
mkdir /var/www/obmdocuments
chown www-data:www-data /var/www/obmdocuments
\end{verbatim}
  \end{minipage}
}


\subsubsection{Configuration et création de la base de données}

Les scripts de création de la base de données se trouvent dans le
répertoire \fichier{scripts/2.0}.

\subsubsection*{Langue par défaut des données}

Certaines valeurs de références sont insérées par défaut. Ces valeurs
peuvent être en français (par défaut) ou en anglais. Pour passer de
l'un à l'autre, il faut éditer le fichier
\fichier{scripts/2.0/install\_obmdb\_2.0.mysql.sh} (ou
\fichier{scripts/2.0/install\_obmdb\_2.0.pgsql.sh}) pour changer la
valeur de DATA\_LANG à \texttt{en} : 

\shadowbox{
  \begin{minipage}{13cm}
\begin{verbatim}
# Data lang var definition 
DATA_LANG="en"
\end{verbatim}
  \end{minipage}
}

\subsubsection{MySQL}

Pré-requis : disposer d'un login/mot de passe valide pour MySQL.

Exemple de création d'un utilisateur MySQL : 

\shadowbox{
  \begin{minipage}{13cm}
\begin{verbatim}
GRANT CREATE, DROP, SELECT, UPDATE, INSERT, DELETE, LOCK TABLES, INDEX
ON obm.* TO obm@localhost IDENTIFIED BY 'obm'; 
\end{verbatim}
  \end{minipage}
}
ou plus simplement :

\shadowbox{
  \begin{minipage}{13cm}
\begin{verbatim}
GRANT ALL ON obm.* TO obm@localhost IDENTIFIED BY 'obm';
\end{verbatim}
  \end{minipage}
}

Le script \fichier{install\_obmdb\_2.0.mysql.sh} gère les différentes
étapes de création de la base de données. Positionner les bonnes
valeurs de login/mot de passe dans le fichier \fichier{conf/obm\_conf.ini} 
avant de lancer le script :

\shadowbox{
  \begin{minipage}{13cm}
\begin{verbatim}
cd obm
gedit conf/obm\_conf.ini
cd scripts/2.0
./install_obmdb_2.0.mysql.sh
\end{verbatim}
  \end{minipage}
}

\subsubsection{PostgreSQL}

OBM supporte PostgreSQL depuis la version 0.7.5.

Création de l'utilisateur obm et de la base de données éponyme :

\shadowbox{
  \begin{minipage}{13cm}
\begin{verbatim}
create user obm password 'obm';
create database obm with owner = obm;
\end{verbatim}
  \end{minipage}
}

\subsubsection*{Jeux de caractères}

Si votre server PostgreSQL n'est pas configuré pour le jeu de
caractères latin1, éditez le fichier \fichier{scripts/2.0/postgres-pre.sql}
et décommentez la ligne :

\shadowbox{
  \begin{minipage}{13cm}
\begin{verbatim}
\encoding latin1
\end{verbatim}
  \end{minipage}
}

\subsubsection*{Création de la base Postgres}

Le script \fichier{install\_obmdb\_2.0.pgsql.sh} gère toutes les 
étapes de création de la base de données. Positionner les bonnes
valeurs de login/mot de passe dans le fichier \fichier{conf/obm\_conf.ini} 
avant de lancer le script :

\shadowbox{
  \begin{minipage}{13cm}
\begin{verbatim}
cd obm
gedit conf/obm\_conf.ini
cd scripts/2.0
./install_obmdb_2.0.pgsql.sh
\end{verbatim}
  \end{minipage}
}

\subsection{Support du système d'authentification CAS (Central Authentication Service)}

\subsubsection{Pré-requis}

Le support CAS s'appuie sur la librairie cliente PHP, phpCAS (http://esup-phpcas.sourceforge.net/index.html).

Les librairies Curl doivent être installées sur votre système et compilées avec le support de SSL.
Sous Debian, il faut installer les packages : curl, libcurl3, libcurl3-gnutls.  

La version de PHP doit être supérieure à la 4.2.2.
PHP doit être installé avec le support curl, openssl, domxml et zlib.
Sous Debian, il faut installer les packages php4-curl et php4-domxml.

Il est nécessaire d'installer PEAR. Les libariries PEAR::DB et PEAR:Log sont utilisées.
Sous Debian, il faut installer le packages php4-pear.

\subsubsection{Configuration d'OBM}

Éditer le fichier \fichier{obm\_conf.inc} dans \fichier{obminclude} et

\begin{itemize}
\item spécifier le mode d'authentification "CAS"; 
\item déclarer les paramètres de connection au serveur CAS.
\end{itemize}

Les variables suivantes du fichier de configuration doivent être rennseignées:

\shadowbox{
  \begin{minipage}{13cm}
\begin{verbatim}

...
// authentification : 'CAS' (SSO) or 'standalone' (default)
$auth_kind="CAS";
$cas_server = "sso.aliacom.local";
$cas_server_port = "8443";
$cas_server_uri = "/cas";
...

\end{verbatim}
  \end{minipage}
}

\subsection{Lancer OBM (accéder à obm.php depuis un navigateur)}

D'abord, redémarrer le serveur web apache : 

\shadowbox{
  \begin{minipage}{13cm}
\begin{verbatim}
/etc/init.d/apache restart
\end{verbatim}
  \end{minipage}
}

Si tout marche bien, lancer un navigateur (Firefox par exemple) et
aller à l'URL : \url{http://yourvirtualhost/}, puis se connecter en
utilisant le compte \texttt{uadmin/padmin}.

