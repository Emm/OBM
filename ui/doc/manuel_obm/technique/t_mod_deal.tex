% Documentation technique d'OBM : module Deal
% ALIACOM Pierre Baudracco
% $Id$


\clearpage
\section{Affaire (module \deal)}

Le module \deal d'\obm.

\subsection{Organisation de la base de données}

Le module \deal utilise 6 tables :
\begin{itemize}
 \item Deal
 \item DealStatus
 \item DealType
 \item DealCategory
 \item DealCategoryLink
 \item taskType
\end{itemize}

\subsection{Deal}
Table principale des informations d'une affaire.\\

\begin{tabular}{|p{3cm}|c|p{5.4cm}|p{2.6cm}|}
\hline
\textbf{Champs} & \textbf{Type} & \textbf{Description} & \textbf{Commentaire} \\
\hline
\_id & int 8 & Identifiant & Clé primaire \\
\hline
\_timeupdate & timestamp 14 & Date de mise à jour & \\
\hline
\_timecreate & timestamp 14 & Date de création & \\
\hline
\_userupdate & int 8 & Id du modificateur & \\
\hline
\_usercreate & int 8 & Id du créateur & \\
\hline
\_number & varchar 32 & Numéro ou identifiant de l'affaire & \\
\hline
\_label & varchar 128 & Label de l'affaire & \\
\hline
\_datebegin & date & Date de début de l'affaire & \\
\hline
\_parentdeal\_id & int 8 & Suraffaire d'appartenance de l'affaire & \\
\hline
\_type\_id & int 8 & Type de l'affaire & (Achat, Vente,...) \\
\hline
\_tasktype\_id & int 8 & Domaine d'activité de l'affaire & (Intégration,...) \\
\hline
\_company\_id & int 8 & Société en relation & \\
\hline
\_contact1\_id & int 8 & Contact 1 de l'affaire & \\
\hline
\_contact2\_id & int 8 & Contact 2 de l'affaire & \\
\hline
\_marketingmanager\_id & int 8 & Responsable commercial de l'affaire & \\
\hline
\_technicalmanager\_id & int 8 & Responsable technique de l'affaire & \\
\hline
\_dateproposal & date & Date de la dernière proposition commerciale & \\
\hline
\_amount & Decimal(12,2) & Montant estimé ou proposé de l'affaire & \\
\hline
\_hitrate & int 3 & Pourcentage de réussite estimé & \\
\hline
\_status\_id & int 2 & Etat de l'affaire & \\
\hline
\_datealarm & date & Date d'alarme pour la prochaine action & \\
\hline
\_archive & char 1 & Indicateur d'archivage & (1 = 0ui)\\
\hline
\_todo & varchar 128 & Action à faire & \\
\hline
\_privacy & int 2 & Indicateur de visibilité de l'affaire & \\
\hline
\_comment & text (64k) & Commentaire &\\
\hline
\end{tabular}


\subsection{DealStatus}
Table de référence des états d'une affaire.\\

\begin{tabular}{|p{3cm}|c|p{5.4cm}|p{2.6cm}|}
\hline
\textbf{Champs} & \textbf{Type} & \textbf{Description} & \textbf{Commentaire} \\
\hline
\_id & int 8 & Identifiant & Clé primaire \\
\hline
\_project\_id & int 8 & Projet de la tâche & \\
\hline
\_timeupdate & timestamp 14 & Date de mise à jour & \\
\hline
\_timecreate & timestamp 14 & Date de création & \\
\hline
\_userupdate & int 8 & Id du modificateur & \\
\hline
\_usercreate & int 8 & Id du createur & \\
\hline
\_label & varchar 24 & Label de l'état & \\
\hline
\_order & int 2 & Ordre d'affichage de l'état & \\
\hline
\_hitrate & char 3 & Pourcentage de réussite associé à l'état & \\
\hline
\end{tabular}

\subsubsection{Remarques}

\paragraph{dealstatus\_hitrate} : Pourcentage de réussite associé à l'état. La sélection de cet état positionne automatiquement par défaut le pourcentage de réussite de l'affaire à ce pourcentage.
La valeur positionnée dans l'affaire peut cependant être modifiée.


\subsection{DealType}
Table de référence des types d'affaire.

\begin{tabular}{|p{3cm}|c|p{5.4cm}|p{2.6cm}|}
\hline
\textbf{Champs} & \textbf{Type} & \textbf{Description} & \textbf{Commentaire} \\
\hline
\_id & int 8 & Identifiant & Clé primaire \\
\hline
\_timeupdate & timestamp 14 & Date de mise à jour & \\
\hline
\_timecreate & timestamp 14 & Date de création & \\
\hline
\_userupdate & int 8 & Id du modificateur & \\
\hline
\_usercreate & int 8 & Id du createur & \\
\hline
\_label & varchar 16 & Label du type & \\
\hline
\_inout & char 1 & Indicateur d'entrée ou sortie du type de l'affaire & (+ = entrée) \\
\hline
\end{tabular}

\subsubsection{Remarques}

\paragraph{dealtype\_inout} : Indicateur d'entrée ou sortie du type d'affaire. Une affaire dont le type a cet indicateur à '+' sera comptée positivement dans les calculs ou estimations de chiffre d'affaire.

Une affaire dont le type a l'indicateur à '-' aura son montant compté négativement dans les calculs.
Cependant le module \deal étant surtout destiné aux forces de vente, ceci ne devrait pas être fréquent.


\subsection{DealCategory}
Table des catégories d'affaire. Une affaire peut être reliée à plusieurs catégories.\\

\begin{tabular}{|p{3cm}|c|p{5.4cm}|p{2.6cm}|}
\hline
\textbf{Champs} & \textbf{Type} & \textbf{Description} & \textbf{Commentaire} \\
\hline
\_id & int 8 & Identifiant & Clé primaire \\
\hline
\_timeupdate & timestamp 14 & Date de mise à jour & \\
\hline
\_timecreate & timestamp 14 & Date de création & \\
\hline
\_userupdate & int 8 & Id du modificateur & \\
\hline
\_usercreate & int 8 & Id du createur & \\
\hline
\_code & int 8 & Code de la catégorie & \\
\hline
\_label & varchar 100 & Label de la catégorie & \\
\hline
\end{tabular}


\subsubsection{DealCategoryLink}
Table de liaison entre les affaires et catégories d'affaire.\\

\begin{tabular}{|p{3cm}|c|p{5.4cm}|p{2.6cm}|}
\hline
\textbf{Champs} & \textbf{Type} & \textbf{Description} & \textbf{Commentaire} \\
\hline
\_category\_id & int 8 & Référence à la catégorie & \\
\hline
\_deal\_id & int 8 & Référence à l'affaire & \\
\hline
\end{tabular}


\subsection{Actions et droits}

Voici la liste des actions du module \project, avec le droit d'accès requis ainsi qu'une description sommaire de chacune d'entre elles.\\

\begin{tabular}{|l|c|p{9.5cm}|}
 \hline
 \textbf{Intitulé} & \textbf{Droit} & \textbf{Description} \\
 \hline
 \hline
  index & read & (Défaut) formulaire de recherche d'affaires. \\ 
 \hline
  search & read & Résultat de recherche d'affaires. \\
 \hline
  new & write & Formulaire de création d'une affaire. \\
 \hline
  detailconsult & read & Fiche détail d'une affaire. \\
 \hline
  detailupdate & write & Formulaire de modification d'une affaire. \\
 \hline
  insert & write & Insertion d'une affaire. \\
 \hline
  update & write & Mise à jour d'une affaire. \\
 \hline
  quick\_update & write & Mise à jour rapide d'une affaire. \\
 \hline
  check\_delete & write & Vérification avant suppression d'une affaire. \\
 \hline
  delete & write & Suppression d'une affaire. \\
 \hline
  affect & write & Formulaire d'affectation à une suraffaire. \\
 \hline
  affect\_update & write & Affectation à une suraffaire. \\
 \hline
  admin & read\_admin & Ecran d'administration des affaires. \\
 \hline
  kind\_insert & write\_admin & Ajout d'un type d'affaire. \\
 \hline
  kind\_update & write\_admin & Modification d'un type d'affaire. \\
 \hline
  kind\_checklink & write\_admin & Vérification avant suppression d'un type d'affaire. \\
 \hline
  kind\_delete & write\_admin & Suppression d'un type d'affaire. \\
 \hline
  status\_insert & write\_admin & Ajout d'un état d'affaire. \\
 \hline
  status\_update & write\_admin & Modification d'un état d'affaire. \\
 \hline
  status\_checklink & write\_admin & Vérification avant suppression d'un état d'affaire. \\
 \hline
  status\_delete & write\_admin & Suppression d'un état d'affaire. \\
 \hline
  cat\_insert & write\_admin & Ajout d'une catégorie d'affaire. \\
 \hline
  cat\_update & write\_admin & Modification d'une catégorie d'affaire. \\
 \hline
  cat\_checklink & write\_admin & Vérification avant suppression d'une catégorie d'affaire. \\
 \hline
  cat\_delete & write\_admin & Suppression d'une catégorie d'affaire. \\
 \hline
  display & read & Ecran de modification des préférences d'affichage. \\
 \hline
  dispref\_display & read & Modifie l'affichage d'un élément. \\
 \hline
  dispref\_level & read & Modifie l'ordre d'affichage d'un élément. \\
 \hline
  document\_add & write & Ajout de liens vers des documents. \\
 \hline
  parent\_search & read & Résultat de recherche de suraffaires. \\
 \hline
  parent\_new & write & Formulaire de création d'une suraffaire. \\
 \hline
  parent\_detailconsult & read & Fiche détail d'une suraffaire. \\
 \hline
  parent\_detailupdate & write & Formulaire de modification d'une suraffaire. \\
 \hline
  parent\_insert & write & Insertion d'une suraffaire. \\
 \hline
  parent\_update & write & Mise à jour d'une suraffaire. \\
 \hline
  parent\_delete & write & Suppression d'une suraffaire. \\
 \hline
\end{tabular}
