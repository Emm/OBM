% Documentation technique d'OBM : Variables globales
% ALIACOM Pierre Baudracco
% $Id$


\subsection{Les variables globales définies par \obm}
\label{global_vars}

\obm met à la disposition du développeur certaines variables, définies de façon globale et donc utilisables partout avec un scope global.


\subsubsection{Variables globales générées pas programme}

\begin{tabular}{|p{2cm}|p{5cm}|p{6cm}|}
\hline
\textbf{Nom} & \textbf{Valeur} & \textbf{description}  \\
\hline
\$path & chemin relatif vers la racine du répertoire des modules (php/) & défini en tête des modules, utilisé pour construire des urls \\ 
\hline
\$module & module effectuant le traitement courant (company, contact,..) & défini en tête des modules \\ 
\hline
\$obminclude & nom du répertoire ``obminclude'' & défini en tête des modules \\ 
\hline
\$obm & informations globales
\begin{itemize}
\item[->[uid]] : id de l'utilisateur
\item[->[profile]] : profil de l'utilisateur
\item[->[domain\_id]] : domaine de l'utilisateur
\item[->[domain\_label]] : nom du domaine de l'utilisateur
\end{itemize}
 & remplace les variables de l'ancien \$auth\\ 
\hline
\$auth & informations d'authentification
\begin{itemize}
\item[->auth[exp]] : domaine de l'utilisateur
\item[->auth[refresh]] : nom du domaine de l'utilisateur
\end{itemize}
 & hérité de phplib. A faire évoluer \\ 
\hline
\$sess & information sur la session & hérité de phplib. A faire évoluer \\ 

\hline
\$extra\_js & code Javascript à inclure & Permet d'ajouter du code Javascript qui sera placé en en-tête de page <head> \\ 
\hline
\$extra\_css & nom de fichier d'une css à inclure (ex: calendar.css)& Permet d'ajouter une CSS spécifique à un module \\ 
\hline
\$display & zones graphiques définies & voir \ref{display} \\ 
\hline
\end{tabular}


\subsubsection{Variables globales passées par paramètre}

\begin{tabular}{|p{2cm}|p{5cm}|p{6cm}|}
\hline
\textbf{Nom} & \textbf{Valeur} & \textbf{description}  \\
\hline
\$action & traitement à effectuer (ex: new, insert,...) & défini par programme \\ 
\hline
\$view & vue spécifique à afficher (action detailconsult) & Permet d'afficher différentes sous-vues d'une entité \\ 
\hline
\$of\_display\_name & Nom du dataset of\_display concerné & Permet de gérer plusieurs dataset dans le même écran \\ 
\hline
\$popup & indicateur d'affichage en popup (pas de menu) & a vérifier si utilisé directement encore \\ 
\hline
\end{tabular}

\subsubsection{Gestion des paramètres}

Aucun paramètre ne doit être utilisé directement ! Donc traiter les paramètres du tableau précédent.

Afin de permettre la vérification de tous les paramètres (et la désactivation de la directive PHP register\_globals) tous les paramètres doivent être récupérés par la fonction \fonction{get\_params()} qui retourne le haschage \$params à utiliser.
