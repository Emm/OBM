% Documentation technique d'OBM : OF : Gestion d'un selecteur multivalue riche
% ALIACOM Pierre Baudracco
% $Id$


\subsection{Gestion d'un sélecteur multi-valué riche}
\label{of_select}

Afin d'améliorer l'ergonomie et l'utilisation, \obm propose un nouvel objet sélecteur présentant un affichage plus riche et convivial qu'un widget de type SELECT, avec les fonctionnalités :\\ 

\begin{itemize}
\item Sélection d'entrée dans le sélecteur par recherches
\item Seules sont affichées les entités sélectionnées
\item Possibilité de supprimer une entité de l'ensemble
\item Possibilité de supprimer l'ensemble des entités
\item Possibilité d'insérer dans l'ensemble des entités de types différents (des utilisateurs + des ressources)
\end{itemize}
\vspace{0.3cm}

Cet objet est par exemple utilisé par la gestion des droits acl, par la sélection d'utilisateurs et de ressources dans les vues de l'agenda, par la sélection d'utilisateurs et groupes pour un nouvel événement, dans les recherches par type de tâche dans les affaires...\\

Ces fonctionalités sont fournies par le fichier \fichier{of/of\_select.inc}


\subsubsection{Spécifications techniques du sélecteur of\_select}

\paragraph{Le principe} consiste à gérer un widget multivalué (sel[]) caché, et de gérer simultanément un élément <div> (pour l'affichage) pour chaque entité sélectionnée.

L'élément désigne le widget global et doit être le nom donné au container et au widget multivalué (chaque champ input).

Pour chaque entité on trouve :\\
\begin{itemize}
\item une ligne de l'élément : <div>
\item un champ input du widget
\end{itemize}
\vspace{0.3cm}

La section <div> contient un texte et une image et le clic sur l'image efface automatiquement cette ligne.
Le champ input est inclus dans le div. La suppression de la ligne div supprime par ce fait le champ input et donc l'entité ou ligne de l'élément.

\paragraph{Le nommage de l'élément} : Une ligne d'élément div doit être unique pour la page.
Comme il peut y avoir plusieurs éléments dans une même page (exemple: liste d'utilisateurs avec droits de lecture et liste d'utilisateurs avec droits d'écriture dans une page de gestion des droits acl), le nom (ou id) du div doit contenir le nom de l'élément.
Comme un élément peut contenir plusieurs types d'entités (exemple: utilisateurs et ressources), le nom du div doit contenir le type de l'entité en plus de l'id de l'entité.

Principe de nommage.\\

\begin{tabular}{|p{3cm}|p{4cm}|p{4cm}|}
\hline
\textbf{Elément} & \textbf{Input (valeur)} & \textbf{Id du div} \\
\hline
sel\_user\_id & data-user-\$id & sel\_user\_id-data-user-\$id \\
\hline
\end{tabular}


\paragraph{Code XHTML d'un Elément avec 1 ligne} :\\

\shadowbox{
\begin{minipage}{13cm}
\begin{verbatim}
<td class="detail" id="sel_ent">
  <div class="elementRow" id="sel_ent-data-user-6">
    <a href="javascript: remove_element('sel_ent-data-user-6','sel_ent');">
    <img src="http://obm/images/standard/ico_calendar0.gif">
    </a>
    Pierre BAUDRACCO
    <input value="data-user-6" name="sel_ent[]" type="hidden" />
  </div>
\end{verbatim}
\end{minipage}
}


\subsubsection{Liste de résultat vers ensemble d'éléments}

\paragraph{Implémentation : module popup}

Le module appelé va utiliser les fonctions définies par cette bibliothèque pour renseigner l'élément du module appelant.

C'est le module appelé (popup) qui doit donc inclure la librairie (\fichier{of\_select.inc}).
Ce module doit aussi ajouter un cas à la fonction \fonction{dis\_data\_module()} (qui gère l'affichage des résultats), pour retourner la valeur à insérer (data-module-\$id) dans l'élément pour une ligne de résultat cochée.


\subsubsection{Implémentation : API fonctions publiques}

\shadowbox{
\begin{minipage}{13cm}
\begin{verbatim}
function of_select_fill_from_checkbox(int_form) {
\end{verbatim}
\end{minipage}
}

anciennement of\_extmod\_fill\_ext\_element(int\_form)
Fonction Javascript qui renseigne l'élément de la fenêtre appelante en fonction des cases cochées dans la liste de résultat.
Exemple: Liste des utilisateurs qui ont le droit de voir l'agenda d'un utilisateur.

Cette fonction est appelée à la validation (soumission du formulaire) des résultats sélectionnés dans la fenêtre popup.
