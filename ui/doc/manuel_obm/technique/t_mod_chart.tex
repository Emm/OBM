% Documentation technique d'OBM : Module Chart
% ALIACOM Pierre Baudracco
% $Id$

\clearpage
\section{Diagrammes}

Depuis la version 1.1, \obm int�gre un module de g�n�ration de diagrammes.

\obm utilise actuellement la librairie ou le projet Artichow pour la cr�ation des diagrammes.
Le module \chart propose une couche d'abstraction d'utilisation des diagrammes permettant �ventuellement de changer de librairie sous-jacente.\\

Le module \chart est particulier car il n'apparait pas dans les menus de l'application mais est un service pour les modules devant g�n�rer des diagrammes.
Le module \chart est appel� via une URL, et retourne, sous forme d'image, le diagramme correspondant aux param�tres pass�s.

Il n'effectue pas de gestion des droits ou de session.


\subsection{Diagrammes propos�s}

Artichow �tant assez riche, \obm s'enrichi de nouveaux diagrammes au fur et � mesure des besoins.\\

\begin{tabular}{|p{4cm}|p{4cm}|}
\hline
\textbf{Diagramme} & \textbf{Version d'\obm} \\
\hline
Barre simple & 1.1.1 \\
\hline
Barres multiples & 1.2.0 \\
\hline
\end{tabular}


\subsection{Param�tres et utilisation}

\subsubsection{Diagramme � barre simple}

\begin{tabular}{|p{2cm}|p{3cm}|p{8cm}|}
\hline
\textbf{Param�tre} & \textbf{Valeur} & \textbf{Description}\\
\hline
action & bar & \\ 
\hline
title & texte & Titre du diagramme \\
\hline
values & tableau & Valeur des points (barres) \\
\hline
labels & tableau & Labels affich�s des points (barres) \\
\hline
xlabels & tableau & Labels de l'axe X \\
\hline
\end{tabular}


\subsubsection{Diagramme � barres multiples}

\paragraph{Note :} 1 plot correspond � un ensemble de donn�es (1 jeu de barres), 1 bar correspond � une barre. Une barre peut pr�senter des informations de plusieurs plots (barre d�coup�e en parties ou couleurs, comme le potentiel des factures).\\

\begin{tabular}{|p{2cm}|p{3cm}|p{8cm}|}
\hline
\textbf{Param�tre} & \textbf{Valeur} & \textbf{Description}\\
\hline
action & bar\_multiple & \\ 
\hline
title & texte & Titre du diagramme \\
\hline
xlabels & tableau & Labels de l'axe X \\
\hline
plots & \multicolumn{2}{|c|}{tableau avec entr�es du tableau suivant} \\
\hline
\end{tabular}
\vspace{0.3cm}

Entr�es du hachage plots :\\

\begin{tabular}{|p{2cm}|p{3cm}|p{8cm}|}
\hline
\textbf{Param�tre} & \textbf{Valeur} & \textbf{Description}\\
\hline
values & tableau & values[0] : Valeur des points du 1er plot \\
\hline
labels & tableau & labels[0] : Labels affich�s des points du 1er plot \\
\hline
legends & tableau & legend[0] : L�gende associ�e au 1er plot \\
\hline
new\_bar & tableau (1|0) & new\_bar[n]=1 indique si plot n affich� dans une nouvelle barre (sinon affich� sur la barre actuelle) \\
\hline
\end{tabular}


\paragraph{Exemple :}

\begin{verbatim}
  $title = "$l_billed $year_prev / $year";
  $plots["new_bar"] = array(1, 1);
  $plots["legends"] = array("$year_prev", "$year");
  $plots["values"] = array($values_prev, $values);
  $plots["labels"] = array($labels_prev, $labels);

  $chart_ytoy_params .= "&amp;title=".urlencode($title).
  "&amp;plots=".urlencode(serialize($plots)) .
  "&amp;xlabels=".urlencode(serialize($xlabels));

  ...

  $block = "
    <img border=\"0\"
    src=\"$path/chart/chart_index.php?action=bar_multiple$chart_ytoy_params\"
     alt=\"\" />
  ...
\end{verbatim}
