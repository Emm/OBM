% Documentation technique d'OBM : Gestion des données utilisateur
% AliaSource Pierre Baudracco
% $Id$


\subsection{Gestion des données spécifiques utilisateurs}
\label{of_userdata}
révision : \obm 2.0.0\\

\obm permet la gestion de données spécifiques à un site, appelées données utilisateurs, sans modification du modèle de base de données.

Les types de données pouvant être ajoutées sont :\\

\begin{tabular}{|p{8cm}|c|}
\hline
\textbf{Type de données} & \textbf{Depuis version OBM} \\
\hline
Catégories multi-valuées & 2.0.0 \\
\hline
Catégories mono-valuées & 2.0.0 \\
\hline
Champs texte simple & futur \\
\hline
\end{tabular}
\vspace{0.3cm}

Les fonctionalités de données utilisateurs sont fournies par le fichier \fichier{of/of\_category.inc} et les fonctions sont préfixées par \variable{of\_userdata\_}.


\subsubsection{Définition de données utilisateurs}

Les données utilisateurs sont déclarées et définies dans la tableau global \variable{\$cgp\_user}

\paragraph{Ajout de catégorie multi-valuée} : ex : ajout de la catégorie ``cat1'' liée aux sociétés. Une même catégorie peut être liée à plusieurs modules.\\

\variable{\$cgp\_user["company"]["category"]["cat1"] = array("mode"=>"multi");}

\paragraph{Ajout de catégorie mono-valuée} : \\

\variable{\$cgp\_user["company"]["category"]["cat2"] = array("mode"=>"mono");}


\subsubsection{Implémentation : API fonctions publiques}

Les fonctions de gestion des données utilisateurs sont celles à intégrer dans les modules d'\obm et proposent un appel unique (wrapper) pour gérer l'ensemble des données utilisateurs (catégories, champs,...).
Les fonctions de gestion de catégories ou de champs utilisateurs ne doivent donc pas être utilisées directement dans les modules.\\

L'implémentation dans un module nécessite en général :
\begin{itemize}
\item La prise en compte dans le fichier \fichier{module\_index.php}
  \begin{itemize}
  \item[*] Inclusion du fichier source\\
\shadowbox{
\begin{minipage}{14cm}
\begin{verbatim}
require_once("$obminclude/of/of_category.inc");)
\end{verbatim}
\end{minipage}
}

  \item[*] Définition des actions fournies\\
\shadowbox{
\begin{minipage}{14cm}
\begin{verbatim}
function get_module_actions() {
  ...
  of_category_user_module_action("module");
\end{verbatim}
\end{minipage}
}

  \item[*] Utilisation des actions fournies\\
\shadowbox{
\begin{minipage}{14cm}
\begin{verbatim}
} else if ($action == "dispref_level") {
///////////////////////////////////////////////////////////////////////////////
  update_display_pref($params);
  $prefs = get_display_pref($obm["uid"], "document", 1);
  $display["detail"] = dis_document_display_pref($prefs);
}

of_category_user_action_switch($module, $action, $params);
\end{verbatim}
\end{minipage}
}

  \end{itemize}
\item l'appel des fonctions correspondantes dans les fonctions suivantes du module :
\end{itemize}

\begin{tabular}{|p{6cm}|c|}
\hline
\textbf{Fonction du module} & \textbf{Fonction de données utilisateurs appelée} \\
\hline
\fonction{html\_module\_search\_form()} & \fonction{of\_userdata\_dis\_search()} \\
\hline
\fonction{html\_module\_search\_list()} & \fonction{of\_userdata\_get\_url\_search\_params()} \\
\hline
\fonction{html\_module\_consult()} & \fonction{of\_userdata\_dis\_entity\_consult()} \\
\hline
\fonction{html\_module\_form()} & \fonction{of\_userdata\_dis\_entity\_form()} \\
\hline
\fonction{dis\_module\_warn\_insert()} & \fonction{of\_userdata\_dis\_hidden\_fields()} \\
\hline
\fonction{dis\_module\_admin\_form()} & \fonction{of\_userdata\_dis\_admin\_form()} \\
\hline
\fonction{run\_query\_module\_search()} & \fonction{of\_userdata\_query\_search()} \\
\hline
\fonction{run\_query\_module\_insert()} & \fonction{of\_userdata\_query\_update()} \\
\hline
\fonction{run\_query\_module\_update()} & \fonction{of\_userdata\_query\_update()} \\
\hline
\fonction{run\_query\_module\_delete()} & \fonction{of\_userdata\_query\_delete()} \\
\hline
\end{tabular}


\paragraph{Fonctions d'affichage} : \\

\shadowbox{
\begin{minipage}{14cm}
\begin{verbatim}
of_userdata_dis_search($entity, $params) {
\end{verbatim}
\end{minipage}
}

\begin{tabular}{|p{3cm}|p{10cm}|}
\hline
\textbf{Paramètres} & \textbf{Description}\\
\hline
\multicolumn{2}{|c|}{Fonction d'affichage du champ de recherche de la catégorie pour le formulaire de recherche}\\
\hline
\$entity & entité (ex: ``company'', ``contact'',..)\\
\hline
\hline
\$params & tableau des paramètres contenant les valeurs des données utilisateurs\\
\hline
\hline
\textbf{Retour} & \textbf{Description}\\
\hline
String XHTML & Block de recherche des données utilisateurs\\
\hline
\end{tabular}
\vspace{0.4cm}


\shadowbox{
\begin{minipage}{14cm}
\begin{verbatim}
of_userdata_get_url_search_params($entity, $params) {
\end{verbatim}
\end{minipage}
}

\begin{tabular}{|p{3cm}|p{10cm}|}
\hline
\textbf{Paramètres} & \textbf{Description}\\
\hline
\multicolumn{2}{|c|}{Fonction qui retourne la liste des paramètres de recherche des données utilisateurs sous forme d'url}\\
\hline
\$entity & entité (ex: ``company'', ``contact'',..)\\
\hline
\hline
\$params & tableau des paramètres contenant les valeurs des données utilisateurs\\
\hline
\hline
\textbf{Retour} & \textbf{Description}\\
\hline
String url & Block des paramètres utilisateurs : \&amp;sel\_companycategory1=2\&amp;...\\
\hline
\end{tabular}
\vspace{0.4cm}


\shadowbox{
\begin{minipage}{14cm}
\begin{verbatim}
of_userdata_dis_entity_consult($entity, $ent_id) {
\end{verbatim}
\end{minipage}
}

\begin{tabular}{|p{3cm}|p{10cm}|}
\hline
\textbf{Paramètres} & \textbf{Description}\\
\hline
\multicolumn{2}{|p{13cm}|}{Fonction d'affichage de la zone données utilisateurs (ensemble des champs utilisateurs) en mode consultation pour une entité}\\
\hline
\$entity & entité (ex: ``company'', ``contact'',..)\\
\hline
\$ent\_id & Id de l'entité affichée\\
\hline
\hline
\textbf{Retour} & \textbf{Description}\\
\hline
String XHTML & Block de données utilisateurs en affichage consultation\\
\hline
\end{tabular}
\vspace{0.4cm}


\shadowbox{
\begin{minipage}{14cm}
\begin{verbatim}
of_userdata_dis_entity_form($entity, $ent_id, $params) {
\end{verbatim}
\end{minipage}
}

\begin{tabular}{|p{3cm}|p{10cm}|}
\hline
\textbf{Paramètres} & \textbf{Description}\\
\hline
\multicolumn{2}{|p{13cm}|}{Fonction d'affichage de la zone données utilisateurs (ensemble des champs utilisateurs) en mode modification (formulaire) pour une entité}\\
\hline
\$entity & entité (ex: ``company'', ``contact'',..)\\
\hline
\$ent\_id & Id de l'entité affichée\\
\hline
\$params & Paramètres (permet de préciser des sélections)\\
\hline
\hline
\textbf{Retour} & \textbf{Description}\\
\hline
String XHTML & Block de données formulaire\\
\hline
\end{tabular}
\vspace{0.4cm}


\shadowbox{
\begin{minipage}{14cm}
\begin{verbatim}
of_userdata_dis_hidden_fields($entity, $params) {
\end{verbatim}
\end{minipage}
}

\begin{tabular}{|p{3cm}|p{10cm}|}
\hline
\textbf{Paramètres} & \textbf{Description}\\
\hline
\multicolumn{2}{|p{13cm}|}{Fonction qui retourne les champs utilisateurs sous forme de champs caches pour un formulaire. Utilisé pour faire passer les information utilisateurs sur des écrans intermédiaires, ex : warning avant insertion}\\
\hline
\$entity & entité (ex: ``company'', ``contact'',..)\\
\hline
\$params & tableau des paramètres contenant les valeurs des données utilisateurs\\
\hline
\hline
\textbf{Retour} & \textbf{Description}\\
\hline
String XHTML & Block de données formulaire\\
\hline
\end{tabular}
\vspace{0.4cm}


\shadowbox{
\begin{minipage}{14cm}
\begin{verbatim}
of_userdata_dis_admin_form($entity) {
\end{verbatim}
\end{minipage}
}

\begin{tabular}{|p{3cm}|p{10cm}|}
\hline
\textbf{Paramètres} & \textbf{Description}\\
\hline
\multicolumn{2}{|p{13cm}|}{Fonction d'affichage des formulaires d'administrations des données utilisateurs liées à une entité (catégories)}\\
\hline
\$entity & entité (ex: ``company'', ``contact'',..)\\
\hline
\hline
\textbf{Retour} & \textbf{Description}\\
\hline
String XHTML & Block de données formulaire\\
\hline
\end{tabular}
\vspace{0.4cm}


\paragraph{Fonctions bases de données} : \\

\shadowbox{
\begin{minipage}{14cm}
\begin{verbatim}
of_userdata_query_search($entity, $params) {
\end{verbatim}
\end{minipage}
}

\begin{tabular}{|p{3cm}|p{10cm}|}
\hline
\textbf{Paramètres} & \textbf{Description}\\
\hline
\multicolumn{2}{|p{13cm}|}{Construction des clauses de recherche pour les données utilisateurs (appelée dans la fonction recherche de l'entité : run\_query\_entity\_search())}\\
\hline
\$entity & entité (ex: ``company'', ``contact'',..)\\
\hline
\$params & tableau des paramètres contenant les valeurs des données utilisateurs\\
\hline
\hline
\textbf{Retour} & \textbf{Description}\\
\hline
array & (``where'', ``join'') : clauses WHERE et JOIN à ajouter \\
\hline
\end{tabular}
\vspace{0.4cm}


\shadowbox{
\begin{minipage}{14cm}
\begin{verbatim}
of_userdata_query_update($entity, $ent_id, $params) {
\end{verbatim}
\end{minipage}
}

\begin{tabular}{|p{3cm}|p{10cm}|}
\hline
\textbf{Paramètres} & \textbf{Description}\\
\hline
\multicolumn{2}{|p{13cm}|}{Mise à jour des données utilisateurs pour une entité donnée selon les paramètres passés (appelée dans la fonction de mise à jour de l'entité : run\_query\_entity\_update())}\\
\hline
\$entity & entité (ex: ``company'', ``contact'',..)\\
\hline
\$ent\_id & Id de l'entité\\
\hline
\$params & tableau des paramètres contenant les valeurs des données utilisateurs\\
\hline
\hline
\textbf{Retour} & \textbf{Description}\\
\hline
True | false & \\
\hline
\end{tabular}
\vspace{0.4cm}


\shadowbox{
\begin{minipage}{14cm}
\begin{verbatim}
of_userdata_query_delete($entity, $ent_id) {
\end{verbatim}
\end{minipage}
}

\begin{tabular}{|p{3cm}|p{10cm}|}
\hline
\textbf{Paramètres} & \textbf{Description}\\
\hline
\multicolumn{2}{|p{13cm}|}{Suppression des données utilisateurs pour une entité donnée (appelée suite à la suppression de l'entité)}\\
\hline
\$entity & entité (ex: ``company'', ``contact'',..)\\
\hline
\$ent\_id & Id de l'entité\\
\hline
\hline
\textbf{Retour} & \textbf{Description}\\
\hline
True | false & \\
\hline
\end{tabular}
\vspace{0.4cm}


\subsubsection{Gestion des catégories utilisateurs mono ou multi-valuées}

La gestion des catégories utilisateurs est facilitée par un ensemble de fonctions, similaires aux fonctions de gestion des catégories, permettant la gestion automatique de ces catégories utiisateurs.\\

Les fonctionalités de catégories utilisateurs sont fournies par le fichier \fichier{of/of\_category.inc} et les fonctions sont préfixées par \variable{of\_category\_user\_}.

\paragraph{Types de catégories}

\obm distingue 2 types de catégories utilisateurs :\\
\begin{itemize}
\item \textbf{mono} : catégories avec liaisons 1-n. L'entité n'est liée qu'à 0 ou 1 des catégories
\item \textbf{multi} : catégories avec liaisons n-n. L'entité peut être liée à 0, 1  ou n catégories.
\end{itemize}

\subsubsection{Modèle de stockage générique}

La principale différence par rapport aux catégories standards d'\obm vient du modèle de stockage.
Pour l'ensemble des catégories utilisateurs, uniquement 2 tables sont nécesaires et intégrées en standard à \obm, ce qui permet d'ajouter des catégories utilisateurs par simple configuration sans altérer le modèle de données.

Afin de permettre l'ajout de catégories mono-valuées sans modification du modèle de données, les 2 types de catégories, mono ou multi, sont gérées par une liaison n-n (avec table de liaison, même pour les catégories mono-valuées).\\

\textbf{Category} : Table des informations de catégories : \\

\begin{tabular}{|p{3cm}|c|p{4cm}|p{4cm}|}
\hline
\textbf{Champs} & \textbf{Type} & \textbf{Description} & \textbf{Commentaire} \\
\hline
\_id & int 8 & Id de la catégorie & \\
\hline
\_domain\_id & int 8 & Id du domaine d'appartenance & \\
\hline
\_timeupdate & timestamp & date de mise à jour & \\
\hline
\_timecreate & timestamp & date de création & \\
\hline
\_userupdate & int 8 & utilisateur ayant effectué la mise à jour & \\
\hline
\_usercreate & int 8 & créateur & \\
\hline
\_category & varchar 24 & Nom de la catégorie & Tous les enregistrements de toutes les catégories sont stockées dans cette table\\
\hline
\_code & varchar 10 & Code de la catégorie & ex: 1.0, 1.1, 1.1.4 \\
\hline
\_label & varchar 100 & Label de la catégorie & \\
\hline
\end{tabular}
\vspace{0.3cm}

\textbf{CategoryLink} : Table des informations de liaison d'une catégorie avec une entité :\\

\begin{tabular}{|p{3cm}|c|p{4cm}|}
\hline
\textbf{Champs} & \textbf{Type} & \textbf{Description}\\
\hline
\_category\_id & int 8 & Id de la catégorie\\
\hline
\_entity\_id & int 8 & Id de l'entité\\
\hline
\_category & varchar 24 & Nom de ma catégorie\\
\hline
\_entity & varchar 32 & Nom de l'entité\\
\hline
\end{tabular}
