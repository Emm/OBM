% Documentation technique d'OBM : Gestion des entités privées
% ALIACOM Pierre Baudracco
% $Id$


%%\clearpage
\subsection{Gestion des entités privées, visibilité et droits sur entité}
révision : \obm 2.1.0\\

OBM permet de créer des entités privées, soit non visibles par tout le monde.\\

La gestion des entités privées est standardisée afin de permettre une évolution possible des règles de visibilité des entités.
Elle a été repensée et rendue non intrusive depuis \obm 0.8.5.\\

Depuis \obm 2.0.2 certains modules implémentent une séparation des droits entre les entités publiques et privées (gérer des données publiques nécessite des droits plus importants).


\subsubsection{Spécifications de la notion de visibilité}

Actuellement une entité est visible par un utilisateur si l'entité est publique (privacy == 0) ou que l'utilisateur en est le créateur.

\paragraph{Base de données :} La table de l'entité doit contenir un champ : \textbf{entite\_privacy} défini comme un entier.
Les valeurs définies sont :\\

\begin{tabular}{|c|c|}
\hline
\textbf{Valeur} & \textbf{Description} \\
\hline
0 & entité publique \\
\hline
1 & entité privée \\
\hline
\end{tabular}
%%\vspace{0.3cm}

\paragraph{Implémentation du test de visibilité}
\begin{verbatim}
  $field_pri = "${entity}_privacy";
  $field_uc = "${entity}_usercreate";
  ...

  if ( ($q->f("$field_pri") == 0)
    || ($q->f("$field_uc") == $uid) ) {
    return true;
  } else {
    return false;
  }
\end{verbatim}


\subsubsection{API définie}

Une api très simple (1 fonction définie dans global.inc) est disponible :\\

\shadowbox{
\begin{minipage}{13cm}
\begin{verbatim}
function check_privacy($module, $table, $action, $id, $p_uid="") {
\end{verbatim}
\end{minipage}
}
\begin{itemize}
 \item \textbf{\$module} : menu (module) à vérifier.\\
Exemple : ``CONTACT'', mais \$module peut être utilisé. Indispensable pour récupérer l'information dans le tableau global \$actions
 \item \textbf{\$table} : table ou entité à vérifier.\\
Exemple : ``Contact''... Attention ce n'est pas toujours le module (ex: ParenDeal dans module deal). L'entité (pour nom des colonnes) est tiré de cette chaîne (lowercase)
 \item \textbf{\$action} : Action à vérifier
 \item \textbf{\$id} : Id de l'entité à vérifier
 \item \textbf{\$uid} (optionnel) : uid de l'utilisateur pour lequel le droit de visibilité doit être vérifié. Si non donné, l'utilisateur courant est utilisé.
\end{itemize}


\subsubsection{Implémentation dans un module}

\paragraph{Actions soumises au test d'accès :}
Tout accès à une entité doit être validé (passer le test de visibilité).
Ceci doit donc être vrai pour les actions de consultations (detailconsult, detailupdate) mais aussi pour toutes les actions sur une entité (update, delete, document\_add...).\\

Pour cette raison et pour éviter la répétition de ce test dans un module, le test est effectué une fois en début de module.\\

Test pour le module \contact :
\begin{verbatim}
if (! check_privacy($module, "Contact", $action, $contact["id"], $uid)) {
  $display["msg"] = display_err_msg($l_error_visibility);
  $action = "index";
} else {
  update_last_visit("contact", $contact["id"], $action);
}
\end{verbatim}

Afin d'être optimal, le test ne requiert un accès à la base de données que lorsque nécessaire.
Les actions définies du module doivent indiquer si elles sont sujettes à la protection privée (attribut \textbf{Privacy}).\\

Définition d'une action sujette à la protection donnée privée.
\begin{verbatim}
// Update
  $actions["CONTACT"]["update"] = array (
    'Url'      => "$path/contact/contact_index.php?action=update",
    'Right'    => $cright_write,
    'Privacy'  => true,
    'Condition'=> array ('None') 
                                     	);
\end{verbatim}

\paragraph{La fonction de recherche} de l'entité doit aussi tenir compte et intégrer le test de visibilité afin de ne pas afficher les entités non visibles dans les résultats de recherche.
Une fonction est définie retournant la clause de visibilité pour l'entité demandée.\\

\shadowbox{
\begin{minipage}{13cm}
\begin{verbatim}
function sql_obm_entity_privacy($entity, $p_uid="") {
\end{verbatim}
\end{minipage}
}
\begin{itemize}
 \item \textbf{\$entity} : entité à vérifier.\\
Exemple : ``contact'', ``deal''
 \item \textbf{\$uid} (optionnel) : uid de l'utilisateur pour lequel le droit de visibilité doit être vérifié. Si non donné, l'utilisateur courant est utilisé.
\end{itemize}
\vspace{0.3cm}

Exemple d'utilisation :
\begin{verbatim}
function run_query_contact_search($contact) {
  global $cdg_sql, $c_all, $ctu_sql_limit;

  $company_id = $contact["company_id"];
  ...

  // only the one which are allowed (ie. publics )
  $where .= sql_obm_entity_privacy("contact");
  ...
\end{verbatim}


\paragraph{La fonction de récupération} de l'entité (\fonction{run\_query\_entity\_detail()}) doit récuperer le champ en base de données et le présenter sous le nom \textbf{privacy} afin d'être traité de façon générique par l'affichage des informations de l'entité.\\

Exemple d'utilisation :
\begin{verbatim}
  $query = "SELECT Deal.*,
      deal_privacy as privacy,
  ...
\end{verbatim}


\subsubsection{Séparation de droits de gestion public / privé}

Pour certains modules, il est important de séparer les droits de gestion des données publiques des droits de gestion des données privées (exemple : gérer les groupes publics demande des privilèges supérieurs à la gestion de ses groupes privés).\\

Les modules implémentant cette séparation sont listés plus bas.
Cette séparation implique l'ajout de tests d'autorisation sur les actions detailupdate, update, check\_delete, delete, (user\_add,...).

Ces tests d'autorisation sont effectués :
\begin{itemize}
\item directement dans les fonctions \fonction{check\_module\_data\_form()}
\item à l'aide d'une fonction définie dans le module en question, du type \fonction{check\_module\_update\_rights()}
\end{itemize}

\shadowbox{
\begin{minipage}{13cm}
\begin{verbatim}
function check_group_update_rights($params) {
\end{verbatim}
\end{minipage}
}
\begin{itemize}
 \item \textbf{\$params} : tableau de paramètres.
 \item \textbf{Retourne true || false} : true si les droits sont suffisants
\end{itemize}
\vspace{0.3cm}

Cette fonction doit être appelée :
\begin{itemize}
\item dans toute action nécessitant la vérification détaillée des droits
\item dans la fonction (\fonction{update\_module\_action()}) qui met à jour les menus contextuels
\end{itemize}


\paragraph{Exemple d'utilisation (\fichier{module\_index.php})} :
\begin{verbatim}
} else if ($action == "detailupdate") {
///////////////////////////////////////////////////////////////////////////////
  if (check_group_update_rights($params)) {
    $obm_q = run_query_group_detail($params["group_id"]);
    $display["detail"] = html_group_form($action, $obm_q, $params);
  } else {
    $display["msg"] .= display_warn_msg($err['msg']);
  }

\end{verbatim}


\subsubsection{Gestion de la notion de visibilité et conséquences}

Une entité créée en tant que privée n'est visible que par son créateur.
Il faut donc veiller à traiter les entités privés d'un utilisateur lorsque celui-ci doit être supprimé.

Ces entités peuvent être supprimées ou rendues publiques (la fonction run\_query\_delete\_profile() effectue ce travail).


\subsubsection{Modules gérant la notion de visibilité}

\begin{tabular}{|p{3cm}|c|c|c|}
\hline
\textbf{Module} & \textbf{Depuis version} & \textbf{Séparation droits} & \textbf{Depuis version} \\
\hline
contact & 0.4.0 & X & 2.1.0\\
\hline
lead & 1.2 & & \\
\hline
deal & 0.4.0 & &\\
\hline
list & 0.8.2 & &\\
\hline
calendarevent & 0.5.0 & &\\
\hline
todo & 1.0 & &\\
\hline
contract & 0.9.0 & &\\
\hline
document & 0.7.0 & &\\
\hline
group & 0.9.0 & X & 2.0.2 \\
\hline
rgroup & 1.0 & &\\
\hline
\end{tabular}
