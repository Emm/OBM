% Documentation technique d'OBM : Gestion des modèles Odt par défaut
% ALIACOM Vincent Coulette
% $Id$


\subsection{Gestion des modèles Odt par défaut}
\label{of_defaultodttemplate}

La gestion des des modèles Odt par défaut a pour but de pouvoir définir des modèles OpenDocument (mais aussi swx) en tant que modèles par défaut utilisables pour l'export openOffice d'une entité. \\

Les modèles par défaut sont des documents \obm (donc des fichiers swx ou odt précédemment ajoutés en tant que documents \obm) que l'on définit par défaut et de manière spécifique à un module (ou une entité). Ceci permet de proposer une liste de modèles par défaut parmi lesquels on peut choisir simplement et rapidement le modèle à utiliser dans le formulaire d'export. \\

Dans la version courante (1.2), seul le module CV est concerné par cette fonctionnalité d'export et par conséquent par la gestion des modèles par défaut, mais d'autres modules sont appelés à implémenter cette fonctionnalité et donc à utiliser les modèles par défaut. \\

Les fonctionnalités de gestion des modèles par défaut sont fournies par le fichier \fichier{of/of\_defaultodttemplate.inc}. \\

\subsubsection{Modèle de stockage des modèles par défaut}

Table des modèles définis par défaut.

\begin{tabular}{|p{3cm}|c|p{4cm}|p{4cm}|}
\hline
\textbf{Champs} & \textbf{Type} & \textbf{Description} \\
\hline
\_id & int 8 & Id du modèle par défaut & \\
\hline
\_entity & varchar 32 & Entité à laquelle est rattaché le modèle \\
\hline
\_document\_id & int 8 & Id du document \obm servant de modèle & \\
\hline
\_label & varchar 64 & Libellé décrivant le modèle & \\
\hline
\end{tabular}
\vspace{0.3cm}

\subsubsection{Implémentation : API fonctions publiques}

\shadowbox{
\begin{minipage}{14cm}
\begin{verbatim}
of_get_defaultodttemplates($entity) {
\end{verbatim}
\end{minipage}
}

Requête de sélection de modèles définis par défaut pour l'entité \$entity. \\

\shadowbox{
\begin{minipage}{14cm}
\begin{verbatim}
of_defaultodttemplates_dis_admin_form($entity, $defaults_q) {
\end{verbatim}
\end{minipage}
}

Affichage de l'écran de consultation, modification, ajout et suppression des modèles par défaut de l'entité \$entity.
Le paramètre \$defaults\_q est le résultat base de données de la requête exécutée par la fonction \fonction{of\_defaultodttemplates\_dis\_admin\_form}. \\
