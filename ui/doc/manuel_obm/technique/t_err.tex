% Documentation technique d'OBM : Gestion des erreurs de formulaire
% AliaSource Pierre Baudracco
% $Id$


\subsection{Gestion des erreurs dans les formulaires}

Pour proposer une ergonomie avancée pour les utilisateurs, \obm implémente la signalisation des champs de formulaires incorrectement saisis ou invalides lors de la modification de données.

Le champ en erreur est signalé d'une couleur différente.

\subsubsection{Principe de fonctionnement}

La fonction d'un module qui vérifie la validité ou la cohérence des données, \fonction{check\_module\_data\_form()}, lorsqu'une erreur est détectée (par exemple adresse email non valide), renseigne la variable globale \variable{\$err} avec le message d'erreur et le champ en erreur.\\

La fonction du module d'affichage du formulaire de modification ajoute le style \textbf{error} au label du champ en erreur, ainsi celui-ci s'affiche de façon distincte.
Tous les champs du formulaire se voient stylés par la classe \variable{\$class[nom\_du\_champ]}, mais celle-ci n'est définie que pour le champ en erreur.\\
Le tableau suivant décrit ce fonctionnement :\\

\begin{tabular}{|p{4cm}|p{10cm}|}
\hline
\textbf{Etape (Ou)} & \textbf{Action}\\
\hline
Vérification des paramètres transmis
\fonction{check\_module\_data\_form()}
&
Renseignement de la variable globale \variable{\$err} :
\begin{itemize}
\item \variable{\$err[``msg'']} : message d'erreur
\item \variable{\$err[``field]} : champ en erreur
\end{itemize}

\begin{minipage}{8cm}
\begin{verbatim}

Exemple:

// MANDATORY: Lastname
if (trim($lname) == "") {
  $err["msg"] = $l_lname_error." : ". $lname;
  $err["field"] = "lastname";
  return false;
}
\end{verbatim}
\end{minipage}\\
\hline
Affichage du formulaire à corriger
\fonction{html\_module\_form()}
&
\begin{itemize}
\item Définition du style erreur pour le champ concerné :
\begin{minipage}{8cm}
\begin{verbatim}
  // Mark the error field
  if ($field != "") {
    $class[$field] = "error";
  }
\end{verbatim}
\end{minipage}
\item Positionnement des styles pour chaque champ :
\begin{minipage}{8cm}
\begin{verbatim}
  <tr>
    <th class=\"$class[lastname]\">$l_lastname</th>
    <td><input name=\"tf_lastname\" size=\"32\" ...
  </tr>
\end{verbatim}
\end{minipage}
\end{itemize}
\\
\hline
\end{tabular}
