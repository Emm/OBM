% Documentation technique d'OBM : Gestion des preferences utilisateur et affichage
% AliaSource Pierre Baudracco
% $Id$

\subsection{Gestion des préférences utilisateurs}

Les préférences utilisateurs sont les préférences spécifiques de l'utilisateur comme par exemple le format d'affichage des dates ou le nombre d'enregistrements par page de résultat de recherche.
Ces préférences sont généralement accessibles et modifiables par l'utilisateur dans le module \settings.

Les fonctionnalités de gestion des préférences utilisateurs sont fournies par le fichier \fichier{of/of\_query.inc}

\subsubsection{Principe général de gestion}

Les préférences utilisateurs sont stockées en base de données dans une table dédiée : UserObmPref.\\

Les préférences d'un utilisateur sont chargées et stockées dans sa session à la connexion de l'utilisateur.\\

Les préférences renseignées pour l'utilisateur d'Id 0, représentent les préférences par défaut.
Depuis \obm 1.0, pour chaque utilisateur, seules les préférences différentes des valeurs par défaut doivent être stockées en base. Sans préférence spécifique, l'utilisateur bénéfice de la préférence par défaut.

La récupération des préférenes d'un utilisateur sélectionn les préférences par défaut plus les préférences de l'utilisateur. Si une préférence spécifique de l'utilisateur existe elle se substitue à la préférence par défaut.\\

Jusqu'aux versions 0.9.x d'\obm chaque préférence devait être stockée pour chaque utilisateur. La mise à jour était plus simple, mais pour un grand nombre d'utilisateurs cela représentait un volume de données très important (900 Mo sur une base de 1,2 Go !).


\subsubsection{Table UserObmPref}


\begin{tabular}{|p{3cm}|c|p{5.4cm}|p{2.6cm}|}
\hline
\textbf{Champs} & \textbf{Type} & \textbf{Description} & \textbf{Commentaire} \\
\hline
\_user\_id & int 8 & Identifiant utilisateur & (0 pour valeur par défaut) \\
\hline
\_option & varchar 50 & Nom de la préférence ou du paramètre & (ex: set\_lang, set\_date,...) \\
\hline
\_value & varchar 50 & Valeur du paramètre & \\
\hline
\end{tabular}


\subsubsection{Gestion des modifications et ordre d'affichage}

Les modifications des préférences d'un utilisateur tiennent compte des valeurs par défaut.

Si la modification spécique à l'utilisateur aboutit à la valeur par défaut de la préférence, l'entrée spécifique à l'utilisateur est supprimée.
Sinon, si l'entrée de l'utilisateur n'existe pas, elle est créée sinon modifiée.


\subsubsection{Implémentation : API fonctions publiques}

\shadowbox{
\begin{minipage}{15cm}
\begin{verbatim}
function get_user_pref($u_id) {
\end{verbatim}
\end{minipage}
}

Récupération des préférences de l'utilisateur \$u\_id.

\shadowbox{
\begin{minipage}{15cm}
\begin{verbatim}
function update_user_pref($user_id, $option, $new_value="") {
\end{verbatim}
\end{minipage}
}
Mise à jour de la préférence \$option pour l'utilisateur \$user\_id avec la valeur \$new\_value.


\subsubsection{Implémentation : fonctions utilitaires internes}

\shadowbox{
\begin{minipage}{15cm}
\begin{verbatim}
function get_one_user_pref($u_id, $option) {
\end{verbatim}
\end{minipage}
}
Récupération d'une préférence utilisateur.\\

\shadowbox{
\begin{minipage}{15cm}
\begin{verbatim}
function run_query_user_pref_update($user_id, $option, $value) {
\end{verbatim}
\end{minipage}
}
Mise à jour en base de données d'une préférence utilisateur.\\

\shadowbox{
\begin{minipage}{15cm}
\begin{verbatim}
function run_query_user_pref_insert($user_id, $option, $value) {
\end{verbatim}
\end{minipage}
}
Insertion en base de données d'une préférence utilisateur.\\

\shadowbox{
\begin{minipage}{15cm}
\begin{verbatim}
function reset_preferences_to_default($user_id="") {
\end{verbatim}
\end{minipage}
}
Re-initialisation des préférences aux valeurs par défaut pour un utilisateur (ou pour tous les utilisateurs si \$user\_id est vide).
Cette fonction réinitialise aussi les préférences de l'utilisateur.

\shadowbox{
\begin{minipage}{15cm}
\begin{verbatim}
function session_load_user_prefs() {
\end{verbatim}
\end{minipage}
}
Chargement en session des préférences d'un utilisateur.





\clearpage
\subsection{Gestion des préférences d'affichage}
\label{display_prefs}

Les préférences d'affichage sont les sélections des champs affichés dans chaque résultat de recherche pour chaque utilisateur. Pour chaque module elle sont accessibes et modifiables par le menu "Affichage".

Les fonctionnalités de gestion des préférences d'affichage sont fournies par le fichier \fichier{of/of\_query.inc}

\subsubsection{Principe général de gestion}

Les préférences d'affichage sont stockées en base de données dans une table dédiée : DisplayPref.\\

Les préférences renseignées pour l'utilisateur d'Id 0, représentent les préférences par défaut.
Depuis \obm 1.0, pour chaque utilisateur, seules les préférences différentes des valeurs par défaut doivent être stockées en base. Sans préférence spécifique, l'utilisateur bénéfice de la préférence par défaut.

La récupération des préférenes d'affichage d'un utilisateur sélectionne les préférences par défaut plus les préférences de l'utilisateur. Si une préférence spécifique de l'utilisateur existe elle se substitue à la préférence par défaut.\\

Jusqu'aux versions 0.9.x d'\obm chaque préférence devait être stockée pour chaque utilisateur. La mise à jour était plus simple, mais pour un grand nombre d'utilisateurs cela représentait un volume de données très important (900 Mo sur une base de 1,2 Go !).


\subsubsection{Table DisplayPref}


\begin{tabular}{|p{3cm}|c|p{5.4cm}|p{2.6cm}|}
\hline
\textbf{Champs} & \textbf{Type} & \textbf{Description} & \textbf{Commentaire} \\
\hline
\_user\_id & int 8 & Identifiant utilisateur & (0 pour valeur par défaut) \\
\hline
\_entity & varchar 32 & Entité concernée (company, contact,...) & \\
\hline
\_fieldname & varchar 64 & Nom du champ & champ réel ou calculé \\
\hline
\_fieldorder & int 8 & Ordre d'affichage & \\
\hline
\_display & int 8 & Etat d'affichage & (0: masqué; 1: affiché; 2: obligatoire)\\
\hline
\end{tabular}


\subsubsection{Champ \_entity}

Le champ entity ne désigne par forcémment un module (\company, \contact,...) mais peut être une vue. Par exemple la liste des contacts dans le module \List (entity=list\_contact) peut être différente de la liste des contacts du module \contact.


\subsubsection{Les champs calculés}

La colonne fieldname doit contenir une valeur retournée par la requète de recherche. Ce peut être un champ de base de données, mais aussi un champ calculé.

L'affichage des champs est générique mais peut être spécialisé par champ (voir la classe of\_display du framework \obm).


\subsubsection{Gestion des modifications et ordre d'affichage}

Les modifications des préférences d'affichage tiennent compte des valeurs par défaut.

Si la modification spécique à l'utilisateur aboutit à la valeur par défaut de la préférence, l'entrée spécifique à l'utilisateur est supprimée.
Sinon, si l'entrée de l'utilisateur n'existe pas, elle est créée sinon modifiée.\\

Dans le cas de l'ordre d'affichage, la modification implique nécessairement 2 champs; le champ modifié directement et le champ déplacé par rebond.
Pour ces  2 champs, si la modification aboutit à la valeur par défaut, l'entrée spécifique à l'utilisateur est supprimée.


\subsubsection{Implémentation : API fonctions publiques}

\shadowbox{
\begin{minipage}{15cm}
\begin{verbatim}
function get_display_pref($u_id, $entity, $all=0) {
\end{verbatim}
\end{minipage}
}

Récupération des préférences d'affichage de l'utilisateur \$u\_id pour l'entité \$entity. Si \$all vaut 0, les préférences sont limitées aux champs visibles, sinon tous les champs sont retournés (par exemple pour afficher l'écran de personnalisation de l'affichage).\\

\shadowbox{
\begin{minipage}{15cm}
\begin{verbatim}
function update_display_pref($params) {
\end{verbatim}
\end{minipage}
}
Mises à jour d'une préférence d'affichage (champs \$params[] : fieldname, fieldstatus, fieldorder) de l'utilisateur connecté pour l'entité \$entity.

\subsubsection{Implémentation : fonctions utilitaires internes}

\shadowbox{
\begin{minipage}{15cm}
\begin{verbatim}
function get_one_display_pref($u_id, $entity, $field) {
\end{verbatim}
\end{minipage}
}
Récupération d'une préférence d'affichage d'après le nom du champ.\\

\shadowbox{
\begin{minipage}{15cm}
\begin{verbatim}
run_query_display_pref_update($u_id, $entity, $fieldname, $status, $order="") {
\end{verbatim}
\end{minipage}
}
Mise à jour en base de données d'une préférence d'affichage.\\

\shadowbox{
\begin{minipage}{15cm}
\begin{verbatim}
run_query_display_pref_insert($u_id, $entity, $fieldname, $status, $order) {
\end{verbatim}
\end{minipage}
}
Insertion en base de données d'une préférence d'affichage.\\

\shadowbox{
\begin{minipage}{15cm}
\begin{verbatim}
run_query_display_pref_level_update($u_id, $entity, $fieldorder, $new_level) {
\end{verbatim}
\end{minipage}
}
Gestion de la modification de l'ordre d'un champ (2 champs impactés) avec gestion de l'effacement ou mise à jour en base.\\

\shadowbox{
\begin{minipage}{15cm}
\begin{verbatim}
function reset_preferences_to_default($user_id="") {
\end{verbatim}
\end{minipage}
}
Re-initialisation des préférences aux valeurs par défaut pour un utilisateur (ou pour tous les utilisateurs si \$user\_id est vide).
Cette fonction réinitialise aussi les préférences de l'utilisateur.


\subsubsection{Exemple d'utilisation}

Gestion des préférences d'affichage du module \company.\\

\shadowbox{
\begin{minipage}{15cm}
\begin{verbatim}
}  elseif ($action == "display") {
///////////////////////////////////////////////////////////////////////////////
  $prefs = get_display_pref($uid, "company", 1);
  $display["detail"] = dis_company_display_pref($prefs);

} else if ($action == "dispref_display") {
///////////////////////////////////////////////////////////////////////////////
  update_display_pref($params);
  $prefs = get_display_pref($uid, "company", 1);
  $display["detail"] = dis_company_display_pref($prefs);

} else if ($action == "dispref_level") {
///////////////////////////////////////////////////////////////////////////////
  update_display_pref($params);
  $prefs = get_display_pref($uid, "company", 1);
  $display["detail"] = dis_company_display_pref($prefs);
\end{verbatim}
\end{minipage}
}
