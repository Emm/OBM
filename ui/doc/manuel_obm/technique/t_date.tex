% Documentation technique d'OBM : Gestion des dates
% AliaSource Pierre Baudracco
% $Id$


\subsection{Gestion des dates}

Les problèmes de format des dates sont génériques et particulièrement sensibles dans une application comme \obm fonctionnant avec différentes bases de données et permettant à chaque utilisateur de sélectionner sa propre langue.\\

Afin de simplifier la gestion des dates \obm définit un cadre avec des règles d'utilisation des dates et une api.

\subsubsection{Cadre général et règles d'utilisation}

\begin{tabular}{|p{7cm}|p{7cm}|}
\hline
\textbf{Règles} & \textbf{Implications} \\
\hline
\obm récupére les dates de la base de données au format \textbf{Timestamp Unix}. & La BD doit savoir retourner une date au format \textbf{Timestamp Unix}.
Voir :
\begin{itemize}
\item \fonction{sql\_date\_format()}
\end{itemize}\\
\hline
\obm propose des fonctions de formattage des dates récupérées tenant compte des paramètres utilisateur.
&
Toute date récupérée de la BD doit utiliser une des fonctions de formattage avant affichage.
Voir :
\begin{itemize}
\item \fonction{date\_format()}
\item \fonction{isodate\_format()}
\item \fonction{datetime\_format()}
\end{itemize}\\
\hline
\obm reçoit les dates à insérer en BD au format ISO ``\textbf{AAAA-MM-JJ HH:MM:SS}''
&
La BD doit accepter en entrée des dates au format ISO.
La fonction \fonction{calendar()} permet de générer les champs de saisie de date. Un champ de date est divisé en deux parties:
\begin{itemize}
\item Un champ de saisie
\item Un popup calendrier
\end{itemize}
La fonction javascript \fonction{live\_check\_date()} permet de convertir la date saisie au format ISO. Ces fonctions sont définies dans le fichier \fichier{calendar\_js.inc}
\\
\hline
OBM acceptant en simultané des langues différentes pour les utilisateurs, propose des labels pour le nom des mois, et jours de semaine.
&
Voir :
\begin{itemize}
\item \variable{\$l\_monthsofyear}
\item \variable{\$l\_monthsofyearshort}
\item \variable{\$l\_daysofweek}
\item \variable{\$l\_daysofweekshort}
\item \variable{\$l\_daysofweekfirst}
\end{itemize}\\
\hline
\end{tabular}


\subsubsection{Récupération de date au format Timestamp Unix}

\paragraph{Implémentation multi-base de données}

\fonction{sql\_date\_format()} définie dans le fichier \fichier{global\_query.inc} :

\shadowbox{
\begin{minipage}{13cm}
\begin{verbatim}
function sql_date_format($db_type, $field, $as="") {
  global $db_type_mysql, $db_type_pgsql;

  if ($db_type == $db_type_mysql) {
    $ret = "UNIX_TIMESTAMP($field)";
    if ($as != "") {
      $ret .= " as $as";
    }
  } elseif ($db_type == $db_type_pgsql) {
    $ret = "EXTRACT (EPOCH from $field)";
    if ($as != "") {
      $ret .= " as $as";
    }
  } else {
    $ret = $field;
  }

  return $ret;
}
\end{verbatim}
\end{minipage}
}
\paragraph{Utilisation} dans une requète :\\

\shadowbox{
\begin{minipage}{13cm}
\begin{verbatim}
  $obm_q = new DB_OBM;
  $db_type = $obm_q->type;
  $datealarm = sql_date_format($db_type, "deal_datealarm", "datealarm");
\end{verbatim}
\end{minipage}
}

\subsubsection{Utilisation du fichier calendar\_js.inc}
Le fichier calendar\_js.inc, disponible dans /obminclude/javascript permet la gestion des champs de saisie date ainsi que de la
popup de saisie de date.

\paragraph{Utilisation} :\\

\shadowbox{
\begin{minipage}{13cm}
\begin{verbatim}
	Syntaxe:
	<script>calendar(nom_du_champ,variable_php [,champ_à_comparer])</script>
	
	Exemple:
	<script>calendar('tf_date_begin','$datebegin', 'tf_date_end')</script>
\end{verbatim}
\end{minipage}
}

Dans cette exemple, le nom du champ de saisie sera tf\_date\_begin et sera affecté à la variable \$datebegin. \\
Le troisième paramètre, optionel, est le champ à laquel la date saisie sera comparé. Si la date de saisie est supérieur à la date comparé, cette dernière prendra la valeur de la date saisie. \\

Lors de la saisie d'une date, la fonction \fonction{live\_check\_date()} transforme la date au format ISO si son format est valide (en fonction des préférences utilisateur) et si la date est valide. \\
Dans le cas de non validité, l'application retourne un message d'erreur à l'utilisateur.