% Documentation technique d'OBM : module Contrat
% ALIASOURCE Nourdine Bouaghaz
% $Id$


\clearpage
\section{Contrat}

révision : \obm 2.0 \\
Le module \contract \obm.\\


\subsection{Organisation de la base de données}

Le module \contract utilise 4 tables :
\begin{itemize}
 \item Contrat
 \item ContractPriority
 \item ContractStatus
 \item ContractType
\end{itemize}


\subsection{Contract}
Table principale des informations d'un contrat.\\

\begin{tabular}{|p{3cm}|c|p{5.4cm}|p{2.6cm}|}
\hline
\textbf{Champs} & \textbf{Type} & \textbf{Description} & \textbf{Commentaire} \\
\hline
\_id & int 8 & Identifiant & Clé primaire \\
\hline
\_timeupdate & timestamp 14 & Date de mise à jour & \\
\hline
\_timecreate & timestamp 14 & Date de création & \\
\hline
\_userupdate & int 8 & Id du modificateur & \\
\hline
\_usercreate & int 8 & Id du créateur & \\
\hline
\_deal\_id & int 8 & Affaire liée au contrat & Facultatif \\
\hline
\_company\_id & int 8 & Société contractante & Client ou fournisseur \\
\hline
\_label & varchar 128 & Nom du contrat & \\
\hline
\_number & varchar 20 & Numéro ou référence du contrat & \\
\hline
\_datesignature & date & Date de signature & \\
\hline
\_datebegin & date & Date de début & \\
\hline
\_dateexp & date & Date d'expiration & \\
\hline
\_daterenew & date & Date de renouvellement & \\
\hline
\_datecancel & date & Date de résiliation & Aspect client \\
\hline
\_type\_id & int 8 & Type de contrat & \\
\hline
\_priority\_id & int 8 & Priorité du contrat & \\
\hline
\_status\_id & int 8 & Etat du contrat & (En cours, terminé)\\
\hline
\_kind & int 2 & Indicateur : Client ou fournisseur & \$cck\_customer=0, \$cck\_supplier=1\\
\hline
\_format & int 2 & Indicateur : par période, durée ou coupons & \$ccf\_period=0, \$ccf\_duration=1, \$ccf\_ticket=2\\
\hline
\_ticketnumber & int 8 & Nombre de coupons & (cas contrat par coupons)\\
\hline
\_duration & int 8 & Durée du contrat (en Heure) & (cas contrat par durée)\\
\hline
\_autorenewal & int 2 & Indicateur de renouvellement automatique & 1 = oui\\
\hline
\_contact1\_id & int 8 & Contact 1 chez le contractant &\\
\hline
\_contact2\_id & int 8 & Contact 2 chez le contractant &\\
\hline
\_techmanager\_id & int 8 & Responsable technique interne &\\
\hline
\_marketmanager\_id & int 8 & Responsable commercial interne &\\
\hline
\_privacy & int 2 & Visibilité du contrat & 0=public, 1=privé\\
\hline
\_archive & int 1 & Indicateur d'archivage & (1 = 0ui)\\
\hline
\_clause & text (64k) & Clauses particulières &\\
\hline
\_comment & text (64k) & Commentaire &\\
\hline
\end{tabular}
	 
	
\subsection{ContractType}

Table de catégorisation de contrats (table de référence).

\begin{tabular}{|p{3cm}|c|p{5.4cm}|p{2.6cm}|}
\hline
\textbf{Champs} & \textbf{Type} & \textbf{Description} & \textbf{Commentaire} \\
\hline
\_id & int 8 & Identifiant & Clé primaire \\
\hline
\_timeupdate & timestamp 14 & Date de mise à jour & \\
\hline
\_timecreate & timestamp 14 & Date de création & \\
\hline
\_userupdate & int 8 & Id du modificateur & \\
\hline
\_usercreate & int 8 & Id du créateur & \\
\hline
\_label & varchar 40 & Label & \\
\hline
\end{tabular}


\subsection{ContractPriority}

Table des informations de priorité des contrats (table de référence).\\

\begin{tabular}{|p{3cm}|c|p{5.4cm}|p{2.6cm}|}
\hline
\textbf{Champs} & \textbf{Type} & \textbf{Description} & \textbf{Commentaire} \\
\hline
\_id & int 8 & Identifiant & Clé primaire \\
\hline
\_timeupdate & timestamp 14 & Date de mise à jour & \\
\hline
\_timecreate & timestamp 14 & Date de création & \\
\hline
\_userupdate & int 8 & Id du modificateur & \\
\hline
\_usercreate & int 8 & Id du créateur & \\
\hline
\_color & varchar 6 & Code couleur associé à la priorité \\
\hline
\_order & int 2 & Ordre d'affichage & \\
\hline
\_label & varchar 32 & Label & \\
\hline
\end{tabular}


\subsection{ContractStatus}

Table des états des contrats (table de référence).\\

\begin{tabular}{|p{3cm}|c|p{5.4cm}|p{2.6cm}|}
\hline
\textbf{Champs} & \textbf{Type} & \textbf{Description} & \textbf{Commentaire} \\
\hline
\_id & int 8 & Identifiant & Clé primaire \\
\hline
\_timeupdate & timestamp 14 & Date de mise à jour & \\
\hline
\_timecreate & timestamp 14 & Date de création & \\
\hline
\_userupdate & int 8 & Id du modificateur & \\
\hline
\_usercreate & int 8 & Id du créateur & \\
\hline
\_order & int 2 & Order & \\
\hline
\_label & varchar 32 & Label \\
\hline
\end{tabular}


\subsection{Actions et droits}

Voici la liste des actions du module \contract, avec le droit d'accès requis ainsi qu'une description sommaire de chacune d'entre elles.\\

\begin{tabular}{|l|c|p{9.5cm}|}
 \hline
 \textbf{Intitulé} & \textbf{Droit} & \textbf{Description} \\
 \hline
 \hline
  index & read & (Défaut) formulaire de recherche de contrats. \\ 
 \hline
  search & read & Résultat de recherche de contrats. \\
 \hline
  new & write & Formulaire de création d'un contrat. \\
 \hline
  detailconsult & read & Fiche détail d'un contrat. \\
 \hline
  detailupdate & write & Formulaire de modification d'un contrat. \\
 \hline
  insert & write & Insertion d'un contrat. \\
 \hline
  update & write & Mise à jour du contrat. \\
 \hline
  check\_delete & write & Vérification avant suppression du contrat. \\
 \hline
  delete & write & Suppression du contrat. \\
 \hline
  priorite & write & Liste des priorites définies et formulaire de nouvelle priorite. \\
 \hline
  priorite\_add & write & Ajout d'une priorite . \\
 \hline
  priorite\_update & write & Modification d'une priorite. \\
 \hline
  priorite\_del & write & Suppression d'une priorite. \\
 \hline
  status & write & Liste des etats définies et formulaire de nouveau etat. \\
 \hline
  status\_add & write & Ajout d'un etat . \\
 \hline
  status\_update & write & Modification d'un etat. \\
 \hline
  status\_del & write & Suppression d'un etat. \\
 \hline
  type & write & Liste des types définies et formulaire de nouveau type. \\
 \hline
  type\_add & write & Ajout d'un type. \\
 \hline
  type\_update & write & Modification d'un type. \\
 \hline
  type\_del & write & Suppression d'un type. \\
 \hline
  display & read & Ecran de modification des préférences d'affichage. \\
 \hline
  dispref\_display & read & Modifie l'affichage d'un élément. \\
 \hline
  dispref\_level & read & Modifie l'ordre d'affichage d'un élément. \\
 \hline
  document\_add & write & Ajout de liens vers des documents. \\
 \hline
\end{tabular}


\subsubsection{Catégories d'un contrat}

Catégories d'un contrat.\\

\begin{tabular}{|p{3cm}|p{10cm}|}
\hline\textbf{Code catégorie} & \textbf{Label catégorie} \\
\hline
Priorite & Catégorie où l'on retrouve les priorités des contrats qui ont été inséré dans le module Administration.\\
\hline
Etat & Catégorie où l'on retrouve les états des contrats qui ont été inséré dans le module Administration.\\
\hline
Type & Catégorie où l'on retrouve les types des contrats qui ont été inséré dans le module Administration.\\
\hline

Contract type & Catégorie permettant de choisir entre les contrats de type Client ou Fournisseurs 
Dans la base de données on reserve le champ contract\_kind. Ce champ peut avoir les valeurs
1 (client) ou 2 (fournisseur) ou 0 pour les anciens contrats qui ont été non-actualisé.
Dans le formulaire d'inscription d'un nouveau contrat on trouve dans cette zone deux boutons de type radio
avec le nom de la variable radio\_kind. On a utilise la variable contract\_kind pour garder la variable qui sort de la base de donnée. Si contract\_kind est égale à 2 alors le bouton avec l'etiquette fournisseur doit être selectionné, et dans le cas contraire le bouton avec l'etiquette client doit être selectionné. Le bouton client est selectionné  par ''default''.
Dans le module de consultation on trouve cette catégorie au-dessus du module avec les informations de la société sous la forme d'un labelle 'client' ou 'fournisseur'.
\\
\hline
Format & Catégorie où l'on peut choisir entre les contracts de type Coupons ou Durée.
 Pour la catégorie Coupons, le champ avec le nombre de coupons du contract doit être renseigné.
 Pour la catégorie Duree, le champ avec la duree du support (h) doit être renseigné.
On considere qu'un coupon est égal à une heure de support. Le temps du support du module incident est transformé
 en coupons ou en heures de support pour le module contrat.
La durée disponible du support d'un contrat est égale à la difference entre la durée du support initiale (le nombre 
 d'heures qui ont été vendu) et la durée utilisée (le nombre d'heures qui ont été necessaires pour resoudre les incidents)
Le nombre de coupons disponibles est égal à la difference entre le nombre de coupons initial ( qui ont été vendus) et la durée utilisée (nombre d'heures qui ont été nécessaires pour résoudre le(s) incident(s)).
Pour gérer le type du contract (par coupons ou par duree) nous avons utilisé dans la base de donnée le champ
contract\_ticket , qui peut prendre des valeurs 1 si on a un contrat par coupons ou 2  si on a un contract par durée.
Le nombre de coupons et le nombre d'heures de support vendues sont enregistrés dans la base de données dans les champs:
contract\_ticketnumber et contract\_supportduration.
Les deux champs ne peuvent pas être rempli en même temps avec des valeurs différentes de zéro.
Dans le formulaire d'inscription d'un nouveau contrat, on trouve trois boutons radios 'Période','Duree' et 'Coupons'. Le champ 'Période' est selectionné à l'état initial.
Si le champ 'Coupons' est sélectionné un champ texte permettant la saisie du 'Nombre coupons' apparaît.
En mode consultation ces informations apparaissent dans la zone 'etat': nombre de 'coupons' ou heures' consommées ou disponibles (en foction des heures nécessaires pour la résolution des incidents).
S'il n'y a pas d'incidents alors le nombre d'heures ou de coupons consommés est égal à zéro.

\\
\hline
\end{tabular}
