% Documentation technique d'OBM : Gestion des droits et profils
% AliaSource Pierre Baudracco
% $Id$


\subsection{Gestion des droits et profils}

La gestion des droits est le mécanisme qui permet le contrôle et l'autorisation des informations accessibles et des actions exécutées par un utilisateur.\\

L'implémentation de la gestion des droits a été effectuée avec les objectifs suivants :\\
\begin{itemize}
\item Système peu intrusif dans le code (Eliminer ou éviter au maximum les tests de droits d'accès dans le code des modules)
\item Niveau de granularité à l'action exécutée
\item Faciliter l'évolution (modification des droits, des profils,...)
\item Système léger et performant (si possible sans accès à la base de données)
\item Système sûr.
\end{itemize}
\vspace{0.3cm}

Les fonctionnalitées proposées en standard (sans code spécifique dans un module) :\\
\begin{itemize}
\item Définition des sections et modules accessibles par profil.
\item Autorisation d'exécution au niveau de l'action d'un module par profil
\item Tout accès ou action non défini est interdit
\item Possibilité de tests plus spécifiques dans un module (ex: champ affiché selon droit précis) par utilisation de l'API de droits dans le module.
\item API simple de tests de droits.
\end{itemize}


\subsubsection{Principe des droits}

Chaque action (consultation, création, modification,...) de tous les modules d'\obm nécessite une autorisation pour être exécutée.
La définition des actions des modules précise donc un ``droit élémentaire'' (ex: droit de lecture) nécessaire pour son exécution.\\

Les droits d'un utilisateur sur un module (ou permission sur un module) sont constitués de l'ensemble des droits élémentaires qu'il possède sur ce module (ex: lecture + écriture sur le module \company).
Les droits élémentaires étant définis comme des champs de bits, la permission sur un module est une combinaison logique des droits élémentaires.\\

Les permissions ne sont pas associées directement aux utilisateurs, mais à des profils utilisateurs. Chaque utilisateur d'\obm est associé à un profil.


\subsubsection{Droits élémentaires}

Droits élémentaires définis par \obm :\\

\begin{tabular}{|p{3cm}|p{2cm}|p{2cm}|p{5cm}|}
\hline
\textbf{Droit} & \textbf{bit} & \textbf{Valeur} & \textbf{Description} \\
\hline
\$cright\_read & 0 & 1 & Lecture données simples \\ 
\hline
\$cright\_own & 1 & 2 & Ecriture données personnelles \\ 
\hline
\$cright\_write & 2 & 4 & Ecriture données simples \\ 
\hline
\$cright\_read\_admin & 3 & 8 & Lecture données admin \\ 
\hline
\$cright\_write\_admin & 4 & 16 & Ecriture données admin \\ 
\hline
\end{tabular}

\subsubsection{Permissions : combinaisons de droits élémentaires}

Toute combinaison de droits élémentaires est une permission.
\obm par commodité définit les permissions des profils par défaut :\\

\begin{tabular}{|p{3cm}|p{2.5cm}|p{2.2cm}|p{5cm}|}
\hline
\textbf{Permission} & \textbf{Droits (bits)} & \textbf{Valeur (hex)} & \textbf{Description} \\
\hline
\$perm\_reader & 0 & 01 & Voir données simples \\ 
\hline
\$perm\_user & 0+1 & 03 & Voir données simples + écriture données personnelles \\ 
\hline
\$perm\_editor & 0+2 & 05 & Lecture + écriture données simples \\ 
\hline
\$perm\_admin & 0+2+3+4 & 1D & Tout faire \\ 
\hline
\end{tabular}


\subsubsection{Droits sur un module}

Chaque action d'un module précise un droit élémentaire d'exécution (voir \ref{actions}).
Exemple pour l'action search du module \user qui nécessite le droit \$cright\_read :\\

\shadowbox{
\begin{minipage}{13cm}
\begin{verbatim}
// Search
  $actions["user"]["search"] = array (
    'Url'      => "$path/user/user_index.php?action=search",
    'Right'    => $cright_read,
    'Condition'=> array ('None') 
                                  );
\end{verbatim}
\end{minipage}
}

Si le droit \$cright\_read est inclus dans la permission de l'utilisateur (de son profil) sur ce module, il pourra exécuter l'action. 


\subsubsection{Définition des profils}

Un profil est un ensemble de :
\begin{itemize}
\item Permissions sur des sections et des modules.
\item Propriétés
\end{itemize}

\begin{tabular}{|p{5cm}|p{8cm}|}
\hline
\textbf{Caractéristique} & \textbf{Contenu} \\
\hline
section (permission) & Tableau des sections. Une valeur à 1 indique onglet affiché pour le profil (l'entrée 'default' définit la valeur pour les sections non listées \\ 
\hline
module (permission) & Tableau des modules. Définition des droits à chaque module. L'entrée 'default' définit la valeur pour les modules non listés \\ 
\hline
level & Niveau du profil. Un administrateur ou utilisateur ne peut pas créer / modifier / supprimer un utilisateur de privilège plus élevé (niveau 0 = privilège maximum).\\ 
\hline
level\_managepeers & Capacité du profil à gérer des utilisateurs de même niveau de profil. Si cet indicateur est à true (ou 1), il est possible de gérer des utilisateurs de même niveau, sinon non \\ 
\hline
properties[admin\_realm] & Granularité d'administration (et de possibilité de mise à jour) pour un administrateur. Valeurs possibles : 'user', 'delegation', 'domain'.\\
\hline
\end{tabular}
\vspace{0.3cm}

Une permission par défaut peut être attribuée à un profil pour les sections et les modules.
Exemple de définition de profil :\\

\shadowbox{
\begin{minipage}{13cm}
\begin{verbatim}
$profiles['editor] = array (
  'section' => array (
    'default' => 0,
    'com' => 1,
    'prod' => 1,
    'user' => 1),
  'module' => array (
    'default' => $perm_editor),
  'level' => 3
);
$profiles['admin'] = array (
  'section' => array (
    'default' => 1,
    'com' => 1,
    'prod' => 1,
    'user' => 1),
  'module' => array (
    'default' => $perm_admin),
  'properties' => array (
    'admin_realm' => array ('user', 'delegation', 'domain')),
  'level' => 1,
  'level_managepeers' => 1
);
\end{verbatim}
\end{minipage}
}

Un utilisateur du profil \textbf{editor} pourra accéder aux sections ``com'', ``prod'' et ``user'' (et uniquement celles-ci) et aura les permissions ``\$perm\_editor'' sur tous les modules.\\

\paragraph{Personnalisation des profils}

Les profils par défaut peuvent être modifiés ou supprimés et de nouveaux profils peuvent être définis dans le fichier de configuration.


\subsubsection{Implémentation interne d'\obm}

\paragraph{Définition du droit d'accès à une section}

La définition du droit d'accès nécessaire pour accéder (et voir) une section est précisée dans la définition des sections dans le fichier de configuration (voir section \ref{cgp_show_section}).\\

La configuration est interprétée dans le \fichier{obminclude/global\_pref.inc} et renseigne le hashage \variable{\$sections} utilisé par \obm.


\paragraph{Définition du droit d'accès à un module}

La définition du droit d'accès à un module est précisée dans la définition des modules dans \fichier{obminclude/global\_pref.inc}.
Un droit d'accès unitaire est requis (l'utilisateur doit avoir le droit de lecture \variable{\$cright\_read} sur le module \settings pour y accéder dans l'exemple suivant).\\

\shadowbox{
\begin{minipage}{13cm}
\begin{verbatim}
  if ($cgp_show["module"]["contact"]) {
    $modules["contact"] = array(
                               'Name'=> $l_module_contact,
                               'Ico' => "$ico_contact",
		               'Url' => "$path/contact/contact_index.php",
			       'Right'=> $cright_read);
  }
\end{verbatim}
\end{minipage}
}


\subsubsection{Limitations actuelles du système de droits}

\begin{itemize}
\item Les droits sont donnés sur les actions sans distinction de particularité des données traitées. Il n'est pas possible par exemple de donner le droit de modification de groupes en fonction des caractéristiques du groupe à modifier (ex: pas les groupes système,...). Le droit donné est : modification de groupe OUI ou Non. Ces cas sont donc vérifiés au moment de l'exécution de l'action.

\item L'ajout de profil n'est pas automatiquement pris en compte dans le module \user (pour permettre l'affectation graphique du profil aux utilisateurs, l'apparition dans les sélections...). Cela nécessite du code dans le module.
\end{itemize}