% Manuel d'utilisation d'OBM : module Groupe
% ALIACOM Pierre Baudracco
% $Id$

\clearpage
\section{Le module \group}

\subsection{Présentation du module \group}

Ce module est une gestion de groupes d'utilisateurs.
Il permet de créer des groupes d'utilisateurs qui seront utilisés par d'autres modules comme l'agenda, ou lorsque \obm est couplé à \aliamin comme listes de diffusions ou groupes d'utilisateurs de domaine Windows\\

Les groupes peuvent être privés à un utilisateur ou public.

Un groupe est constitué d'un ensemble de personnes et/ou de groupes (les groupes sont récursifs).\\

\obm gère la notion de groupes locaux (spécifiques à \obm) ou externes pour faciliter les imports ou synchronisations avec d'autres bases utilisateurs / groupes.\\

Ce module comporte une composante unique : l'entité Groupe.


\subsection{Description d'un groupe}

Composantes d'un groupe :\\

\begin{tabular}{|p{3cm}|p{10cm}|}
\hline
\textbf{Nom} & \textbf{Description} \\
\hline
Nom & Nom du groupe \\
\hline
Système & Un groupe système est défini à l'installation ou intégration d'\obm et ne peut être supprimé (ex: le groupe Production). Ces groupes sont utilisés par différents modules d'\obm et sont nécessaires pour le bon fonctionnement de l'aplication. Par défaut \obm définit 3 groupes système (Administration, Commercial et Production).\\
\hline
Privée & Visibilité du groupe qui peut être public ou privé. Un groupe privé n'est visible (et donc modifiable) que par son créateur. Un groupe public n'est supprimable que par son créateur.\\
\hline
Local & Indique si un groupe est local (spécifique à \obm et non issu d'un imort / export par exemple).\\
\hline
Ext\_id & Identifiant externe pour un groupe externe.\\
\hline
Description & Description du groupe.\\
\hline
E-mail & Adresse e-mail du groupe, utile pour utilisation en tant que liste de diffusion.\\
\hline
Utilisateurs membres & Utilisateurs associés au groupe.\\
\hline
Groupes membres & Groupes associés ou membres du groupe.\\
\hline
\end{tabular}


\subsection{La gestion des groupes privés}
\label{u_group_priv}

\obm supporte 2 types de groupes :\\

\begin{tabular}{|p{3cm}|p{10cm}|}
\hline
\textbf{Type} & \textbf{Description} \\
\hline
Public & Groupe visible par tous les utilisateurs. Seul le créateur du groupe a le droit de supprimer le groupe.\\
\hline
Privé & Groupe personnel, visible, modifiable et supprimable uniquement par son créateur.\\
\hline
\end{tabular}
\vspace{0.3cm}

La notion de groupe privé implique que plusieurs groupes peuvent avoir le même nom selon les règles suivantes :\\

\begin{tabular}{|p{8cm}|p{5cm}|}
\hline
\textbf{Groupes} & \textbf{Même nom possible} \\
\hline
2 groupes publics & Non.\\
\hline
1 groupe public et 1 groupe privé & Non.\\
\hline
2 groupes privés & Oui.\\
\hline
\end{tabular}


\subsection{Les groupes locaux}

\obm supporte 2 portées de groupes :\\

\begin{tabular}{|p{3cm}|p{10cm}|}
\hline
\textbf{Portée} & \textbf{Description} \\
\hline
Local & Le groupe a été créé depuis OBM ou n'est destiné qu'à \obm.\\
\hline
Externe & Le groupe provient sans doute d'une source externe (import d'une autre application, annuaire LDAP,..). Le champ Ext\_id est dans ce cas une aide ou référence à un identifiant de la source externe. Ceci permet de gérer par des imports / exports ne prenant en compte que les groupes externes, sans toucher aux groupes locaux.\\
\hline
\end{tabular}


\subsection{Les membres d'un groupe}

Les groupes gérés par \obm sont récursifs, un groupe peut donc avoir simultanément comme membres :
\begin{itemize}
\item Des utilisateurs
\item Des groupes
\end{itemize}


\subsection{Les sous-menus du module \group}

Le module \group comporte 8 sous-menus :

\begin{tabular}{|p{3.5cm}|p{9.5cm}|}
\hline
\textbf{Nom} & \textbf{Action / Description} \\
\hline
Chercher & Recherche multicritères des groupes \\
\hline
Nouveau & Créer un nouveau groupe (local) \\
\hline
Consulter & Consulter les informations du groupe courant \\
\hline
Modifier & Modifier les informations du groupe courant\\
\hline
Supprimer & Vérifier les liens puis si besoin Supprimer un groupe\\
\hline
Ajouter Groupe & Permet d'ajouter des groupes au groupe\\
\hline
Ajouter Utilisateur & Permet d'ajouter des utilisateurs au groupe\\
\hline
Affichage & Personnalisation de l'affichage de la liste des groupes et des membres des groupes\\
\hline
\end{tabular}


\subsubsection{Le sous-menu : Chercher}

Ce menu permet d'effectuer une recherche selon différents critères :

\begin{itemize}
\item Nom
\item Utilisateur membre (Nom de l'utilisateur)
\item E-mail
\item Visibilitée ou accès
\end{itemize}


\subsubsection{Le sous-menu : Nouveau}

Ce menu permet de créer un nouveau groupe.

Un formulaire vierge de création propose la saisie des champs :

\begin{itemize}
\item Nom (obligatoire),
\item Indicateur de groupe privé,
\item Description,
\item E-mail.
\end{itemize}


\subsubsection{Le sous-menu : Consulter}

Ce menu permet de retourner en mode consultation du groupe en cours.


\subsubsection{Le sous-menu : Modifier}

L'écran de modification affiche les informations du groupe courant dans un formulaire.
Il est possible de modifier les informations suivantes, selon les contraintes de nommage et d'accès aux groupes :

\begin{itemize}
\item Nom,
\item Indicateur d'accès (public, privé),
\item Description,
\item Email,
\end{itemize}

Pour valider les modifications effectuées il faut cliquer sur le bouton``valider les modifications''.


\subsubsection{Le sous-menu : Supprimer}
Ce menu permet la suppression d'un groupe (selon les restrictions liées aux groupes système et l'accès aux groupex).


\subsubsection{Le sous-menu : Ajouter Utilisateur}

Ce menu permet d'ajouter des utilisateurs à un groupe.
Il ouvre une fenêtre qui permet de rechercher et sélectionner les utilisateurs à associer.


\subsubsection{Le sous-menu : Ajouter Groupe}

Ce menu permet d'ajouter des groupes à un groupe.
Il ouvre une fenêtre qui permet de rechercher et sélectionner les groupes à associer.

\paragraph{Note :} cette fenêtre est intelligente et ne propose pas d'associer à un groupe des groupes qui seraient parents du groupe cible.


\subsubsection{Le sous-menu : Affichage}

Ce menu permet de paramétrer l'affichage de la liste des groupes et de la liste des utilisateurs membres. Il est possible de choisir les champs à afficher et de régler leur ordre d'affichage.
