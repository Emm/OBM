% Manuel d'utilisation d'OBM : module Incident
% ALIACOM Pierre Baudracco
% $Id$

\clearpage
\section{Le module \incident}

\subsection{Pr�sentation du module \incident}

Ce module est une gestion des incidents d'un contract, souvent appel�e \textbf{....} permettant de g�rer les incidents des contracts avec les clients, fournisseurs, partenaires...\\

Ce module \incident est  utilisee seulument par le module Contrat


De nombreux attributs permettent de d�crire un incident.
Afin de couvrir des utilisations diverses plusieurs attributs peuvent �tre d�sactiv�s et ne pas appara�tre dans l'application.


\subsection{Description d'un incident}

Composantes d'un incident :\\

\begin{tabular}{|p{3cm}|p{10cm}|}
\hline
\textbf{Nom} & \textbf{Description} \\
\hline
Label \\
\hline
R�cepteur\\
\hline
Responsable Technique.\\
\hline
Archiver (selection si un incident doit etre achiver ou pas)\\
\hline
Contrat (liaison avec le contrat , selection de label du contrat).\\
\hline
Priorit�\\
\hline
Etat\\
\hline
Type de solution.\\
\hline
Date \\
\hline
Dur�e totale de r�solution.\\
\hline
Solution apport�e.\\
\hline
Historique.\\
\hline
Ajouter un commentaire.\\
\hline
Dur�e de r�solution (partielle)\\
\hline

\end{tabular}
\vspace{0.3cm}


Plusieurs attributs peuvent �tre d�sactiv�s si non utiles dans une installation d'OBM, et ainsi ne pas appara�tre dans l'application 


\subsubsection{Cat�gories d'un incident}
composantes d'un incident\\

\begin{tabular}{|p{3cm}|p{10cm}|}
\hline \textbf{Label cat�gorie} \\
\hline
Type de solution -categorie ou on retrouvent les types des solutions qui ont ete insere dans le module administration\\
\hline
Priorite -categorie ou on retrouvent les priorites qui ont ete insere dans le module administration\\
\hline
Etat -categorie ou on retrouvent les etats qui ont ete insere dans le module administration\\
\end{tabular}


\subsubsection{Les liens d'un incident}

Le module \incident propose les liens suivants permettant d'acc�der directement aux entit�s li�es � un incident :\\

\begin{itemize}
\item direct : Soci�t� contractante
\item direct : 2 contacts du contrat
\item multiple : Incidents
\item multiple : Documents
\end{itemize}


\subsection{Les sous-menus du module \incident}

Le module \contract comporte  sous-menus (accessibles selon les droits d'acc�s) :\\

\begin{tabular}{|p{2.5cm}|p{9.5cm}|}
\hline
\textbf{Nom} & \textbf{Action / Description} \\
\hline
Chercher & Recherche multicrit�re des incidents \\
\hline
Nouveau & Cr�er un nouveau incident.\\
\hline
Consulter & Consulter les informations d'un incident.\\
\hline
Modifier & Modifier les informations d'un incident.\\
\hline
Supprimer un incident.\\
\hline
Administrer & Administration des informations annexes (type du solution, status, priorite).\\
\hline
Affichage & Personnalisation de l'affichage de la liste des incidents.\\
\hline
\end{tabular}


\subsubsection{Le sous-menu : Chercher}

Ce menu permet d'effectuer une recherche selon diff�rents crit�res :

\begin{itemize}
\item Label
\item Societe
\item Contrat
\item Date Apr�s
\item Date Avant
\item Priorite
\item Etat
\item Responsable Technique
\item Archive
\item Contract kind
\item Archive
\end{itemize}

Il est possible d'inclure les incidents archiv�es dans les r�sultats de recherche.


\subsubsection{Le sous-menu : Nouveau}

Ce menu permet de cr�er un incident.

Pour ins�rer l'incident saisie, il faut cliquer sur le bouton ``Enregistrer l'incident''.


\subsubsection{Le sous-menu : Consulter}

Ce menu permet de consulter la fiche d'un incident .
Cet �cran est aussi accessible depuis les r�sultats de recherche ou les liens depuis d'autres modules en cliquant sur l'incident s�lectionn�, ou immediatement apres l'insertion du nouveau incident.


\subsubsection{Le sous-menu : Modifier}

L'�cran de modification affiche les informations d'incident dans un formulaire.
Toutes les informations sont modifiables.

Pour valider les modifications effectu�es il faut cliquer sur le bouton ``valider les modifications''.


\subsubsection{Le sous-menu : Supprimer}

Ce menu permet supprim�e l'incident.


\subsubsection{Le sous-menu : Administrer}

Ce menu permet l'administration des informations annexes des incidents :\\

\begin{itemize}
\item Type du solution
\item Etat 
\item Priorite
\end{itemize}


\subsubsection{Le sous-menu : Affichage}

Ce menu permet de param�trer l'affichage de la liste des contrats. Il est possible de choisir les champs � afficher et de r�gler leur ordre d'affichage.
