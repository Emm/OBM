% Manuel d'utilisation d'OBM : module Incident
% ALIACOM Pierre Baudracco
% $Id$

\clearpage
\section{Le module \incident}

révision : \obm 2.0

\subsection{Présentation du module \incident}

Ce module est une gestion d'incidents d'un contract, souvent appelée \textbf{tickets} permettant de gérer les incidents des contracts avec les clients, fournisseurs, partenaires...\\

Ce module \incident est utilisé seulement par le module Contrat.\\

De nombreux attributs permettent de décrire un incident.\\
Afin de couvrir des utilisations diverses plusieurs attributs peuvent être désactivés et ne pas apparaître dans l'application.\\

\subsection{Description d'un incident}

Composantes d'un incident :\\

\begin{tabular}{|p{3cm}|p{10cm}|}
\hline
\textbf{Nom} & \textbf{Description} \\
\hline
Label & Label ou titre de l'incident\\
\hline
Récepteur & Personne prévenue de l'incident\\
\hline
Responsable Technique & Responsable technique de l'incident\\
\hline
Archiver & Etat d'archivage de l'incident\\
\hline
Contrat (liaison avec le contrat , selection de label du contrat) & Contrat sur lequel est intervenu l'incident\\
\hline
Priorité & Niveau de priorité dans le traitement de l'incident\\
\hline
Etat & Etat de l'incident\\
\hline
Type de solution & Solution trouvée pour régler l'incident\\
\hline
Date & Date à laquelle l'incident a été detecté\\
\hline
Durée totale de résolution. & Durée estimée pour la résolution de l'incident\\
\hline
Solution apportée & Solution trouvée pour réparer l'incident\\
\hline
Historique & Historique de l'incident\\
\hline
Ajouter un commentaire & Commentaire sur l'état de l'incident. Ce commentaire est daté et signé par son auteur\\
\hline
Durée de résolution (partielle) & Durée de résolution d'une des parties de l'incident\\
\hline

\end{tabular}
\vspace{0.3cm}


Plusieurs attributs peuvent être désactivés si non utiles dans une installation d'OBM, et ainsi ne pas apparaître dans l'application.\\ 


\subsubsection{Catégories d'un incident}

Catégories d'un incident\\

\begin{tabular}{|p{3cm}|p{10cm}|}
\hline \textbf{Code catégorie} & \textbf{Label catégorie} \\
\hline
Type de solution & Catégorie où se trouve les types des solutions qui ont ete insérés dans le module administration.\\
\hline
Priorité & Catégorie où se trouve les priorités qui ont été inserées dans le module administration.\\
\hline
Etat & Catégorie où se trouve les états qui ont été inserés dans le module administration.\\
\hline
\end{tabular}


\subsubsection{Les liens d'un incident}

Le module \incident propose les liens suivants permettant d'accéder directement aux entités liées à un incident :\\

\begin{itemize}
\item direct : Société contractante
\item direct : 2 contacts du contrat
\item multiple : Incidents
\item multiple : Documents
\end{itemize}


\subsection{Les sous-menus du module \incident}

Le module \contract comporte  sous-menus (accessibles selon les droits d'accès) :\\

\begin{tabular}{|p{2.5cm}|p{9.5cm}|}
\hline
\textbf{Nom} & \textbf{Action / Description} \\
\hline
Chercher & Recherche multicritère des incidents \\
\hline
Nouveau & Créer un nouvel incident\\
\hline
Consulter & Consulter les informations d'un incident\\
\hline
Modifier & Modifier les informations d'un incident\\
\hline
Supprimer & Supprimer un incident\\
\hline
Administrer & Administration des informations annexes (type de solution, état, priorité)\\
\hline
Affichage & Personnalisation de l'affichage de la liste des incidents\\
\hline
\end{tabular}


\subsubsection{Le sous-menu : Chercher}

Ce menu permet d'effectuer une recherche selon différents critères :

\begin{itemize}
\item Label
\item Societe
\item Contrat
\item Date Après
\item Date Avant
\item Priorité
\item Etat
\item Responsable Technique
\item Archive
\item Contract kind
\item Archive
\end{itemize}

Il est possible d'inclure les incidents archivés dans les résultats de recherche.


\subsubsection{Le sous-menu : Nouveau}

Ce menu permet de créer un incident.

Pour insérer l'incident saisi, il faut cliquer sur le bouton ``Enregistrer l'incident''.


\subsubsection{Le sous-menu : Consulter}

Ce menu permet de consulter la fiche d'un incident .
Cet écran est aussi accessible depuis les résultats de recherche ou à partir de liens depuis d'autres modules en cliquant sur l'incident sélectionné, ou immédiatement après l'insertion du nouveau incident.


\subsubsection{Le sous-menu : Modifier}

L'écran de modification affiche les informations d'incident dans un formulaire.
Toutes les informations sont modifiables.

Pour valider les modifications effectuées il faut cliquer sur le bouton ``Valider les modifications''.


\subsubsection{Le sous-menu : Supprimer}

Ce menu permet de supprimer l'incident.


\subsubsection{Le sous-menu : Administrer}

Ce menu permet l'administration des informations annexes des incidents :\\

\begin{itemize}
\item Type du solution
\item Etat 
\item Priorite
\end{itemize}


\subsubsection{Le sous-menu : Affichage}

Ce menu permet de paramétrer l'affichage de la liste des incidents. Il est possible de choisir les champs à afficher et de régler leur ordre d'affichage.
