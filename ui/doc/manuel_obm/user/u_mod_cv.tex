% Manuel d'utilisation d'OBM : module CV
% ALIACOM Pierre Baudracco
% $Id$

\clearpage
\section{Le module \cv}

révision : \obm 1.2

\subsection{Présentation du module \cv}

Ce module est un module de gestion et de génération de CV.
Son rôle est de permettre d'établir facilement des CV pour les utilisateurs \obm. Ces CV peuvent ensuite être exportés, en version anonyme ou non, dans le format OpenDocument.\\

Un CV consiste en des informations d'un utilisateur \obm (état civil, formation), un titre et des références Projet. Ces références peuvent être des références à des projets enregistrés dans \obm, auquel cas les informations de description du projet sont reprises dans le CV, ou des références additionelles ainsi qu'un commentaire.\\

Chaque utilisateur peut donc posséder autant de CV que souhaité. Ces CV sont personnalisés (titres et références différentes) en fonction de l'utilisation à laquelle ces CV sont destinés.\\


\subsection{Description d'un CV}

Composantes d'une CV :\\

\begin{tabular}{|p{3cm}|p{10cm}|}
\hline
\textbf{Nom} & \textbf{Description} \\
\hline
Utilisateur & Utilisateur \obm auquel est rattaché le CV \\
\hline
Titre & Titre du CV \\
\hline
Références & Références à des projets d'\obm \\
\hline
Références additionelles & Autres projets non référencés dans \obm \\
\hline
Commentaire & Commentaire sur le CV \\
\hline
\end{tabular}
\vspace{0.3cm}

\subsubsection{Les liens vers CV}

Afin de pouvoir accéder plus rapidement aux CV liés à un utilisateur ou à un projet des liens sont prévus (mais pas encore faits) :\\

\begin{itemize}
\item module Utilisateur : liste des CV de l'utilisateur
\item module Utilisateur : création d'un CV avec l'utilisateur pré-sélectionné
\item module Projet : liste des CV référençant ce projet
\end{itemize}




\subsection{Les sous-menus du module \cv}

Le module \cv comporte 9 sous-menus :\\

\begin{tabular}{|p{2.5cm}|p{9.5cm}|}
\hline
\textbf{Nom} & \textbf{Action / Description} \\
\hline
Chercher & Recherche multicritères des CV\\
\hline
Nouveau & Créer un nouveau CV\\
\hline
Dupliquer & Dupliquer un CV\\
\hline
Consulter & Consulter les informations d'une CV\\
\hline
Modifier & Modifier les informations d'un CV\\
\hline
Supprimer & Vérifier puis si besoin Supprimer un CV\\
\hline
Exporter & Exporter un CV\\
\hline
Administration & Gestion des modèles Odt par défaut\\
\hline
Affichage & Personnalisation de l'affichage de la liste des CV\\
\hline
\end{tabular}


\subsubsection{Le sous-menu : Chercher}

Ce menu permet d'effectuer une recherche sur les CV.

La recherche sur les CV propose les critères suivant :

\begin{itemize}
\item Nom de l'utilisateur
\item Prénom de l'utilisateur
\item Mot-clef dans le titre du CV
\item Mot-clef dans la formation de l'utilisateur
\item Nom d'un projet sélectionné comme référence dans le CV
\item Nom d'une affaire comportant un projet sélectionné comme référence dans le CV
\end{itemize}
\vspace{0.3cm}

\subsubsection{Le sous-menu : Nouveau}

Ce menu permet de créer un CV.

Un écran s'ouvre et permet de sélectioner l'utilisateur, ensuite les informations de l'utilisateur utiles pour un CV sont affichées et un formulaire propose la saisie des champs :

\begin{itemize}
\item Titre (obligatoire),
\item Référence : sélection d'un projet dont les informations intéressantes s'affichent (on peut ajouter ou enlever les références dynamiquement grâce aux boutons et liens disponibles),
\item Référence additionelle : saisie manuelle de références, selon le formattage Date - Projet - Rôle dans le projet - Description - Description technique (on peut ajouter ou enlever les références dynamiquement grâce aux boutons et liens disponibles),
\item Commentaire
\end{itemize}
\vspace{0.3cm}

Notons qu'un contrôle est effectué pour empêcher d'établir plusieurs références pour le même projet.
Si le CV est valide, il est inséré dans la base de données et un message d'acceptation est affiché; sinon le formulaire de saisie de Nouveau CV est affiché de nouveau avec un message indiquant la cause de non validité.


\subsubsection{Le sous-menu : Consulter}

Ce menu permet de retourner en mode consultation du CV en cours.

\subsubsection{Le sous-menu : Dupliquer}

Ce menu permet de dupliquer un CV : il permet de créer un nouveau CV (avec sélection d'un nouvel utilisateur) qui reprend toutes les informations et les références du CV en cours. Après sélection de l'utilisateur, le CV dupliqué est affiché en mode modification et doit être enregistré pour être effectivement créé.

\subsubsection{Le sous-menu : Modifier}

L'écran de modification affiche les informations du CV dans un formulaire de modification.
Il est possible de modifier toutes les informations disponibles en mode création, excepté l'utilisateur lié au CV.

Pour valider les modifications effectuées il faut cliquer sur le bouton``Modifier''.

\subsubsection{Le sous-menu : Supprimer}

Ce menu permet la suppression d'un CV.

\subsubsection{Le sous-menu : Export}

Ce menu permet d'exporter un CV au format OpenOffice 1.X (sxw) ou au format OpenDocument (odt). Pour cela il propose de choisir un modèle openOffice de deux manières différentes : soit parmi une liste déroulante contenant les modèles par défaut (définis en tant que tels via l'écran d'administration), soit parmi les modules documents \obm en choisissant le document dans l'arborescence des documents \obm.  
Le menu propose également le choix de l'anonymat du CV : s'il est anonyme, seules les initiales et la formation de l'utilisateur apparaissent en tant qu'informations personnelles dans le CV.


\subsubsection{Le sous-menu : Administration}

Ce menu permet d'accéder à la gestion des modèles par défaut : il s'agit de sélectionner des documents \obm comme modèles par défaut. Il est également possible de modifier le label d'un modèle par défaut, ou de le retirer de la liste des modèles par défaut.

\subsubsection{Le sous-menu : Affichage}

Ce menu permet de paramétrer l'affichage de la liste des CV. Il est possible de choisir les champs à afficher et de régler leur ordre d'affichage.
