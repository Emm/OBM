% Manuel d'utilisation d'OBM : module Liste
% ALIACOM Pierre Baudracco
% $Id$

\clearpage
\section{Le module \List}

révision : \obm 2.0.0

\subsection{Présentation du module \List}

Ce module est une gestion de liste de contacts.
Il permet de créer des listes ou cibles de contacts en fonction de nombreux critères portant sur les données de la société, du contact ou des abonnements du contact.\\

A une liste peuvent être associés des contacts directs et des critères permettant de sélectionner des contacts dynamiquement.

Les critères peuvent être saisis graphiquement ou par un mode avancé qui autorise l'utilisation directe de requêtes SQL.\\

La sauvegarde des listes et l'export CSV font de ce module un outil de ciblage complet ouvert vers les autres outils (publipostage,...).

Ce module comporte une composante unique : l'entité Liste.


\subsection{Description d'une liste}

Composantes d'une liste :\\

\begin{tabular}{|p{3cm}|p{10cm}|}
\hline
\textbf{Nom} & \textbf{Description} \\
\hline
Nom & Nom de la liste \\
\hline
Privée & Visibilité de la liste qui peut être publique ou privée. Une liste privée n'est visible que par son auteur.\\
\hline
Sujet & Sujet de la liste.\\
\hline
E-mail & Adresse e-mail de la liste, utile pour utilisation en tant que liste de diffusion.\\
\hline
Mailing activé & Indicateur de restriction de la liste aux contacts étant notés comme OK pour l'envoi d'e-mail.\\
\hline
Contacts directs & Contacts associés directement (statiquement) à la liste.\\
\hline
Requête & Critères de sélection des contacts. La requête supporte deux modes, le mode Normal ou le mode expert.\\
\hline
\end{tabular}


\subsection{Description du champ requête / critères}
\label{u_list_req}

Les critères de ciblage supportent 2 modes de saisie :\\

\begin{tabular}{|p{3cm}|p{10cm}|}
\hline
\textbf{Mode} & \textbf{Description} \\
\hline
Normal & Critères saisis graphiquement par l'interface dédiée \\
\hline
Expert & Une requête SQL est attendue, donc libre de proposer ses critères.\\
\hline
\end{tabular}


\subsubsection{Le mode Normal}
Le mode normal propose de saisir graphiquement les critères de ciblage.

La fenêtre de critères comporte 3 parties : \\

\begin{itemize}
\item Critères relatifs à la société du contact
\item Critères relatifs au contact
\item Critères relatifs aux abonnements du contact
\end{itemize}
\vspace{0.3cm}

\paragraph{Règles de gestion des critères graphiques}.\\

\begin{tabular}{|p{14cm}|}
\hline
Les critères de la même fenêtre sont cumulés et liés par des \textbf{ET} logiques.\\
\hline
Pour les critères à choix multiples (à partir de \obm 1.2.5), les choix sélectionnés pour ce critère sont combinés par des \textbf{OU} logiques (valeur parmis celles sélectionnées).\\
\hline
Entre 2 lignes (ou fenêtre de critères) il est possible de choisir le mode de combinaison logique (\textbf{OU} ou \textbf{ET}) et d'inverser le critère (opérateur \textbf{NOT}).\\
\hline
Un opérateur spécial \textbf{"Exclude"} permet d'ajouter des critères d'exclusions globaux, c'est à dire qui s'appliquent à l'ensemble des critères, peut importe sa position dans la liste des critères.
Lorsque l'opérateur \textbf{Exclude} est sélectionné, l'opérateur logique de la ligne (\textbf{OU}, \textbf{ET}) n'est plus pris en compte.
\\
\hline
\end{tabular}


\paragraph{Exemple :} Si à une requête comprenant plusieurs critères, les contacts appartenant à la société X ne doivent pas faire partie du résultat, il suffit de rajouter la ligne de critères d'exclusion (opérateur "Exclude") avec le critère société X, n'importe ou dans la liste des critères.


\subsubsection{Le mode Expert}

Le mode expert propose la saisie directe d'une requête SQL.\\

Ce mode nécessite donc la connaissance du language SQL et du modèle de la base de données d'OBM pour être utilisé.

\paragraph{Avertissement} Ce mode est réservé aux utilisateurs avertis connaissant le langage SQL et le modèle de base de données d'\obm.
Attention le modèle de la base de données OBM est susceptible d'évoluer, et certaines requêtes peuvent ne plus être fonctionnelles après une mise à jour d'OBM.

\paragraph{La Sécurité} est prise en compte. En effet le mode expert permettant un accès SQL direct à la base, et donc potentiellement la saisie de toute commande SQL, les requêtes saisies sont vérifiées avant leur validation.
Ainsi certains mot clés SQL sont interdits comme : Alter, Drop, Insert, Delete, Update, Create; ou la récupération de certains champs (mots de passe,...).

\paragraph{Les ressources} Une vérification est aussi effectuée sur l'estimation des ressources requises pour la requête afin d'interdire les requêtes trop lourdes, souvent dues à des erreurs. Les paramètres de limite de ressources sont ajustables.


\subsubsection{Changement de mode de critères}

Il est possible à tout moment dans une liste de donnée de basculer de mode de sélection de critères.
Cependant il est important de noter que ceci ne s'effectue pas sans conséquences.
Lors du changement de mode, un message d'alerte est affiché détaillant ces conséquences.\\

\begin{tabular}{|p{3cm}|p{10cm}|}
\hline
\textbf{Bascule} & \textbf{Description} \\
\hline
Normal -> Expert & Les critères saisis sont conservés et transformés en requête SQL qui est reprise comme base pour la requête manuelle. Les critères du mode normal sont sauvegardés dans la liste.\\
\hline
Expert -> Normal & La requête saisie est perdue. Les critères sont récupérés de la dernière bascule Normal -> Expert, soit du dernier état en mode normal.\\
\hline
\end{tabular}
\vspace{0.3cm}

La modification du mode de sélection des critères n'est validée que lorsque la mise à jour de la liste est validée.
En cas de problème ou de mauvaise manipulation, il suffit donc de ne pas valider la mise à jour.
 

\subsection{La consultation d'une liste}
\label{u_list_consult}

L'accès à l'écran de consultation d'une liste s'effectue depuis l'écran de recherche, depuis les bookmarks ou dernières visites ou depuis le menu Consulter.

La consultation d'une liste affiche différents types d'informations :\\

\begin{itemize}
\item les informations générales : le nom de la liste, le sujet, l'email,...
\item la liste des contacts directs sélectionnés
\item la cible complète comprenant les contact correspondant aux critères de recherche plus les contacts directs.
\end{itemize}


\subsection{Les sous-menus du module \List}

Le module \List comporte les sous-menus suivants :

\begin{tabular}{|p{2.5cm}|p{9.5cm}|}
\hline
\textbf{Nom} & \textbf{Action / Description} \\
\hline
Chercher & Recherche multicritères des listes \\
\hline
Nouveau & Créer une nouvelle liste \\
\hline
Consulter & Consulter les informations de la liste courante et du résultat de ciblage \\
\hline
Dupliquer & Dupliquer le contenu d'une liste dans une nouvelle liste \\
\hline
Modifier & Modifier les informations de la liste courante\\
\hline
Supprimer & Vérifier les liens puis si besoin Supprimer une liste\\
\hline
Ajouter Contact & Permet d'ajouter des contacts directs à la liste\\
\hline
Affichage & Personnalisation de l'affichage de la liste des listes et de la cible\\
\hline
\end{tabular}


\subsubsection{Le sous-menu : Chercher}

Ce menu permet d'effectuer une recherche selon différents critères :

\begin{itemize}
\item Nom
\item Contact membre
\item E-mail
\item Responsable
\end{itemize}


\subsubsection{Le sous-menu : Nouveau}

Ce menu permet de créer une nouvelle liste.

Un formulaire vierge de création propose la saisie des champs :

\begin{itemize}
\item Nom (obligatoire),
\item Indicateur de liste privée,
\item Sujet (obligatoire),
\item E-mail,
\item Indicateur de restriction sur contacts activés pour mailing.
\item Mode de requête / critères.
\end{itemize}
\vspace{0.3cm}

Le mode de requête / critères est détaillé dans la section \ref{u_list_req}.


\subsubsection{Le sous-menu : Consulter}

Ce menu permet de retourner en mode consultation de la liste en cours.
Voir la section \ref{u_list_consult} ``Consultation d'une liste''.


\subsubsection{Le sous-menu : Dupliquer}

Ce menu permet de créer une nouvelle liste pré-renseignée avec les données de la liste en cours de consultation.


\subsubsection{Le sous-menu : Modifier}

L'écran de modification affiche les informations de la liste dans un formulaire.
Il est possible de modifier les informations suivantes :

\begin{itemize}
\item Nom,
\item Sujet,
\item Email,
\item Indicateur de restriction sur contacts activés pour mailing.
\item Mode de requête / critères.
\end{itemize}

Pour valider les modifications effectuées il faut cliquer sur le bouton``valider les modifications''.


\subsubsection{Le sous-menu : Supprimer}
Ce menu permet la suppression d'une liste.


\subsubsection{Le sous-menu : Ajouter Contact}

Ce menu permet d'ajouter des contacts directs à une liste.
Il ouvre une fenêtre qui permet de rechercher et sélectionner les contacts à associer.


\subsubsection{Le sous-menu : Affichage}

Ce menu permet de paramétrer l'affichage de la liste des listes et de la liste des contacts résultant du ciblage. Il est possible de choisir les champs à afficher et de régler leur ordre d'affichage.
