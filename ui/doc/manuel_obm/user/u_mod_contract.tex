% Manuel d'utilisation d'OBM : module Incident
% ALIACOM Pierre Baudracco
% $Id$

\clearpage
\section{Le module \contract}

révision : \obm 2.0

\subsection{Présentation du module \contract}

Ce module est une gestion de contracts, qui couplée avec le module \incident offre des fonctionnalités de support intéressantes,souvent appelé ``Help Desk''.
Il permet de gérer toute forme de contrats : contrats clients ou fournisseurs, à échéance, par durée ou par coupons...

\subsection{Description d'un contrat}

Composantes d'un contrat :\\

\begin{tabular}{|p{3cm}|p{10cm}|}
\hline
\textbf{Nom} & \textbf{Description} \\
\hline
Label & Label ou titre du contrat\\
\hline
Numéro & Numéro de contrat\\ 
\hline
Privé & Visibilité du contrat (publique ou privée)\\
\hline
Signature & Date de signature du contrat\\
\hline
Début & Date de début du contrat \\
\hline
Expiration & Date d'expiration du contrat \\
\hline
Renouvellement & Date de renouvellement du contrat \\
\hline
Résiliation & Date de résiliation du contrat \\
\hline
Type & Type de contrat \\
\hline
Responsable commercial &  Responsable commercial du contrat\\
\hline
Responsable technique & Responsable technique du contrat\\
\hline
Priorité &  Priorité de traitement(low,medium,high)\\
\hline
Etat & Etat du contrat (open ou close) \\
\hline
Tacite reconduction &  Clause de tacite reconduction\\
\hline
Archive &  Archivage du contrat\\
\hline
Relation &  Type de relation entre la société et le signataire du contrat (client ou fournisseur)\\
\hline
Format & Format du contrat (période, durée, coupons)\\
\hline
Clauses & Clauses incluses dans le contrat\\
\hline
Commentaire & Commentaire horodaté sur le contrat\\
\hline
\end{tabular}
\vspace{0.3cm}


\subsubsection{Catégories d'un contrat}

\begin{tabular}{|p{3cm}|p{10cm}|}
\hline\textbf{Code catégorie} & \textbf{Label catégorie} \\
\hline
Kind & Catégorie où l'on retrouve les types des contrats qui ont été inséré via l'interface d' ''Administration'' du module\\
\hline
Contract type & Catégorie permettant de choisir entre les contrats de type Client ou Fournisseurs \\
\hline
Contract kind & Catégorie permettant de choisir entre les contrats de type Période, Coupons ou Durée.
 Pour la catégorie Coupons, le champ avec le nombre de coupons du contrat doit être renseigné.
 Pour la catégorie Durée, le champ avec la durée de support (heures) doit être renseigné \\
\hline
\end{tabular}


\subsubsection{Les liens d'un contract}

Le module \contract propose les liens suivants permettant d'accéder directement aux entités liées à un contrat :\\

\begin{itemize}
\item direct : Société contractante
\item direct : 2 contacts du contrat
\item multiple : Incidents
\item multiple : Affaires
\item multiple : Documents
\end{itemize}


\subsection{Les sous-menus du module \contract}

Le module \contract comporte  sous-menus (accessibles selon les droits d'accès) :\\

\begin{tabular}{|p{2.5cm}|p{9.5cm}|}
\hline
\textbf{Nom} & \textbf{Action / Description} \\
\hline
Chercher & Recherche multicritère des contrats \\
\hline
Nouveau & Créer un nouveau contrat\\
\hline
Consulter & Consulter les informations d'un contrat\\
\hline
Modifier & Modifier les informations d'un contrat\\
\hline
Exporter & Exporter les informations d'un contrat.\\
\hline
Supprimer & Vérifier les liens puis si besoin Supprimer un contrat.\\
\hline
Administrer & Administration des informations annexes (types, status, priorite).\\
\hline
Affichage & Personnalisation de l'affichage de la liste des contrats.\\
\hline
\end{tabular}


\subsubsection{Le sous-menu : Chercher}

Ce menu permet d'effectuer une recherche selon différents critères :

\begin{itemize}
\item Numéro
\item Société
\item Date Expiration Avant
\item Date Expiration Après
\item Libellé
\item Type
\item Responsable
\item Priorité
\item Etat
\item Relation
\item Archive
\end{itemize}

Il est possible d'inclure les contrats archivés dans les résultats de recherche.


\subsubsection{Le sous-menu : Nouveau}

Ce menu permet de créer un contrat.\\

Pour insérer le contrat saisi, il faut cliquer sur le bouton ``Enregistrer le contrat''.\\

Si des contrats similaires existent déjà (possédant un nom ressemblant, même racine), avant d'insérer le nouveau contrat, \obm affiche une page d'alerte présentant les contrats similaires et permet de continuer l'insertion ou de l'annuler.\\
Ceci est particulièrement utile lorsque un contrat déjà existant est créé à nouveau, et permet d'éviter la saisie de doublons.\\


\subsubsection{Le sous-menu : Consulter}

Ce menu permet de consulter la fiche d'un contrat (accessible uniquement depuis la fiche de la société en mode mise à jour ou suppression).
Cet écran est aussi accessible depuis les résultats de recherche ou depuis d'autres modules en cliquant sur la société sélectionnée, ou immédiatement apres l'insertion du nouveau contrat.

\subsubsection{Le sous-menu : Modifier}

L'écran de modification affiche les informations du contract dans un formulaire.
Toutes les informations sont modifiables.

Pour valider les modifications effectuées, il faut cliquer sur le bouton ``Valider les modifications''.

\subsubsection{Le sous-menu : Exporter}

Ce menu permet l'export du contrat sur un format .txt (l'export est aussi multilangue).


\subsubsection{Le sous-menu : Supprimer}

Ce menu permet la vérification des liens du contrat et d'indiquer si celle-ci est supprimable.
Si le contract peut être supprimé, un bouton ``Supprimer le contrat'' est alors affiché.

\paragraph{Pour être supprimé},un contrat ne doit pas :\\

\begin{itemize}
\item Avoir des incidents
\end{itemize}


\subsubsection{Le sous-menu : Administrer}

Ce menu permet l'administration des informations annexes des contrats :\\

\begin{itemize}
\item Types
\item Status
\item Priority
\end{itemize}

\subsubsection{Le sous-menu : Affichage}

Ce menu permet de paramétrer l'affichage de la liste des contrats. Il est possible de choisir les champs à afficher et de définir leur ordre d'affichage.
