% Manuel d'utilisation d'OBM : module Document
% ALIACOM Nourdine Bouaghaz
% $Id$

\clearpage
\section{Le module \doc}
 
révision: \obm 2.0
 
\subsection{Présentation du module \doc}

Ce module est une gestion de base documentaire, permettant l'indexation, la gestion et l'archivage de documents...\\

Ce module est utilisé par le module \cv.\\

Plusieurs format de documents sont gérés (pdf,jpg, word...). L'utilisateur peut déterminer le format du document de façon automatique ou manuelle en sélectionnant le type \textbf{mime} dans la liste déroulante(Type Mime).\\

\subsection{Description d'un document}

Composantes d'un document :\\

\begin{tabular}{|p{3cm}|p{10cm}|}
\hline
\textbf{Nom} & \textbf{Description} \\
\hline
Titre & Titre du document\\
\hline
Auteur & Auteur du document. Il décrit la personne qui crée le document.\\
\hline
Privé & Visibilité du document (public ou privé).\\
\hline
Catégorie 1 & 1ère classification du document par catégories. Voir la description des catégories.\\
\hline
Catgégorie 2 & 2ème classification du document par catégories. Voir la description des catgéories.\\
\hline
Type MIME & Type MIME du fichier.\\
\hline
Type de document & Moyen utilisé pour télécharger (lien,répertoire,fichier).\\
\hline
Chemin & Chemin d'accès au document.\\
\hline
Fichier & Fichier cible à télécharger.\\
\hline
\end{tabular}
\vspace{0.3cm}


Les attributs suivant peuvent être désactivés si non utiles dans une installation d'OBM, et ainsi ne pas apparaître dans l'application :

\begin{itemize}
\item Catégories 1
\item Catégories 2
\end{itemize}


\subsubsection{Catégories d'un document}
Les catégories d'un document permettent de classifier et organiser les documents.\\

OBM gère 2 familles de catégories de documents.\\

Un document peut référencer plusieurs catégories dans chacune des 2 familles.\\

Les catégories sont administrables par le menu ``Administration'' du module \doc, et il est donc possible de les définir selon les besoins.\\

Par exemple :\\ 
\textbf{Catégorie 1} Compte rendu de réunion, proposition commerciale, contrat ....\\
\textbf{Catégorie 2} Client, prospect ...\\


\subsection{La gestion des types MIME}

Les types MIME sont gérés via l'interface d'Administration (sous-menu Administration). Un fichier peut avoir un type MIME:\\
- par défaut : dans ce cas, OBM essaie de déterminer automatiquement le type MIME associé au fichier en fonction de son extension\\
- défini et choisi par l'utilisateur dans la liste déroulante : différents types de fichier sont présents dans la liste déroulante (document word, jpg...).\\

OBM associe à chaque type MIME une application à utiliser pour ouvrir le fichier.\\

\subsection{La gestion du dépôt de documents}

Le dépôt de document se fait :\\
- via le bouton ''Chemin'' qui permet de détermine la racine du dépôt de document sur le disque.\\
- via le bouton ''Parcourir'' afin de chercher l'emplacement du fichier à partir de la racine selectionnée.\\

\subsection{Les sous-menus du module \doc}

Le module \doc comporte 9 sous-menus (accessibles selon les droits d'accès et le contexte) :\\

\begin{tabular}{|p{2.5cm}|p{9.5cm}|}
\hline
\textbf{Nom} & \textbf{Action / Description} \\
\hline
Chercher & Recherche multicritères des documents \\
\hline
Arborescence & Consulter l'organisation de l'arborescence des documents.\\
\hline
Nouveau & Créer un nouveau document.\\
\hline
Nouveau répertoire & Créer un nouveau répertoire dans l'arborescence des documents.\\
\hline
Consulter & Consulter les documents disponibles dans la base de documents.\\
\hline
Modifier & Modifier les informations d'un document.\\
\hline
Supprimer & Vérifier les liens du document puis si besoin supprimer le document.\\
\hline
Administrer & Administration des informations annexes (catégories 1 et catégories 2, type MIME).\\
\hline
Affichage & Personnalisation de l'affichage de la liste des documents (possibilité de masquer ou non certains champs).\\
\hline
\end{tabular}


\subsubsection{Le sous-menu : Chercher}

Ce menu permet d'effectuer une recherche selon différents critères :

\begin{itemize}
\item Titre
\item Nom
\item Catégorie 1
\item Catégorie 2
\item Auteur
\item Type Mime
\end{itemize}


\subsubsection{Le sous-menu : Nouveau}

Ce menu permet d'insérer un document dans la base documentaire.

Pour insérer le document crée il faut cliquer sur le bouton ``Enregistrer le document''.

\subsubsection{Le sous-menu : Consulter}

Ce menu permet de consulter la fiche d'un document présent dans la base documentaire.
Cet écran est aussi accessible depuis les résultats de recherche ou depuis le module CV (administration des templates).

\subsubsection{Le sous-menu : Modifier}

L'écran de modification affiche les informations du document dans un formulaire.
Toutes les informations sont modifiables.

L'accès au changement du fichier est donné que si l'utilisateur le demande en cliquant sur le bouton radio ''Changer le fichier''.

Une fois cette option choisie, un écran apparaît et permet à l'utilisateur de modifier les informations relatives au fichier (type MIME, type de document,chemin,fichier).

Pour valider les modifications effectuées il faut cliquer sur le bouton ''Modifier le Document''.

\subsubsection{Le sous-menu : Supprimer}
Ce menu permet la vérification des liens du document et d'indiquer si celui-ci est supprimable.\\
Si le document peut être supprimé, un bouton ''Supprimer le document'' est alors affiché sinon un message d'avertissement montrant les liens existants avec ce document signale que la suppression est impossible.

\paragraph{Pour être supprimé} un document ne doit pas :\\

\begin{itemize}
\item Etre utilisé en tant que template dans le module CV.
\end{itemize}

\subsubsection{Le sous-menu : Administration}

Ce menu permet l'administration des informations annexes des documents :\\

\begin{itemize}
\item Catégorie 1
\item Catégorie 2
\item Type MIME
\end{itemize}

\subsubsection{Le sous-menu : Affichage}

Ce menu permet de paramétrer l'affichage de la liste des documents. Il est possible de choisir les champs à afficher et de régler leur ordre d'affichage.

\subsection{Gestion des types MIME :}

Le type MIME d'un document est défini par 3 informations liées :\\

\begin{itemize}
\item Le label du type MIME, qui une fois choisie donne accès au type MIME du document
\item L'extension, qui une fois choisie donne accès aux extensions de fichier (exemple : .doc,.odt,.gif...) 
\item Le type MIME, qui une fois choisi donne accès à l'application utilisée pour ouvrir le document
\end{itemize}

