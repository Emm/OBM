% Manuel d'utilisation d'OBM : module Utilisateur
% ALIACOM Pierre Baudracco
% $Id$

\clearpage
\section{Le module \user}

\subsection{Présentation du module \user}

Ce module est une gestion d'utilisateurs, central à l'application \obm.
Il permet de gérer les utilisateurs de l'application, définir leur profil ou permission et fait partie du socle de base d'\obm.\\

Lorsque \obm est couplé à \aliamin, la gestion des utilisateurs est déléguée à \aliamin dont c'est le rôle et qui propose une gestion plus riche et étendue.
\aliamin gère tous les attributs gérés par \obm mais ajoute de nombreuses fonctionnalités comme la gestion des informations de messagerie (différentes adresses, quota, redirection, droits, partages,...), d'accès à Internet, d'appartenance aux domaines Windows (répertoire home, script de logon, disque de montage,..) ainsi que les services associés (Postfix, Cyrus, Squid, Samba,..) et le stockage dans un annuaire LDAP.\\

Un utilisateur est associé à un profil qui lui confère les permissions sur l'application \obm.\\

\obm gère la notion d'utilisateurs locaux (spécifiques à \obm) ou externes pour faciliter les imports ou synchronisations avec d'autres bases utilisateurs / groupes.\\

Ce module comporte une composante unique : l'entité Utilisateur.


\subsection{Description d'un utilisateur}

Composantes d'un utilisateur :\\

\begin{tabular}{|p{3cm}|p{10cm}|}
\hline
\textbf{Nom} & \textbf{Description} \\
\hline
Login & Identifiant unique de connexion de l'utilisateur \\
\hline
Mot de passe & Mot de passe de l'utilisateur. Cette donnée n'est jamais affichée. \obm gère différents formes de chiffrage pour le stockage du mot de passe (voir le guide technique).\\
\hline
Archive & Indicateur d'archivage de l'utilisateur. Un utilisateur archivé ne peut plus se connecter à \obm. \\
\hline
Permissions & Profil de l'utilisateur (indiquant les droits sur l'application)\\
\hline
Local & Indique si l'utilisateur est local (spécifique à \obm et non issu d'un imort / export par exemple).\\
\hline
Ext\_id & Identifiant externe pour un utilisateur externe.\\
\hline
Nom & Nom de l'utilisateur.\\
\hline
Prénom & Prénom de l'utilisateur.\\
\hline
Nom & Nom de l'utilisateur.\\
\hline
Nom & Nom de l'utilisateur.\\
\hline
Tél 1 & Téléphone 1 de l'utilisateur.\\
\hline
Tél 2 & Téléphone 2 de l'utilisateur.\\
\hline
Fax 1 & Fax 1 de l'utilisateur.\\
\hline
Fax 2 & Fax 2 de l'utilisateur.\\
\hline
E-mail & Adresse e-mail de l'utilisateur.\\
\hline
Date d'entrée & Date d'entrée de l'utilisateur dans la structure.\\
\hline
Dernier accès & Date de dernière connexion à \obm.\\
\hline
Description & Description de l'utilisateur.\\
\hline
Groupes membres & Groupes auxquels appartient l'utilisateur.\\
\hline
\end{tabular}


\subsection{Les profils ou permissions utilisateur}

\obm gère les droits d'accès à l'application et aux diverses fonctionnalités par l'intermédiaire des profils utilisateur.
Un profil est une somme de droits unitaires sur chaque module de l'application. Chaque utilisateur est associé à un profil dont il hérite les droits.\\

\obm distingue 2 types de données :\\

\begin{tabular}{|p{3cm}|p{10cm}|}
\hline
\textbf{Données} & \textbf{Description} \\
\hline
Non protégée & Donnée consultable et éditable par les utilisateurs ou éditeurs. Par exemple les entités des modules : une société, un contact, une affaire, un événement...\\
\hline
Protégée ou d'administration & Donnée modifiable uniquement par un administrateur. Par exemple :
\begin{itemize}
\item Les tables de référence globales (pays, sources de données,...).
\item Les données de référence d'un module (catégorie de contact, catégorie d'événement de l'agenda,...) toujours définies dans les onglets ``Administration'' de chaque module.
\item Le contenu de la section ``Administration''.
\item Les utilisateurs de l'application.
\end{itemize}\\
\hline
\end{tabular}
\vspace{0.3cm}

La définition des profils est à la charge de l'administrateur d'\obm et se définie généralement durant la phase de spécification de mise en oeuvre de l'application.
\obm propose 3 profils génériques par défaut :\\

\begin{tabular}{|p{3cm}|p{10cm}|}
\hline
\textbf{Profil} & \textbf{Description} \\
\hline
user & Utilisateur en consultation. Uniquement les droits de consultation des données non protégées sur l'ensemble des modules.\\
\hline
editor & Editeur ou utilisateur en consultation/édition. Droits de consultation et d'édition des données non protégées sur l'ensemble des modules.\\
\hline
admin & Administration. Tous les droits de consultation et d'édition (données non protégées et données d'administration) offerts par \obm.\\
\hline
\end{tabular}


\subsection{La gestion du SSO (authentification unique)}

\obm peut s'intégrer à \aliasuite qui propose une authentification SSO (Single Sign On) ou authentification unique pour diverses applications.
Dans ce cas la connexion à l'application est déléguée à une application SSO tierce (CAS, application libre dans le cadre d'\aliasuite) et \obm n'a pas a gérer les utilisateurs.


\subsection{Les utilisateurs locaux}

\obm supporte 2 portées d'utilisateurs :\\

\begin{tabular}{|p{3cm}|p{10cm}|}
\hline
\textbf{Portée} & \textbf{Description} \\
\hline
Local & L'utilisateur a été créé depuis OBM ou n'est destiné qu'à \obm.\\
\hline
Externe & L'utilisateur provient sans doute d'une source externe (import d'une autre application, annuaire LDAP,..). Le champ Ext\_id est dans ce cas une aide ou référence à un identifiant de la source externe. Ceci permet de gérer par des imports / exports ne prenant en compte que les utilisateurs externes, sans toucher aux utilisateurs locaux.\\
\hline
\end{tabular}


\subsection{Les groupes d'appartenance}

La consultation d'un utilisateur affiche les groupes auxquels il appartient.

\paragraph{Note :} : L'utilisateur connecté n'apercoit ici que les groupes visibles pour lui (groupes publics et ses propres groupes privés), il ne voit donc par nécessairement l'ensemble des groupes auxquels l'utilisateur consulté appartient.\\

Il est possible de modifier l'appartenance d'un utilisateurs aux groupes directement depuis la fiche de l'utilisateur.


\subsection{Les sous-menus du module \user}

Le module \user comporte 8 sous-menus :

\begin{tabular}{|p{3.5cm}|p{9.5cm}|}
\hline
\textbf{Nom} & \textbf{Action / Description} \\
\hline
Chercher & Recherche multicritères des utilisateurs \\
\hline
Nouveau & Créer un nouvel utilisateur local \\
\hline
Consulter & Consulter les informations de l'utilisateur \\
\hline
Modifier & Modifier les informations de l'utilisateur\\
\hline
Modifier groupes & Modifier l'appartenance de l'utilisateur aux groupes\\
\hline
Préférences par défaut & Rétablir les préférences par défaut pour l'utilisateur\\
\hline
Supprimer & Vérifier les liens puis si besoin Supprimer un utilisateur\\
\hline
Affichage & Personnalisation de l'affichage de la liste des utilisateurs\\
\hline
\end{tabular}


\subsubsection{Le sous-menu : Chercher}

Ce menu permet d'effectuer une recherche selon différents critères :

\begin{itemize}
\item Login
\item Nom de l'utilisateur
\item Profil ou permissions
\item Adresse e-mail
\end{itemize}


\subsubsection{Le sous-menu : Nouveau}

Ce menu permet de créer un nouvel utilisateur.

Un formulaire vierge de création propose la saisie des champs :

\begin{itemize}
\item Login (obligatoire),
\item Mot de passe,
\item Profil,
\item Nom (obligatoire),
\item Prénom,
\item Tel 1,
\item Tel 2,
\item Fax 1,
\item Fax 2,
\item E-mail.
\item Date d'entrée.
\item Description,
\end{itemize}


\subsubsection{Le sous-menu : Consulter}

Ce menu permet de retourner en mode consultation de l'utilisateur courant.


\subsubsection{Le sous-menu : Modifier}

L'écran de modification affiche les informations de l'utilisateur courant dans un formulaire.
Il est possible de modifier les informations suivantes (uniquement pour les administrateurs) :

\begin{itemize}
\item Login (obligatoire),
\item Indicateur d'archivage,
\item Mot de passe,
\item Profil,
\item Nom (obligatoire),
\item Prénom,
\item Tel 1,
\item Tel 2,
\item Fax 1,
\item Fax 2,
\item E-mail.
\item Date d'entrée.
\item Description,
\end{itemize}

Pour valider les modifications effectuées il faut cliquer sur le bouton ``valider les modifications''.


\subsubsection{Le sous-menu : Modifier groupes}

Ce menu permet la modification de l'appartenance de l'utilisateur aux groupes visibles.\\

Pour valider les modifications effectuées il faut cliquer sur le bouton ``Modifier les groupes de la personne''.


\subsubsection{Le sous-menu : Supprimer}
Ce menu permet la suppression d'un utilisateur (uniquement pour les administrateurs).

\paragraph{Note :} \obm interdit la suppression du dernier administrateur.


\subsubsection{Le sous-menu : Affichage}

Ce menu permet de paramétrer l'affichage de la liste des utilisateurs. Il est possible de choisir les champs à afficher et de régler leur ordre d'affichage.
