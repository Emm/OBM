\subsection{Le module Gestion des temps}

Ce module permet.\\

Les principales entit�s du module \timemanager sont d�crites ci-dessous :

\subsubsection{Description d'une t�che}



\subsubsection{Les sous-menus du module Gestion des temps}

Le module \timemanager comporte 5 sous-menus :

\begin{itemize}
\item Semaine, qui permet de voir l'�tat de remplissage de la gestion des temps pour une semaine choisie, la liste des t�ches d�j� saisies et de saisir d'autres t�ches.
\item Mois, qui permet de visualiser l'�tat de remplissage de la gestion des temps pour un mois complet.
\item Validation, qui permet de chercher des accords par activit�s,
de chercher, consulter et modifier l'arborescence des activit�s de DIOR. 
\item Statistiques, qui permet de consulter, modifier, supprimer et de chercher
les segments d'apr�s leur libell� et d�partement.
\item Affichage, qui permet la cr�ation d'un nouveau segment.

\end{itemize}


\subsubsection{Le sous-menu : Cr�er}

Ce menu permet de cr�er un nouvel accord.
Un formulaire vierge d'accord propose la saisie des champs :

\begin{itemize}
\item Titre,
\item Num�ro,
\item Indicateur de message,
\item Type,
\item D�partements,
\item Segment,
\item Texte li� au r�gime suppl�mentaire,
\item Indicateur du tableau ARRCO libre,
\item Texte du tableau ARRCO libre,
\item du Tableau ARRCO,
\item du Tableau AGIRC.
\end{itemize}

L'insertion du nouvel accord s'effectue en validant le formulaire par
le bouton ``Enregistrer l'Accord''.

Pour �tre valide un nouvel accord doit avoir un num�ro non d�j� utilis� par
un autre accord.

Si l'accord est valide, il est ins�r� dans la base de donn�es et un message
d'acceptation est affich�;
sinon le formulaire de saisie de nouvel accord est affich� de nouveau
avec un message indiquant la cause de non validit�.
