% Manuel d'utilisation d'OBM : module Company
% ALIACOM Pierre Baudracco
% $Id$

\clearpage
\section{Le module \company}

\subsection{Présentation du module \company}

Ce module est une gestion de société, souvent appelée \textbf{pages jaunes} permettant de gérer les clients, fournisseurs, partenaires...\\

C'est un module central dans OBM car de nombreux autres modules utilisent \company (comme Affaire, Projet, Contrat,..).

De nombreux attributs permettent de décrire une société.
Afin de couvrir des utilisations diverses plusieurs attributs peuvent être désactivés et ne pas apparaître dans l'application.

Le module \company bénéficie d'une recherche avancée (recherche phonétique et approchée).

\subsection{Description d'une société}

Composantes d'une société :\\

\begin{tabular}{|p{3cm}|p{10cm}|}
\hline
\textbf{Nom} & \textbf{Description} \\
\hline
Nom & Nom de la société (d'autres noms peuvent être saisis pour les recherches)\\
\hline
Numéro & Un numéro de référence pour la société.\\
\hline
Numéro VAT & Numéro intracommunautaire de la société (obligatoire sur factures).\\
\hline
Source de données & Identification de l'origine de l'information.\\
\hline
Archive & Etat d'archivage de la société. Une société archivée ne pourra plus être modifiée 
% %PC
et n'apparaitra pas dans les résultats de recherche par défaut.\\
\hline
Type & Classification de la société (client, propect, fournisseur,...).\\
\hline
Secteur d'activité & Secteur d'activité de la société.\\
\hline
Code NAF & Code NAF de la société.\\
\hline
Responsable & Responsable interne de cette société (commercial, responsable de compte).\\
\hline
Coordonnées & Adresse, code postal, cedex, ville et pays.\\
\hline
Tel et Fax & Téléphone et fax.\\
\hline
Email & Adresse e-mail de la société (adresse générique).\\
\hline
Web & Adresse web de la société (ne pas saisir http://).\\
\hline
Catégorie & Catégories de la société. Voir la description des catégories.\\
\hline
Commentaire & Commentaire sur la société.\\
\hline
\end{tabular}
\vspace{0.3cm}


Les attributs suivants peuvent être désactivés si non utiles dans une installation d'OBM, et ainsi ne pas apparaître dans l'application :

\begin{itemize}
\item Numéro
\item Numéro VAT Intracommunautaire
\item Type
\item Secteur d'activité
\item Code NAF
\item Catégories
\item Ligne Adresse 3
\item Cedex
\end{itemize}


\subsubsection{Catégories d'une société}
Les catégories d'une société permettent de classifier et organiser les sociétés.
Une société peut référencer plusieurs catégories.

Les catégories sont administrables par la menu ``Administration'' du module \company, et il est donc possible de les définir selon les besoins.

Par exemple : des secteurs géographiques, des secteurs d'activité détaillés, une classification interne (selon des produits, cibles...).

Les catégories peuvent être organisées hiérarchiquement (avec une profondeur de 3 niveaux maximum) par exemple :\\

\begin{tabular}{|p{3cm}|p{10cm}|}
\hline\textbf{Code catégorie} & \textbf{Label catégorie} \\
\hline
1 & Education\\
\hline
1.1 & Universités\\
\hline
1.2 & Ecoles\\
\hline
1.2.1 & Ecoles publiques\\
\hline
1.2.2 & Ecoles privées\\
\hline
1.3 & Lycées\\
\hline
2 & Administrations\\
\hline
3 & Industrie\\
\hline
... & ...\\
\hline
\end{tabular}


\subsubsection{Les liens d'une société}

Le module \company étant central à OBM, de nombreux liens sont disponibles depuis une société et permettent d'accéder directement aux entités liées à cette société :\\

\begin{itemize}
\item Contact
\item Affaire
\item Projet
\item Contrat
\item Document
\end{itemize}

\subsection{Les sous-menus du module \company}

Le module \company comporte 7 sous-menus (accessibles selon les droits d'accès) :\\

\begin{tabular}{|p{2.5cm}|p{9.5cm}|}
\hline
\textbf{Nom} & \textbf{Action / Description} \\
\hline
Chercher & Recherche multicritère et avancée des sociétés \\
\hline
Nouveau & Créer une nouvelle société.\\
\hline
Consulter & Consulter les informations d'une société.\\
\hline
Modifier & Modifier les informations d'une société.\\
\hline
Supprimer & Vérifier les liens puis si besoin Supprimer une société.\\
\hline
Administrer & Administration des informations annexes (types, catégories, secteurs d'activité et codes NAF).\\
\hline
Affichage & Personnalisation de l'affichage de la liste des société.\\
\hline
\end{tabular}


\subsubsection{Le sous-menu : Chercher}

Ce menu permet d'effectuer une recherche selon différents critères :

\begin{itemize}
\item Nom de la société
\item Téléphone
\item Type
\item Catégorie (catégorie exacte ou inclusion des sous-catégories)
\item Secteur d'activité
\item Code NAF
\item Code postal
\item Ville
\item Pays
\item Responsable du compte
\item Source de données
\item Date de dernière modification (avant / après la date)
\item Etat d'archivage
\end{itemize}

Il est possible d'inclure les sociétés archivées dans les résultats de recherche.

\paragraph{Recherche avancée : phonétique et approchée}

La recherche sur le nom de la société bénéficie d'une recherche avancée et s'effectue en 2 étapes :\\

\begin{itemize}
\item Recherche exacte
\item Recherche avancée
\end{itemize}
\vspace{0.3cm}

La recherche avancée cumule une recherche phonétique et approchée.\\

La recherche phonétique permet de chercher sur la signature sonore du nom (par exemple la recherche d'''Alliacom'' trouvera ``Aliacom'', ou la recherche de ``restorant'' trouvera ``restaurant'').\\

La recherche avancée est une recherche ne tenant pas compte de divers caracères (par exemple la recherche d'''edf'' trouvera ``E.D.F'', ou la recherche de ``lautre'' trouvera ``l'autre''). Ce mode de recherche s'appuie sur le champ ``Aussi connue comme'', généré automatiquement par l'application.

% %PC ne devrait-on pas masquer ce champ ?

\subsubsection{Le sous-menu : Nouveau}

Ce menu permet de créer une société.

Pour insérer la société saisie, il faut cliquer sur le bouton ``Enregistrer la société''.

Si des sociétés similaires existent déjà (possédant un nom ressemblant, même racine), avant d'insérer la nouvelle société, \obm affiche une page d'alerte présentant les sociétés similaires et permet de continuer l'insertion ou de l'annuler.
Ceci est particulièrement utile lorsque une société déjà existante est créée à nouveau, et permet d'éviter la saisie de doublons.

Le champ ``aussi connue sous'' est rempli automatiquement avec une version simplifiée du nom.

\subsubsection{Le sous-menu : Consulter}

Ce menu permet de consulter la fiche d'une société (accessible uniquement depuis la fiche de la société en mode mise à jour ou suppression).
Cet écran est aussi accessible depuis les résultats de recherche ou les liens depuis d'autres modules en cliquant sur la société sélectionnée.

\subsubsection{Le sous-menu : Modifier}

L'écran de modification affiche les informations de la société dans un formulaire.
Toutes les informations sont modifiables.

L'association de catégories à la société courante s'effectue via une fenêtre popup listant les catégories sous forme arborescente et permettant de les sélectionner.\\

Pour valider les modifications effectuées il faut cliquer sur le bouton ``valider les modifications''.


\subsubsection{Le sous-menu : Supprimer}
Ce menu permet la vérification des liens de la société et d'indiquer si celle-ci est supprimable.
Si la société peut être supprimée, un bouton ``Supprimer la société'' est alors affiché.

\paragraph{Pour être supprimée} une société ne doit pas :\\

\begin{itemize}
\item Avoir des contacts
\item Etre référencée dans les affaires
\item Etre référencée dans les projets
\item Etre référencée dans les contrats
\item Etre référencée dans les documents.
\end{itemize}

% %PC
Cette fonctionnalité ne doit servir qu'en cas de fausse manipulation suite à la création d'une société. On préfèrera l'option d'archivage pour masquer (dans le mode de consultation standard) une société aux utilisateurs.


\subsubsection{Le sous-menu : Administrer}

Ce menu permet l'administration des informations annexes des sociétés :\\

\begin{itemize}
\item Types
\item Catégories
\item Secteurs d'activité
\item Codes NAF
\end{itemize}

\subsubsection{Le sous-menu : Affichage}

Ce menu permet de paramétrer l'affichage de la liste des sociétés. Il est possible de choisir les champs à afficher et de régler leur ordre d'affichage.
