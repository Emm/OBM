% Manuel d'utilisation d'OBM : module Time manager
% ALIACOM Bastien Continsouzas - Pierre Baudracco
% $Id$

\clearpage
\section{Le module Gestion des temps}

Ce module permet de suivre l'activité des différents utilisateurs et leur participation aux projets en cours.\\

Les principales entités du module \timemanager sont décrites ci-dessous :

\subsection{Description d'une saisie de temps}

Composantes d'une saisie de temps :\\

\begin{tabular}{|p{3cm}|p{10cm}|}
\hline
\textbf{Nom} & \textbf{Description} \\
\hline
Date & Date à laquelle a été effectuée la tâche\\
\hline
Durée & Durée du travail sur la tâche exprimée en fraction de journée\\
\hline
Catégorie & Catégorie du projet. Celles-ci sont divisées en trois groupes :
\begin{itemize}
\item Projets facturés
\item Projets internes R\&D,
\item Tâches non facturées
\end{itemize}
Les catégories de tâches sont gérées dans le module référentiel de la section administration.\\
\hline
Projet & Projet imputé. Cette information n'est présente que pour les catégories de tâches ``projets facturés'' et ``projets internes''\\
\hline
Tâche & Tâche du projet imputé. Cette information n'est présente que pour les catégories de tâches ``projets facturés'' et ``projets internes''\\
\hline
Commentaire & Descriptif optionnel\\
\hline
Statut & Les imputations de temps ont un statut différent en fonction de leur niveau de validation\\
\hline
\end{tabular}
\vspace{0.3cm}


\subsubsection{Description des statuts d'une imputation}

Les imputations ont un statut différent en fonction de leur niveau de validation :\\

\begin{itemize}
\item Non validée : c'est le cas lorsque la tâche vient d'être saisie.
\item Validation utilisateur : elle est réalisée automatiquement lorsqu'une journée est complètement remplie.
\item Validation administrateur elle est réalisée manuellement par l'administrateur qui veut valider le mois d'un utilisateur lorsque celui-ci est entièrement rempli. Il n'est alors plus possible de modifier ces tâches.
\end{itemize}


\subsection{Les sous-menus du module Gestion des temps}

Le module \timemanager comporte 5 sous-menus :

\begin{itemize}
\item Semaine : qui permet de voir l'état de remplissage de la gestion des temps pour une semaine choisie, la liste des tâches déjà saisies et de saisir d'autres tâches.
\item Mois : qui permet de visualiser l'état de remplissage de la gestion des temps pour un mois complet.
\item Validation : qui permet de chercher des accords par activités,
de chercher, consulter et modifier l'arborescence des activités de DIOR. 
\item Statistiques : qui permet de consulter, modifier, supprimer et de chercher
les segments d'après leur libellé et département.
\item Affichage : qui permet de personnaliser l'affichage de la liste de tâches.
\end{itemize}

\subsubsection{Le sous-menu : Semaine}

Ce menu permet d'afficher une vue de la semaine, le formulaire permettant la saisie d'une tâche et la liste des tâches déjà saisies pour la semaine.

\subsubsection{Le sous-menu : Mois}

Ce menu permet d'afficher une vue de l'état de la saisie pour un mois. Elle permet de visualiser les parties du mois pour lesquelles la gestion des temps a été remplie.

\subsubsection{Le sous-menu : Validation}

Ce menu affiche un écran qui permet de visualiser, pour un mois,  l'état de remplissage de la gestion des temps pour tous les utilisateurs. Il est possible de valider le mois pour les personnes qui ont entièrement rempli le mois.

\subsubsection{Le sous-menu : Statistiques}

Ce menu permet de visualiser les informations concernant la répartition du temps de travail entre les différents projets et les différents types de tâches.

\subsubsection{Le sous-menu : Affichage}

Ce menu permet de paramétrer l'affichage de la liste des tâches. Il est possible de choisir les champs à afficher et de régler leur ordre d'affichage.

\subsection{Saisir une tâche}

A partir de l'écran de vue hebdomadaire, il est possible de saisir une tâche pour la semaine en cours.
Le formulaire permet de remplir les informations suivantes :

\begin{itemize}
\item Date : il est possible de saisir un ou plusieurs jours de la semaine.
\item Durée : elle est exprimée en fraction de journées.
\item Catégorie : indique le type de tâche qui a été réalisé.
\item Projet : indique le projet pour lequel cette tâche a été réalisée (sauf s'il s'agit d'une tâche non facturée).
\item Sous-tâche du projet : indique plus précisément pour quelle partie du projet cette tâche a été réalisée.
\item Commentaire : permet de décrire en quelques mots la tâche réalisée (non obligatoire).
\end{itemize}

Afin de faciliter la saisie, l'action sur les listes catégorie ou projet modifie le contenu de ces mêmes listes et de la liste de sous-tâches. Les différents cas de figures sont :

\begin{itemize}
\item Si une catégorie est sélectionnée, seuls les projets correspondant à cette catégorie restent visibles dans la liste de projets. La liste des sous-tâches est également mise à jour.
\item Si le projet sélectionné change, la catégorie adéquate est selectionnée dans la liste des catégories. La liste de sous-tâches est aussi mise à jour pour ne contenir que celles qui sont rattachées au projet choisi.
\item Si la tâche vide est sélectionnée, la liste des projets est mise à jour pour contenir tous les projets auquel participe la personne qui saisit les informations.
\end{itemize}

L'administrateur a la possibilité de visualiser les informations de n'importe quel utilisateur.

\subsection{Visualiser un mois}

La vue mensuelle permet d'avoir une vue plus globale de l'état de remplissage pour le mois en cours. les jours qui n'ont pas été entièrement remplis apparaissent de façon différente des autres et il est possible de se rendre directement à l'écran de vue hebdomadaire de la semaine voulue.

L'administrateur, ici aussi, a la possibilité de visualiser le mois de n'importe quel utilisateur.

\subsection{Valider}

Cette fonctionnalité n'est disponible que pour l'administrateur. Quand un utilisateur a terminé le remplissage des informations pour un mois, l'administrateur peut décider de valider ce mois. Il ne sera alors plus possible de modifier les informations qui ont été validées.

\subsection{Consulter les statistiques}

L'écran de statistiques permet à l'administrateur de consulter une vue de synthèse des activités des utilisateurs d'\obm sur un mois. Il est possible d'afficher la répartition de l'activité d'un ou de plusieurs utilisateurs par projet ou par type de tâche.