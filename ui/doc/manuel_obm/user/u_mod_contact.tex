% Manuel d'utilisation d'OBM : module Contact
% ALIACOM Pierre Baudracco
% $Id$

\clearpage
\section{Le module \contact}

\subsection{Présentation du module \contact}

Ce module est une gestion de contacts, souvent appelée \textbf{pages blanches} permettant de gérer les personnes...\\

C'est un module important dans OBM car de nombreux autres modules utilisent \contact (comme \company, \deal,..).

De nombreux attributs permettent de décrire un contact ou une personne.
Afin de couvrir des utilisations diverses plusieurs attributs peuvent être désactivés et ne pas apparaître dans l'application.

\subsection{Description d'un contact}

Composantes d'un contact :\\

\begin{tabular}{|p{3cm}|p{10cm}|}
\hline
\textbf{Nom} & \textbf{Description} \\
\hline
Nom & Nom de la personne\\
\hline
Prénom & Prénom de la personne\\
\hline
Genre & Langue, genre et en-tête de lettre utilisés pour la personne.\\
\hline
Source de données & Identification de l'origine de l'information.\\
\hline
Archive & Etat d'archivage du contact. Un contact archivé n'apparaît plus en utilisation (recherche,...) normale de l'application.\\
\hline
Société & Société ou institution de la personne (il est possible de créer des sociétés virtuelles du type ``amis'', ``freelance'' pour enregistrer des contacts indépendants,...).
Si le module société n'est pas activé, la donnée société devient une simple zone de texte dans le module \contact.\\
\hline
Fonction & Fonction du contact (dans une liste prédéfinie).\\
\hline
Titre & Titre de la personne (texte libre, contrairement à fonction).\\
\hline
Responsable & Responsable interne de cette personne (commercial, responsable de compte).\\
\hline
Catégorie 1 & 1ère classification du contact par Catégories. Voir la description des catégories.\\
\hline
Catégorie 2 & 2ème classification du contact par Catégories. Voir la description des catégories.\\
\hline
Coordonnées & Adresse, service, code postal, cedex, ville et pays.
Si aucune information n'est diponible au niveau du contact, ces informations sont récupées de la société du contact et préfixées par "(Société) - ".\\
\hline
Tel et Fax & Téléphone et fax. Le téléphone ou le fax de la personne sont affichés entre parenthèses lorsque ils sont issus de la société.\\
\hline
Email & Adresse e-mail du contact (2 adresses sont possibles).\\
\hline
Activé pour mailing & Prise en compte autorisée dans les diffusions ou publipostages.\\
\hline
Commentaire & Commentaire sur le contact.\\
\hline
\end{tabular}
\vspace{0.3cm}


Les attributs suivant peuvent être désactivés si non utiles dans une installation d'OBM, et ainsi ne pas apparaître dans l'application :

\begin{itemize}
\item Fonction
\item Titre
\item Catégories 1
\item Catégories 2
\item Service
\item Ligne Adresse 3 (dépendant de celui du module \company)
\item Cedex (dépendant de celui du module \company)
\end{itemize}


\subsubsection{Catégories d'un contact}
Les catégories d'un contact permettent de classifier et organiser les contacts.

OBM gère 2 familles de catégories de personnes.

Un contact peut référencer plusieurs catégories dans chacune des 2 familles.

Les catégories sont administrables par la menu ``Administration'' du module \contact, et il est donc possible de les définir selon les besoins.

Par exemple : des compétences, des classifications internes (selon les domaines, les revenus, les centres d'intérêts...).


\subsubsection{Les liens d'un contact}

Le module \contact étant central à OBM de nombreux liens sont disponibles depuis un contact et permettent d'accéder directement aux entités liées à cette personne :\\

\begin{itemize}
\item direct : Société du contact
\item multiple : Affaires
\item multiple : Listes (enregistré en tant que contact statique)
\item multiple : Contrats
\item multiple : Documents
\item multiple : Abonnements aux publications
\end{itemize}


\subsection{La gestion des coordonnées des contacts}

Les coordonnées affichées des contacts (téléphone, fax, adresse) sont soit ceux du contact directement soit ceux de la société du contact.\\

Si le contact a une de ses coordonnées non remplie c'est la coordonnée correspondante de la société qui est affichée, avec une indication.\\

L'adresse est traitée dans sa globalité (adresse1 + adresse2 + adresse3 + code postal + ville + cédex + pays).

\paragraph{Par exemple} si le téléphone du contact est vide, c'est le téléphone de la société qui est affiché entre parenthèses.
Si l'adresse (complète) du contact est vide c'est l'adresse de la société qui est affichée, préfixée du texte : "(Société) - ".\\

Ceci est valable dans les fiches de détail et les listes de résultats de recherche de contacts (excepté pour l'export CSV des résultats de recherche pour lequel l'indication n'est pas présente).\\

En mode mise à jour, les coordonnées de la société sont indiquées à la suite des champs de saisie des coordonnées du contact afin de faciliter les modifications en permettant le copier / coller.


\subsection{Les sous-menus du module \contact}

Le module \contact comporte 7 sous-menus (accessibles selon les droits d'accès et le contexte) :\\

\begin{tabular}{|p{2.5cm}|p{9.5cm}|}
\hline
\textbf{Nom} & \textbf{Action / Description} \\
\hline
Chercher & Recherche multicritères des contacts \\
\hline
Nouveau & Créer un nouveau contact.\\
\hline
Consulter & Consulter les informations d'un contact.\\
\hline
Modifier & Modifier les informations d'un contact.\\
\hline
Supprimer & Vérifier les liens puis si besoin Supprimer un contact.\\
\hline
Administrer & Administration des informations annexes (genres, fonctions, catégories 1 et catégories 2).\\
\hline
Affichage & Personnalisation de l'affichage de la liste des contacts.\\
\hline
\end{tabular}


\subsubsection{Le sous-menu : Chercher}

Ce menu permet d'effectuer une recherche selon différents critères :

\begin{itemize}
\item Nom et prénom du contact
\item Téléphone
\item Mail
\item Société
\item Code postal
\item Ville
\item Pays
\item Fonction
\item Catégorie 1
\item Catégorie 2
\item Responsable du compte
\item Etat d'archivage
\end{itemize}

Il est possible d'inclure les contacts archivés dans les résultats de recherche.


\subsubsection{Le sous-menu : Nouveau}

Ce menu permet de créer un contact.

Pour insérer le contact saisi il faut cliquer sur le bouton ``Enregistrer le contact''.

Si des contacts similaires existent déjà (personnes de même nom et prénom, ou de même nom dans la même société), avant d'insérer la nouvelle personne, \obm affiche une page d'alerte présentant les contacts similaires et permet de continuer l'insertion ou de l'annuler.
Ceci est particulièrement utile lorsque un contact déjà existant est créée à nouveau, et permet d'éviter la saisie de doublons.

\paragraph{Créer un contact depuis \company} : Le module \company permet de créer un contact directement (lien "Nouveau Contact").
Les informations : société, adresse et téléphone sont alors renseignées par défaut dans la fiche contact avec les valeurs correspondantes de la société.

\subsubsection{Le sous-menu : Consulter}

Ce menu permet de consulter la fiche d'un contact (accessible uniquement depuis la fiche du contact en mode mise à jour ou suppression).
Cet écran est aussi accessible depuis les résultats de recherche ou les liens depuis d'autres modules en cliquant sur le contact sélectionné.

\subsubsection{Le sous-menu : Modifier}

L'écran de modification affiche les informations du contact dans un formulaire.
Toutes les informations sont modifiables.

L'association de catégories au contact courant s'effectue via une fenêtre popup listant les catégories et permettant de les sélectionner ou en les sélectionnant directement dans la liste affichée.\\

Pour valider les modifications effectuées il faut cliquer sur le bouton ``Modifier le Contact''.


\subsubsection{Le sous-menu : Supprimer}
Ce menu permet la vérification des liens du contact et d'indiquer si celui-ci est supprimable.
Si le contact peut être supprimé, un bouton ``Supprimer le contact'' est alors affiché.

\paragraph{Pour être supprimé} un contact doit pas :\\

\begin{itemize}
\item Etre référencé dans les affaires
\item Etre référencé dans les contrats
\item Etre référencé dans les listes (statiquement)
\item Etre abonné à des publications.
\end{itemize}

\subsubsection{Le sous-menu : Administrer}

Ce menu permet l'administration des informations annexes des contacts :\\

\begin{itemize}
\item Genres (avec la langue, le label et l'en-tête de lettre)
\item Fonctions
\item Catégories 1
\item Catégories 2
\end{itemize}

\paragraph{Gestion des genres : } Le genre d'un contact est défini par 3 informations liées :\\

\begin{itemize}
\item La langue du contact, qui une fois choisie donnera accès aux genres labels de cette langue
\item Le label du genre qui peut être ``M., Mme., Mlle. ou autres'', qui une fois choisi donnera accès aux en-têtes de lettre (ou formule de politesse ou présentation) de ce label 
\item L'en-tête de lettre, utilisé par exemple dans les diffusions ou publipostages.
\end{itemize}


\subsubsection{Le sous-menu : Affichage}

Ce menu permet de paramétrer l'affichage de la liste des contacts. Il est possible de choisir les champs à afficher et de régler leur ordre d'affichage.
