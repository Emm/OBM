% Manuel d'utilisation d'OBM : Presentation generale
% ALIACOM Pierre Baudracco
% $Id$

\clearpage
\section{Pr�sentation g�n�rale}

\subsection{La navigation clavier (ACCESSKEY)}

OBM permet de naviguer dans les r�sultats de recherche � l'aide du clavier.
Liste des touches d�finies avec leur action :\\

\begin{tabular}{|p{4cm}|p{8cm}|}
\hline
\textbf{Touche} & \textbf{Action : Aller �} \\
\hline
ALT + B & D�but de r�sultat\\
\hline
ALT + E & Fin de r�sultat\\
\hline
ALT + P & Page pr�c�dente\\
\hline
ALT + N & Page suivante\\
\hline
ALT + 4 & Groupe de page pr�c�dent\\
\hline
ALT + 6 & Groupe de page suivant\\
\hline

\end{tabular}
\vspace{0.3cm}

\subsection{Format de saisie de date}

OBM accepte plusieur format de saisie de date, en fonction de la pr�f�rence utilisateur:
\begin{itemize}
\item ISO (AAAA-MM-JJ)
\item JJ/MM/A
\item JJ-MM-A
\item JJ.MM.A
\item JJMMA
\item MM/JJ/A
\item MM-JJ-A
\item MM.JJ.A
\item MMJJA
\end{itemize}

L'ann�e peut �tre saisie sur 2 ou 4 caract�res  (ex : 2006 ou 06).
Lors de la saisie, la date sera automatiquement convertie au format ISO (AAAA-MM-DD).

