% Manuel d'utilisation d'OBM : Presentation generale
% ALIACOM Pierre Baudracco
% $Id$

\clearpage
\section{Présentation générale}

\subsection{La navigation clavier (ACCESSKEY)}

OBM permet de naviguer dans les résultats de recherche à l'aide du clavier.
Liste des touches définies avec leur action :\\

\begin{tabular}{|p{4cm}|p{8cm}|}
\hline
\textbf{Touche} & \textbf{Action : Aller à} \\
\hline
ALT + B & Début de résultat\\
\hline
ALT + E & Fin de résultat\\
\hline
ALT + P & Page précédente\\
\hline
ALT + N & Page suivante\\
\hline
ALT + 4 & Groupe de page précédent\\
\hline
ALT + 6 & Groupe de page suivant\\
\hline
\end{tabular}


\subsection{Gestion des dates}

Au niveau des dates, \obm distingue les dates affichées et la saisie de dates.
Chaque utilisateur, via son écran de configuration ou préférences, peut configurer :\\
\begin{itemize}
\item le format d'affichage des dates
\item le format de saisie des dates
\end{itemize}

\subsubsection{Format d'affichage des dates}

Les formats d'affichage proposés sont listés dans le tableau ci-dessous :\\

\begin{tabular}{|p{4cm}|p{4cm}|p{4cm}|}
\hline
\textbf{Format d'affichage} & \textbf{Exemple résultat} & \textbf{Description} \\
\hline
yyyy-mm-dd & 2006-03-29 & format Iso\\
\hline
mm/dd/yyyy & 03/29/2006 & format anglais\\
\hline
dd/mm/yyyy & 29/03/2006 & format français\\
\hline
dd mois yyyy & 29 mars 2006 & format texte\\
\hline
\end{tabular}

\subsubsection{Format de saisie de date}

Les formats de saisie proposés sont listés dans le tableau ci-dessous :\\

\begin{tabular}{|p{4cm}|p{4cm}|p{4cm}|}
\hline
\textbf{Format de saisie} & \textbf{Exemple de saisie} & \textbf{Description} \\
\hline
tous & 2006-03-29 & format Iso\\
\hline
dmy &
\begin{itemize}
\item 29/03/06
\item 29/03/2006
\item 29-03-06
\item 29-03-2006
\item 29.03.06
\item 29.03.2006
\item 290306
\item 29032006
\end{itemize}
& format jour mois année\\
\hline
mdy &
\begin{itemize}
\item 03/29/06
\item 03/29/2006
\item 03-29-06
\item 03-29-2006
\item 03.29.06
\item 03.29.2006
\item 032906
\item 03292006
\end{itemize}
& format mois jour année\\
\hline
\end{tabular}
\vspace{0.3cm}

L'année peut toujours être saisie sur 2 ou 4 caractères  (ex : 2006 ou 06).\\
Lors de la saisie, la date sera automatiquement convertie au format ISO (AAAA-MM-DD).

