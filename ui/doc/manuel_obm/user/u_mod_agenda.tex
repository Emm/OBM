% Manuel d'utilisation d'OBM : module Agenda
% ALIACOM Mehdi Rande
% $Id$

\clearpage
\section{Le module \agenda}

\subsection{Présentation du module \agenda}

Ce module est un calendrier partagé. Il permet de gérer son calendrier, ainsi que de consulter voire gérer le planning d'autres utilisateurs.

C'est un outil trés riche, sans équivalent dans le libre au niveau fonctionnel et performances.\\

Il propose différentes vues (annuelle, mensuelle, hebdomadaire, quotidienne) toutes consultables en mode multi-utilisateurs.
Un outil de gestion de réunion permet le parcours d'un calendrier d'un groupe de personnes en mode plages disponibles.
Il permet une gestion des droits fine et la délégation de la gestion de son agenda.
La catégorisation des rendez-vous, les rendez-vous privés, les alertes par mail, tous les types de répétition d'événements sont supportés.\\

Un utilisateur peut exporter son agenda au format iCalendar et des connecteurs Evolution et Outlook sont en cours de finalisation pour permettre l'utilisation du calendrier depuis ces outils.\\


Les principales entités du module \agenda sont décrites ci-dessous :

\subsection{Description d'un événement}

Composantes d'un événement :\\

\begin{tabular}{|p{3cm}|p{10cm}|}
\hline
\textbf{Nom} & \textbf{Description} \\
\hline
Titre & Titre de votre évenement.\\ 
\hline
Privé & Visibilité de l'événement qui peut être publique ou privée. Un 
évenement privé n'est visible que par ses participants.\\
\hline
Catégorie & Catégories de l'événement. Les catégories sont gérées dans
la section administration du module \agenda.\\
\hline
Utilisateurs & Liste des participants de l'événement. La valeur entre
paranthèses (A,W ou R) représente l'etat du rendez-vous: A pour accepter, W 
pour
en attente de réponse, R pour refuser.\\
\hline
Date de début & Date et heure de début de l'événement. Dans le cas d'un
événement répétitif, il s'agit de la date de début de la première occurence de
l'événement.\\
\hline
Date de fin & Date et heure de fin de l'événement. Dans le cas d'un
événement répétitif, il s'agit de la date de fin de la première occurence de
l'événement.\\
\hline
Priorité & Décrit le niveau d'importance d'un événement (Basse, Normale,
Haute).\\
\hline
Type de répétition & Voir la description des types de répétition.\\
\hline
Date de fin de répétition & Dans le cas d'un événement répétitif, date jusqu'a
laquelle l'événement se répétera.\\
\hline
Jours de répétition  & Voir la description des types de répétition : type
hebdomadaire.\\
\hline
Description & Description de l'événement.\\
\hline
\end{tabular}
\vspace{0.3cm}


\subsubsection{Type de répétition}

Un événement peut être répétitif, c'est à dire qu'il va se répéter suivant un
fréquence définie lors de sa création. 
Les differentes répétitions possibles sont :\\

\begin{tabular}{|p{3cm}|p{10cm}|}
\hline
\textbf{---} & Pas de répétition\\
\hline
\textbf{quotidienne} & Evénement a lieu tous les jours.\\
\hline
\textbf{hebdomadaire} & Evénement a lieu toutes les semaines. Il est possible de préciser les jours de la semaine pour lesquels l'événement doit être indiqué (ex: le mardi et le jeudi).\\
\hline
\textbf{mensuelle (date)} & Evénement a lieu tous les mois à une date donnée (le 4 de chaque mois par exemple).\\
\hline
\textbf{mensuelle (jour)} & Evénement a lieu tous les mois à un jour donné (exemple tout les seconds lundi du mois).\\
\hline
\textbf{annuelle} & Evénement a lieu tous les ans à la même date.\\
\hline
\end{tabular}


\subsubsection{Les liens d'un événement}

Actuellement un événement est associé à des utilisateurs.
Une des évolutions prévues du module agenda est de pouvoir lier aux événements différentes entités comme des documents, des contacts...


\subsection{Les sous-menus du module \agenda}

Le module \agenda comporte 12 sous-menus (certains contextuels ou accessibles selon les droits d'accès)
:\\

\begin{tabular}{|p{2.5cm}|p{9.5cm}|}
\hline
\textbf{Nom} & \textbf{Action / Description} \\
\hline
Nouveau rdv & Créer un nouvel événement \\
\hline
Consulter & Consulter les informations d'un événement (et changer sa participation à l'événement si on y est invité).\\
\hline
Modifier & Modifier les informations d'un événement.\\
\hline
Supprimer & Supprimer un événement (uniquement pour les administrateurs).\\
\hline
Année & Affichage de la vue annuelle de l'agenda.\\
\hline
Mois & Affichage de la vue mensuelle de l'agenda.\\
\hline
Semaine & Affichage de la vue hebdomadaire de l'agenda.\\
\hline
Jour & Affichage de la vue quotidienne de l'agenda.\\
\hline
Nouvelle réunion & Plannifier une nouvelle réunion. Visualisation des créneaux disponibles pour un ensemble d'utilisateurs.\\
\hline
Gestion des droits & Gérer les permissions de son agenda.\\
\hline
Administrer & Administration des informations annexes (catégories d'événements).\\
\hline
Exporter & Exporter l'agenda au format iCalendar.\\
\hline
\end{tabular}


\subsubsection{Le sous-menu : Nouveau rdv}

Ce menu permet de créer un événement dans l'agenda.
Il permet de renseigner les différentes informations de l'événement.\\

Pour insérer l'événement saisi, cliquer sur le bouton ``Enregistrer le rdv''.\\

La création d'un événement génère un courrier électronique à destination des participants (ayant dans leurs préférences l'envoi d'e-mail activé) les informant de l'événement.\\

A noter :
\begin{description}
\item[Conflits] : L'agenda ne prévient plus lorsque l'événement inséré entre
en conflit avec un autre événement.
\item[Forcer la notification par mail] : Si cette case est cochée tous les
utilisateurs participants à l'événements recevront une notification par mail, y
compris ceux ayant désactivé l'envoi de mails dans leurs préférences.
\item[Groupes] : Selectionner un (ou plusieurs) groupes dans cette liste aura pour
effet de rajouter les membres de ce(s) groupe(s) aux participants de
l'évenement.
\end{description}

\subsubsection{Le sous-menu : Consulter}

Cette fonctionnalité permet de consulter les details d'un événement. Si vous
participez à cet événement vous pouvez également modifiez votre participation
depuis cet écran en précisant votre choix dans la zone "Votre participation à l'événement".

\subsubsection{Le sous-menu : Modifier}

L'écran de modification affiche les informations de l'événement dans un
formulaire. Toutes les informations sont modifiables.\\

Pour valider les modifications effectuées il faut cliquer sur le bouton 
``Valider les modifications''.

\subsubsection{Le sous-menu : Supprimer}

Permet de supprimer un événement. Cette fonctionnalité est uniquement
disponible pour l'administrateur.\\
A noter : Lors de la confirmation de la suppression de l'événement, la
sélection "Confirmer+forcer l'envoi de mail" permet de notifier par mail
tous les participants, y compris ceux ayant désactivé l'envoi de mails dans leurs préférences.

\subsubsection{Le sous-menu : Année}

Ce menu affiche la vue annuelle de votre agenda.
Les événements sont signalés par des carrés de couleur. Un carré signifie la présence d'un ou plusieurs événements pour le jour en question.\\

En cliquant sur l'intitulé d'un mois, l'affichage passe en vue mensuelle sur le mois sélectionné.\\
En cliquant sur la case représentant un jour vous passerez automatiquement
dans la vue quotidienne de ce jour.

\subsubsection{Le sous-menu : Mois}

Ce menu affiche la vue mensuelle de votre agenda.
Pour chaque événement est indiqué l'heure de début et l'heure de fin.
Lorsque la souris survole un événement des informations supplémentaires 
sont indiquées (type de l'événement, titre, description...).\\

Les numéros sur la colonne de gauche représentent les numéros des semaines
des lignes correspondantes.
En cliquant sur ces numéros, l'affichage passe en vue hebdomadaire sur la semaine sélectionnée.\\

En cliquant sur la date d'une case représentant un jour vous passerez
automatiquement dans la vue quotidienne pour le jour sélectionné.\\

En cliquant sur le + d'une case représentant un jour vous accederez au 
formulaire de prise de rendez vous avec la date de début et de fin 
pré-remplies pour ce jour.

\subsubsection{Le sous-menu : Semaine}

Ce menu affiche la vue hebdomadaire de votre agenda.
Pour chaque événement est indiqué le titre de l'événement.\\
Lorsque la souris survole un événement des informations supplémentaires 
sont indiquées (type de l'événement, titre, description...).\\
Le numéro en haut à gauche de la vue représente le numéro de la semaine en
cours.
Si deux événements se superposent, ils seront fusionnés dans la vue
hebdomadaire.\\
En cliquant sur l'intitulé d'un jour vous passerez automatiquement dans la vue
quotidienne pour le sélectionné.\\
En cliquant sur le + à coté de l'intitulé d'un jour vous accederez au 
formulaire de prise de rendez vous avec la date de début et de fin 
pré-remplies pour ce jour.

\subsubsection{Le sous-menu : Jour}

Ce menu affiche la vue quotidienne de votre agenda.
Pour chaque événement est indiqué le titre de l'événement ainsi que ses
heures de début et de fin..\\
Lorsque la souris survole un événement des informations supplémentaires 
sont indiquées (type de l'événement, titre, description...).\\

Si deux événements se superposent, ils seront fusionnés dans la vue hebdomadaire.\\

Le numéro en haut à gauche de la vue représente le numéro de la semaine en cours.
En cliquant sur ce numéro l'affichage passera automatiquement dans
la vue hebdomadaire pour cette semaine.\\
En cliquant sur le + en dessous de l'intitulé d'un jour vous accederez au 
formulaire de prise de rendez vous avec la date de début et de fin 
pré-remplies pour ce jour.

\subsubsection{Le sous-menu : Nouvelle réunion}

La gestion des réunions est un outil de planification ou d'aide à la prise de rendez-vous en proposant un affichage par créneaux disponibles.\\

Cet outil est idéal pour planifier des réunions entre plusieurs utilisateurs
d'\obm.
Ce menu affiche un formulaire proposant de sélectionner les participants membres de la réunion ainsi que la durée et la date approximative souhaitées.\\

Une fois le formulaire validé, l'application affiche une vue hebdomadaire de la date
choisie représentant la disponibilité globale pour la réunion.
Les différentes cases de la vue sont alors représentées de trois couleurs différentes dont la légende est indiquée à l'ecran:
\begin{description}
\item[Libre] Tous les participants sont disponibles à ce moment là et pour la durée souhaitée.
\item[Libre mais restreint] Tous les participants sont disponibles à ce moment là mais  pour une durée restreinte inférieure à la durée souhaitée.
\item[Non disponible] Tous les participants ne sont pas libres à ce moment là.
\end{description}

\subsubsection{Le sous-menu : Gestion des droits}

Ce formulaire vous permettra de gérer les droits de votre agenda.
\begin{description}
\item[Gestion des permissions de lecture de votre agenda] Cette liste présente
les personnes ayant le droit de consulter votre agenda. Cependant ils ne
pourront pas voir le détail de vos rendez-vous de type privé.
\item[Gestion des permissions d'écriture sur votre agenda] Cette liste présente
les personnes ayant le droit de gérer votre agenda. Une personne ayant les
droits d'écriture sur votre agenda pourra accepter ou refuser des rendez-vous pour
vous et définir les droits sur votre agenda.
Les événements pris par une personne possédant les droits d'écriture sur votre agenda
seront automatiquement acceptés.
\end{description}

A noter que tout le monde à le droit de vous proposer un rendez-vous, cette
fonctionnalité ne nécessite aucun droit, libre à vous par la suite d'accepter ou
refuser les rendez-vous que l'on vous propose.\\

Pour gérer les droits de l'agenda d'une autre personne (disponible uniquement si
vous avez les autorisations nécessaires), il vous suffit de selectionner la
personne dont vous voulez gérer les droits dans le portlet \textbf{Gérer les droits
de} et de sélectionner la personne en question.

\subsubsection{Le sous-menu : Administrer}

Ce menu permet l'administration des informations annexes des événements.\\

Il n'est accessible qu'aux administrateurs et permet de gérer les catégories d'événements.

\subsection{Les outils de navigation de l'\agenda}

L'\agenda propose une série d'outils de navigation et d'information offrant une meilleure ergonomie d'utilisation.

Ces outils affichés dans des portlets (petites zones rectangulaires situés sur la gauche de l'\agenda dans le thème par défaut) sont décrits ci-dessous :

\subsubsection{Le portlet : Utilisateurs}

Le premier portlet \textbf{Utilisateurs} indique la liste des utilisateurs dont
l'\agenda est actuellement affiché.\\
Pour chaque utilisateur sont indiqués le nom, le prénom ainsi qu'une couleur permettant de différencier les événements de l'utilisateur dans l'\agenda. La couleur d'un utilisateur n'est pas fixe et varie selon le nombre et les utilisateurs affichés.

\subsubsection{Le portlet : Utilisateurs}

Le second portlet \textbf{Utilisateurs} permet de sélectionner les utilisateurs
dont vous voulez voir l'agenda. La liste ne contient que les personnes dont vous
pouvez consulter l'agenda. De plus pour des raisons de lisibilité vous ne pouvez
consulter au maximum que l'agenda de six personnes simultanément.\\
Lorsque vous consultez l'agenda de personne autres que vous même un lien
\textbf{Revenir à votre vue} vous permet de revenir à la consultation de votre
agenda.

\subsubsection{Le portlet : Mois en cours}

Ce portlet présente une vue simplifié du mois en cours, il vous propose une aide à la navigation en permettant d'accéder rapidement à n'importe quel jour du mois. La
date du jour courant est surlignée.

\subsubsection{Le portlet : Aller à}

Ce portlet est un raccourci de navigation permettant d'accéder rapidement à :
\begin{itemize}
\item[N'importe quelle année dans un intervalle de plus ou moins trois ans par
rapport à l'année en cours.]
\item[N'importe quel mois de l'année en cours]
\item[N'importe quelle semaine de l'année en cours]
\end{itemize}
