\documentclass[french]{article}
%
% Packages supplementaires utilises
%
\usepackage[T1]{fontenc}
\usepackage{babel}
\usepackage{indentfirst}
\usepackage{graphicx}
\usepackage{epsf}
\usepackage{fancybox}
\usepackage{url}
\usepackage{fancyheadings}
\usepackage{xspace}
%
% Definition des commandes personnalis�es
%[1]{\textsf{#1}\xspace
\newcommand{\obm}{\textsc{obm}\xspace}
\newcommand{\calendar}{\textsc{calendar}\xspace}
\newcommand{\css}{\textsc{css}\xspace}
\newcommand{\fonction}[1]{\textit{#1}\xspace}
\newcommand{\fichier}[1]{\textbf{\textit{#1}}\xspace}
\newcommand{\variable}[1]{\textbf{#1}\xspace}
%
% Modification des marges par defaut
%
\addtolength\textwidth{2cm}
%
% Profondeur de la table des matieres
%
\setcounter{tocdepth}{3}
%
%
%
\pagestyle{fancy}
\lhead[ALIACOM]{ALIACOM}
%
%
%
\title{\Huge
\textbf{OBM : Open Buisness Managment}\\
\vspace{1.5cm}
Manuel d'utilisation\\
\vspace{.5cm}
et\\
\vspace{.5cm}
Documentation technique\\
\vspace{1.5cm}
}
\author{ALIACOM}
%\date{}
%
% Debut du document
%
\begin{document}

\maketitle

\clearpage

\tableofcontents

\clearpage

% Documentation technique d'OBM
% ALIACOM Pierre Baudracco
% $Id$

\part{Manuel technique}
\setcounter{section}{0}

% Documentation technique d'OBM : Configuration specifique � un site
% ALIACOM Pierre Baudracco
% $Id$

\section{Configuration sp�cifique d'un site}

Depuis la version 0.8, OBM est pens� pour faciliter la personnalisation d'une mise en oeuvre en restant compatible avec le produit g�n�rique.

Pour ce faire les options de configurations et les donn�es sp�cifiques d'un site sont situ�es dans un fichier et un r�pertoire d�di�s.\\

\begin{tabular}{|p{7cm}|p{5cm}|}
\hline
Fichier de configuration sp�cifique & obminclude/obm\_conf.inc \\
\hline
R�pertoire de configuration sp�cifique & obminclude/site \\
\hline
\end{tabular}

\subsection{Le param�trage des sections visibles et leur lien}

Le tableau \$cgp\_show[``section''] d�finit pour chaque section si celle-ci doit �tre affich�e ou non et pr�cise son lien (url) par d�faut.
Par d�faut toutes les sections sont affich�es et pointent vers le premier module affich�.

\paragraph{La configuration des liens} des sections est disponible depuis OBM 0.8.1.\\

Pour ne pas afficher une section il faut ins�rer dans le fichier de configuration la ligne :

\begin{verbatim}  
$cgp_show[``section''][``com''] = false;
\end{verbatim}  
\vspace{0.3cm}

Pour modifier l'url par d�faut d'une section il faut ins�rer dans le fichier de configuration la ligne (pour faire pointer la section Production vers le module Document) :

\begin{verbatim}  
$cgp_show[``section''][``prod''] = ``$path/document/document_index.php'';
\end{verbatim}  
\vspace{0.3cm}

Liste des sections disponibles :\\
\begin{itemize}
\item com
\item prod
\item compta
\item user
\item admin
\item aliamin
\end{itemize}


\subsection{Le param�trage des modules visibles}

Le tableau \$cgp\_show[``module''] d�finit pour chaque module si celui-ci doit �tre affich� ou non.
Par d�faut tous les modules sont affich�s.

Pour ne pas afficher un module il faut ins�rer dans le fichier de configuration la ligne :

\begin{verbatim}  
$cgp_show[``module''][``deal''] = false;
\end{verbatim}  
\vspace{0.3cm}

Liste des modules disponibles :\\
\begin{itemize}
\item company
\item contact
\item deal
\item list
\item agenda
\item todo
\item publication
\item time
\item project
\item contract
\item incident
\item document
\item account
\item invoice
\item payment
\item settings
\item user
\item group
\end{itemize}


\subsection{Le param�trage des champs visibles}

Afin de permettre des utilisations d'\obm d'orientations diverses, certains modules peuvent �tre param�tr�s afin de ne pas afficher et g�rer tous les champs pr�vus.

A partir de la version 0.8 d'\obm, les modules \company, \contact et \deal ont des champs param�trables.\\

Le tableau \$cgp\_hide[``module''] d�finit pour les modules cibles les champs devant �tre cach�s.
Par d�faut tous les champs sont affich�s et il faut donc pr�ciser explicitement les champs non souhait�s.

Pour ne pas afficher un champ il faut ins�rer dans le fichier de configuration la ligne :

\begin{verbatim}  
$cgp_hide[``company''][``company_number''] = true;
\end{verbatim}  

Lorsqu'un champ est cach�, il n'est plus disponible dans les crit�res de recherche, dans les listes de r�sultats ni dans les fiches de d�tail.

Les noms donn�s aux champs du tableau \$cgp\_hide doivent �tre identiques aux noms des champs retourn�s par la requ�te des listes de r�sultat, car la classe OBM\_DISPLAY utilise ces noms pour d�terminer si les champs doivent �tre affich�s ou non.
Si le champ n'est pas propos� dans un r�sultat de requ�te (cat�gories de soci�t�), il n'y a pas cette contrainte.


\subsubsection{Les champs param�trables du module \company}

Liste des champs param�trables (masquables) du module \company :

\begin{verbatim}
// Company module
$cgp_hide["company"]["company_number"] = true;         // num�ro
$cgp_hide["company"]["company_vat"] = true;            // num�ro intracommunautaire
$cgp_hide["company"]["companytype_label"] = true;      // type
$cgp_hide["company"]["companyactivity_label"] = true;  // secteur d'activit�
$cgp_hide["company"]["companynafcode_code"] = true;    // code NAF
$cgp_hide["company"]["category"] = true;               // cat�gorie
$cgp_hide["company"]["company_address3"] = true;       // ligne adresse 3
\end{verbatim}  


\subsubsection{Les champs param�trables du module \contact}

Liste des champs param�trables (masquables) du module \contact :

\begin{verbatim}
// Contact module
$cgp_hide["contact"]["contact_function"] = true;  // fonction
$cgp_hide["contact"]["contact_title"] = true;     // titre
$cgp_hide["contact"]["category1"] = true;         // cat�gorie 1
$cgp_hide["contact"]["category2"] = true;         // cat�gorie 2
$cgp_hide["contact"]["contact_service"] = true;   // ligne service
$cgp_hide["contact"]["contact_address3"] = true;  // ligne adresse 3
\end{verbatim}  


\subsubsection{Les champs param�trables du module \deal}

Liste des champs param�trables (masquables) du module \deal :

\begin{verbatim}
// Company module
$cgp_hide["deal"]["category1"] = true;                 // cat�gorie
\end{verbatim}  

% Documentation technique d'OBM : utilisation des css
% ALIACOM Medhi Rande
% $Id: 

\subsection{Les feuilles de style - CSS}

Le code produit par OBM est et doit �tre conforme XHTML 1.0 Transitionnal.
Les feuilles de style (CSS) sont utilis�es selon le standard CSS2. (1 ?).\\

OBM d�fini des zones graphiques (bandeau de gauche, bandeau de droite,...) ainsi que des objects graphiques (block des sections, fiche de d�tail, block des derni�res entit�s visit�es,...). Une fonction g�n�rale d'affichage \textbf{display\_page()}, sp�cifique � chaque th�me indique la r�partition des objets dans les zones.


\subsubsection{Les zones graphiques d�finies dans le th�me standard}

L'affichage d'OBM est effectu� dans des zones d�finies.
Le th�me standard d�fini les zones :\\

\begin{tabular}{|c|l|}
\hline
\textbf{Zone} & \textbf{Description pour th�me standard}\\
\hline
leftpanel & Bandeau de gauche \\
\hline
detailpanel & Zone principale d'affichage, soit unique soit d�coup�e en middlepanel et rightpanel\\
\hline
middlepanel & Zone centrale d'affichage quand un bandeau est pr�sent � droite (rightpanel)\\
\hline
rightpanel & Bandeau de droite. Si non vide, alors affich� ainsi que middle panel\\
\hline
\end{tabular}



\subsubsection{Les objets graphiques d�finis}

L'affichage d'OBM est d�coup� selon les objets graphiques suivants, pour lesquels des styles sont d�finis :\\

\begin{tabular}{|c|c|p{8cm}|}
\hline
\textbf{Objet} & \textbf{Style} & \textbf{Description}\\
\hline
en-t�te HTML & head & En-t�te de d�finition du type de document, titre (<head>)\\
\hline
en-t�te & header & Block d'en-t�te contenant le logo, les infos de connexions\\
\hline
section & section & Block affichant les sections (onglets par exemple)\\
\hline
menu & menu & Block contenant les menus ou modules d'une section\\
\hline
action & action & Block contenant les actions disponibles du module\\
\hline
\multirow{4}{3cm}{derni�re visite} & last & Block contenant les entit�s derni�rement visit�es\\
\cline{2-3}
 & lastTitle & Titre du block \\
\cline{2-3}
 & lastList & �l�ments de la liste des entit�s visit�es \\
\hline
titre & title & Block contenant le titre de la page (ex: Soci�t� : Chercher)\\
\hline
message & msg & Block contenant la zone de message\\
\hline
\multirow{3}{3cm}{recherche} & search & Block contenant le formulaire de recherche\\
\cline{2-3}
 & searchForm & Donn�e d'un champ de recherche \\
\cline{2-3}
 & searchLabel & Label d'un champ de recherche \\
\hline
\multirow{5}{3cm}{d�tail} & detail & Block contenant le d�tail d'une fiche d'un module\\
\cline{2-3}
 & detailHead & En-t�te d'une fiche d�tail \\
\cline{2-3}
 & detailLabel & Label d'une fiche d�tail \\
\cline{2-3}
 & detailText & Donn�e d'une fiche d�tail \\
\cline{2-3}
 & detailForm & Donn�e d'un formulaire d'une fiche d�tail en modification \\
\hline
infos cr�ation & detailInfo & Utilisateurs et dates de cr�ation / modification\\
\hline
\multirow{2}{3cm}{boutons d�tail} & detailButton & Block contenant les boutons associ�s � une fiche d�tail\\
\cline{2-3}
 & detailButtons & Bouton d'une fiche d�tail \\
\hline
\multirow{3}{3cm}{link} & link & Block contenant les liens vers les entit�s associ�es\\
\cline{2-3}
 & linkTitle & Titre du block \\
\cline{2-3}
 & linkList & �l�ments de la liste des entit�s associ�es \\
\hline
\multirow{7}{3cm}{r�sultat de recherche} & result & Block contenant un r�sultat de recherche\\
\cline{2-3}
 & resultIndex & Block contenant l'index d'un r�sultat de recherche \\
\cline{2-3}
 & resultIndexIcon & ic�nes de l'index d'un r�sultat de recherche \\
\cline{2-3}
 & resultHead & Titre des colonnes de r�sultat \\
\cline{2-3}
 & resultText & Cellules du tableau de r�sultats \\
\cline{2-3}
 & resultTextC & Cellules centr�es du tableau de r�sultats \\
\cline{2-3}
 & resultPref & table des pr�f�rences d'affichage d'un r�sultat de recherche \\
\hline
\multirow{4}{3cm}{formulaire d'admin} & admin & Block contenant un formulaire d'administration\\
\cline{2-3}
 & adminHead & titre du formulaire d'administration \\
\cline{2-3}
 & adminLabel & label d'un formulaire d'administration \\
\cline{2-3}
 & adminText & Donn�e d'un formulaire d'administration \\
\hline
fin HTML & end & Cloture de la page\\
\hline
\end{tabular}


\subsubsection{Insertion}


\clearpage
\section{Sp�cifications globales}
% Documentation technique d'OBM : Gestion des entit�s priv�es
% ALIACOM Pierre Baudracco
% $Id$


%%\clearpage
\subsection{Gestion des entit�s priv�es et visibilit�}

La gestion des entit�s priv�es est standardis�e afin de permettre une �volution possible des r�gles de visibilit� des entit�s.


\subsubsection{Sp�cifications de la notion de visibilit�}

Une api tr�s simple (1 fonction d�finie dans global.inc) est disponible :\\

\shadowbox{
\begin{minipage}{13cm}
\begin{verbatim}
is_entity_visible($entity, $obm_q, $uid='''')
\end{verbatim}
\end{minipage}
}
\begin{itemize}
 \item \textbf{\$entity} : entit� � v�rifier.\\
Exemple : ``company'', ``contact''... Attention ce n'est pas toujours le module (ex: parendeal dans module deal)
 \item \textbf{\$obm\_q} : Objet base de donn�es contenant l'entit�\\
Exemple : souvent l'objet associ� � la requete : run\_query\_detail()
 \item \textbf{\$uid} (optionnel) : uid de l'utilisateur pour lequel le droit de visibilit� doit �tre v�rifi�.
Si non donn�, l'utilisateur courant est utilis�.
\end{itemize}

\vspace{0.4cm}

Actuellement une entit� est visible par un utilisateur si l'entit� est publique (visibility == 0) ou que l'utilisateur en est le cr�ateur.

Impl�mentation du test de visibilit� :
\begin{verbatim}
  $field_vis = "${entity}_visibility";
  $field_uc = "${entity}_usercreate";

  if ( ($q->f("$field_vis") == 0)
    || ($q->f("$field_uc") == $uid) ) {
    return true;
  } else {
    return false;
  }
\end{verbatim}


\subsubsection{Gestion de la notion de visibilit� et cons�quences}

Une entit� cr��e en tant que priv�e n'est visible que par son cr�ateur.
Il faut donc veiller � traiter les entit�s priv�s d'un utilisateur lorsque celui-ci doit �tre supprim�.

Ces entit�s peuvent �tre supprim�es ou rendues publiques (la fonction run\_query\_delete\_profile() effectue se travail).


\subsubsection{Modules g�rant la notion de visibilit�}

\begin{tabular}{|p{5cm}|c|}
\hline
\textbf{Module} & \textbf{Depuis version OBM} \\
\hline
contact & 0.4.0 \\
\hline
deal & 0.4.0 \\
\hline
list & 0.8.2 \\
\hline
\end{tabular}
% Documentation technique d'OBM : Appels de modules externes (ext_)
% ALIACOM Pierre Baudracco
% $Id$


\subsection{Appels de modules externes, popups}
\label{extmod}

L'appel de modules externes permet depuis un module d'interroger ou d'obtenir des informations d'un autre module.
Exemple : pour ajouter un utilisateur � un groupe, le module groupe fait appel au module utilisateur.\\

Afin de permettre une impl�mentation unique du module appel� qui sera utilis�e par tous les modules le n�cessitant, l'appel de modules externes est standardis� et r�pond aux objectifs suivants :\\
\begin{itemize}
\item Pas de dupplication de code inutile et lourde � maintenir
\item Le module appel� propose une interface ind�pendante du module appelant
\item Diff�rents types d'informations peuvent �tre demand�es (un Id, une liste d'Id)
\item Le module appelant doit r�cup�rer les informations (par appel � une url ou dans des widgets)
\end{itemize}



\subsubsection{Description des diff�rents appels externes}

Un appel externe est caract�ris� par une action du module appel�.
3 types d'appels sont actuellement d�finis.
Pour chaque appel, divers param�tres sont � fournir.\\

\begin{tabular}{|p{3cm}|p{9.5cm}|}
\hline
\textbf{Action} & \textbf{Commentaire} \\
\hline
ext\_get\_ids & Affiche sans bandeau, en popup, la recherche des entit�s du module, permet d'effectuer des recherches et de renvoyer les entit�s s�lectionn�es par des cases � cocher � une url \\
\hline
ext\_get\_id & Affiche sans bandeau, en popup, la recherche des entit�s du module, permet d'effectuer des recherches et de renvoyer l'entit� s�lectionn�e en cliquant dessus � une url ou dans un widget\\
\hline
ext\_get\_id\_url & Voir difference reelle avec ext\_get\_id ? uniquement param ext\_url ? \\
\hline
\end{tabular}

\subsubsection{Sp�cifications de l'action ext\_get\_ids}

L'exemple donn� est l'ajout d'utilisateurs � un groupe.
Le module appelant est \group, le module appel� est \user.
Depuis un groupe, on s�lection l'ajout d'utilisateurs. Une fen�tre externe popup de recherche d'utilisateurs s'ouvre.\\

L'action ``Ajout d'utilisateurs'' doit donc �tre d�finie dans le module groupe comme un appel externe au module utilisateur.
Des param�tres doivent �tre pass�s.\\

\begin{tabular}{|p{2.5cm}|p{6cm}|p{4.5cm}|}
\hline
\textbf{Param�tre} & \textbf{Commentaire} & \textbf{Exemple} \\
\hline
\multicolumn{3}{|c|}{\textbf{Param�tres de configuration du module appel�}}\\
\hline
action & ext\_get\_ids & \\
\hline
popup & (0 | 1) affichage en popup (sans menu) & 1 \\
\hline
ext\_title & Titre affich� dans la fen�tre externe & Ajouter des utilisateurs\\
\hline
\multicolumn{3}{|c|}{\textbf{Param�tres : retour par url + action}}\\
\hline
ext\_action & Action appel�e par la fen�tre externe & user\_add\\
\hline
ext\_target & Fen�tre cible de retour (target) & Groupe \\
\hline
ext\_url & Url appel�e par la fen�tre externe & \$path/group/group\_index.php \\
\hline
ext\_id & Id de l'entit� r�ceptionnant la r�ponse & Id du groupe origine \\
\hline
\multicolumn{3}{|c|}{\textbf{Retour par widget (select) : fonction javascript fill\_ext\_form() }}\\
\hline
ext\_widget & Indique le widget (select) devant s�lectionner les donn�es & \\
\hline
\multicolumn{3}{|c|}{\textbf{Retour par insertion d'�l�ments (div) : fonction javascript fill\_ext\_element() }}\\
\hline
ext\_element & Indique l'�l�ment (widget) devant s�lectionner les donn�es & \\
\hline
\end{tabular}


\subsubsection{Sp�cifications de l'action ext\_get\_id}

L'exemple donn� est la s�lection d'une soci�t� lors de la cr�ation d'une affaire ou contact.
Le module appelant est \deal, le module appel� est \company.
Depuis le module \deal, on s�lectionne la cr�ation d'une nouvelle affaire. Une fen�tre externe popup de recherche de soci�t� s'ouvre.\\

L'action ``Nouvelle'' doit donc �tre d�finie dans le module \deal comme un appel externe au module \company.
Des param�tres doivent �tre pass�s.\\

\begin{tabular}{|p{2.5cm}|p{6cm}|p{4.5cm}|}
\hline
\textbf{Param�tre} & \textbf{Commentaire} & \textbf{Exemple} \\
\hline
\multicolumn{3}{|c|}{\textbf{Param�tres de configuration du module appel�}}\\
\hline
action & ext\_get\_id & \\
\hline
popup & (0 | 1) affichage en popup (sans menu) & 1 \\
\hline
ext\_title & Titre affich� dans la fen�tre externe & S�lectionner une soci�t�\\
\hline
\multicolumn{3}{|c|}{\textbf{Retour par url : fonction javascript check\_get\_id\_url()}}\\
\hline
ext\_url & Url appel�e par la fen�tre externe (l'url n'est pas ferm�e car elle recevra l'id de l'entit� s�lectionn�e) & \$path/deal/deal\_index.php? action=new\&amp; param\_company= \\
\hline
\multicolumn{3}{|c|}{\textbf{Retour par widget : fonction javascript check\_get\_id()}}\\
\hline
ext\_widget & Indique le widget (hidden) devant recevoir l'Id & f\_contact.company\_new\_id \\
\hline
ext\_widget\_text & Indique le widget (textfield) devant recevoir le label ou nom correspondant � l'Id s�lectionn� & f\_contact.company\_new\_name \\
\hline
\end{tabular}


\subsubsection{D�finition d'un appel � un module externe}

Ce peut �tre une action d'un menu (ex: ajouter utilisateur � groupe) ou un lien direct (ajouter une cat�gorie � une soci�t� depuis le formulaire de mise � jour soci�t�).

\paragraph{Appel par d�finition d'une action}
Exemple : Ajouter un utilisateur au groupe courant.\\

\shadowbox{
\begin{minipage}{14cm}
\begin{verbatim}
// Sel user add : Users selection
  $actions["GROUP"]["sel_user_add"] = array (
    'Name'     => $l_header_add_user,
    'Url'      => "$path/user/user_index.php?action=ext_get_ids&amp;popup=1
&amp;ext_title=".urlencode($l_add_user)."&amp;ext_action=user_add
&amp;ext_url=".urlencode($path."/group/group_index.php")."
&amp;ext_id=".$group["id"]."&amp;ext_target=$l_group",
    'Right'    => $cright_write,
    'Popup'    => 1,
    'Target'   => $l_group,
    'Condition'=> array ('detailconsult','user_add','user_del','group_add'...) 
                                    	  )
\end{verbatim}
\end{minipage}
}
\begin{itemize}
 \item \textbf{Popup} : indique si l'action doit ouvrir une fen�tre externe (1) ou se d�rouler dans la fen�tre courante (d�faut ou 0).\\
 \item \textbf{Target} : uniquement si Popup est positionn�, indique le nom donn� � la fen�tre source avant ouverture du Popup. N�cessaire dans le cas d'utilisation d'ext\_target pour identifier la fen�tre retour. Ces deux valeurs doivent �tre identiques.\\
\end{itemize}

\paragraph{Appel par lien direct}
Exemple : Ajouter des groupes au rendez-vous courant.\\

\shadowbox{
\begin{minipage}{14cm}
\begin{verbatim}
  $url2 = "$path/group/group_index.php?action=ext_get_ids&amp;popup=1
&amp;ext_widget=forms[0].elements[6]
&amp;ext_title=" . urlencode($l_agenda_select_group);
  ...
  $block = ``
     <a href=\"javascript: return false;\" onclick=\"window.open('$url2',
'','height=$popup_height,width=$popup_width,scrollbars=yes'); return false;\">
<img src=\"/images/$set_theme/$ico_add_group\" /></a></td>
  ...``;
\end{verbatim}
\end{minipage}
}


\subsubsection{Modules impl�mentant des appels externes}

\begin{tabular}{|p{5cm}|c|c|}
\hline
\textbf{Module} & \textbf{Appel} & \textbf{Depuis version OBM} \\
\hline
\user & ext\_get\_ids & 0.7\\
\hline
\group & ext\_get\_ids & 0.8 \\
\hline
\resource & ext\_get\_ids & 1.0\\
\hline
\company & ext\_get\_id & 0.7 \\
\hline
\contact & ext\_get\_ids & 0.7 \\
\hline
\deal & ext\_get\_id & 0.8.6 \\
\hline
\List & ext\_get\_id & 0.8 \\
\hline
\publication & ext\_get\_id & 0.8 \\
\hline
\project & ext\_get\_id & 0.8 \\
\hline
\contract & ext\_get\_id & 0.8 \\
\hline
\doc & ext\_get\_ids & 0.8 \\
\hline
\end{tabular}


\subsubsection{Impl�mentation dans un module}

\paragraph{D�finition des actions}
Une action doit �tre d�clar�e pour �tre autoris�e.
Il faut donc d�clarer les actions externes.
Il n'y a pas besoin d'indiquer de titre, l'action ne disposant pas de menu (appel depuis un module externe).\\

\shadowbox{
\begin{minipage}{14cm}
\begin{verbatim}
// Ext get Ids : External User selection
  $actions["USER"]["ext_get_ids"] = array (
    'Right'    => $cright_read,
    'Condition'=> array ('none')
                                    );
\end{verbatim}
\end{minipage}
}



\paragraph{Affichage all�g� (menu, liens)}
L'affichage des fen�tres externes, en popup, poss�de quelques particularit�s afin d'am�liorer l'ergonomie d'utilisation :\\
\begin{itemize}
\item Pas de bandeaux ou menus
\item Pas de liens, donn�es suppl�mentaires (formulaire de s�lection,...)
\end{itemize}
\vspace{0.3cm}

Le bandeau ou menu g�n�ral ne sera affich� que si le param�tre popup n'est pas positionn� (module\_index.php).\\

\shadowbox{
\begin{minipage}{14cm}
\begin{verbatim}
if (! $obm_user["popup"]) {
  $display["header"] = display_menu($module);
}
\end{verbatim}
\end{minipage}
}
\vspace{0.3cm}

Les fonctions d'affichage utilis�es par les appels externes sont les m�mes que les fonctions d'affichage classiques du module (ex: affichage de la liste des utilisateurs apr�s une recherche).\\

Il faut donc que ces fonctions tiennent compte de ces contraintes. L'affichage sans liens est support� par la classe d'affichage OBM\_DISPLAY.

De m�me, le code de g�n�ration du formulaire de r�cup�ration des donn�es (ex: s�lection des utilisateurs) est � cr�er quand n�cessaire.\\

\shadowbox{
\begin{minipage}{14cm}
\begin{verbatim}
  $user_d = new OBM_DISPLAY("DATA", $pref_q, "user");
  if ($popup) {
    $user_d->display_link = false;
    $user_d->data_cb_text = "X";
    $user_d->data_idfield = "userobm_id";
    $user_d->data_cb_name = "data-u-";
    if ($ext_widget != "") {
      $user_d->data_form_head = "
      <form onsubmit=\"fill_ext_form(this); return false;\">";
    } else {
      $user_d->data_form_head = "
      <form target=\"$ext_target\" method=\"post\" action=\"$ext_url\">";
    }
    $user_d->data_form_end = "
      <p class=\"detailButton\">
        <p class=\"detailButtons\">
        <input type=\"submit\" value=\"$l_add\" />
        <input type=\"hidden\" name=\"ext_id\" value=\"$ext_id\" />
        <input type=\"hidden\" name=\"action\" value=\"$ext_action\" />
        </p>
      </p>
      </form>'';

    $display_popup_end = "
      <p>
      <a href=\"\" onclick='window.close();'>$l_close</a>
      </p>";
  }
\end{verbatim}
\end{minipage}
}


\paragraph{Transport des param�tres}

Lorsque une navigation est possible dans une fen�tre externe (ex: cas de recherche puis liste de r�sultat) les param�tres externes doivent �tre transmis afin d'�tre conserv�s.\\

\shadowbox{
\begin{minipage}{14cm}
\begin{verbatim}
  if ($popup) {
    $ext_action = $user["ext_action"];
    $ext_target = $user["ext_target"];
    $ext_url = $user["ext_url"];
    $ext_id = $user["ext_id"];
    $url_ext = "&amp;ext_action=$ext_action&amp;ext_url=$ext_url
&amp;ext_id=$ext_id&amp;ext_target=$ext_target";
  }

  $url = url_prepare("user_index.php?action=search&amp;tf_login=$login
&amp;tf_lastname=$lname&amp;sel_perms=$perms&amp;cba_archive=$archive$url_ext");
\end{verbatim}
\end{minipage}
}

\paragraph{Gestion du retour javascript}

Une fois la ou les entit�s s�lectionn�es dans la fen�tre externe, l'appel retour est r�alis� soit par l'url et l'action indiqu�es, soit � l'aide de fonctions javascript, telles que r�f�renc�es dans les tableaux de sp�cifications des actions ext\_get\_ids et ext\_get\_id.


\paragraph{action ext\_get\_ids, retour par widget}
Cette fonction est d�finie uniquement si le param�tre \$ext\_widget est rempli afin d'�viter des erreurs javascript.

\shadowbox{
\begin{minipage}{14cm}
\begin{verbatim}
if ($ext_widget != "") {
  $extra_js .= "

function fill_ext_form(int_form) {
   size = int_form.length;
   ext_field = window.opener.document.$ext_widget;
   for(i=0; i <size ; i++) {
     if(int_form.elements[i].type == 'checkbox'){
       if(int_form.elements[i].checked == true) {
	 ext_size = ext_field.length;
	 for(j=0; j< ext_size; j++) {
	   if('cb_g' + ext_field.options[j].value == int_form.elements[i].name) {
	     window.opener.document.$ext_widget.options[j].selected =true;
	   }
	 }
       }
     }
   }
}";
}
\end{verbatim}
\end{minipage}
}


\paragraph{action ext\_get\_id, retour par widget}
Cette fonction est d�finie uniquement si les param�tres \$ext\_widget sont remplis afin d'�viter des erreurs javascript.

\shadowbox{
\begin{minipage}{14cm}
\begin{verbatim}
if (($ext_widget != "") || ($ext_widget_text != "")) {
  $extra_js .= "

function check_get_id(valeur,text) {
  if ((valeur < 1) || (valeur == null)) {
    alert (\"$l_j_select_company\");
    return false;
  } else {
    window.opener.document.$ext_widget.value=valeur;
    window.opener.document.$ext_widget_text.value=text;
    window.close();
    return true;
  }
}";
}
\end{verbatim}
\end{minipage}
}


\paragraph{action ext\_get\_id, retour par url}
L'id de l'entit� s�lectionn�e est ajout�e � l'url de retour.

\shadowbox{
\begin{minipage}{14cm}
\begin{verbatim}
function check_get_id_url(p_url, valeur) {
  if ((valeur < 1) || (valeur == null)) {
    alert (\"$l_j_select_company\");
    return false;
  } else {
    new_url = p_url + valeur;
    window.opener.location.href=new_url;
    window.close();
    return true;
  }
}

\end{verbatim}
\end{minipage}
}

% Documentation technique d'OBM : Gestion des droits et profils
% AliaSource Pierre Baudracco
% $Id$


\subsection{Gestion des droits et profils}

La gestion des droits est le m�canisme qui permet le contr�le et l'autorisation des informations accessibles et des actions ex�cut�es par un utilisateur.\\

L'impl�mentation de la gestion des droits a �t� effectu�e avec les objectifs suivants :\\
\begin{itemize}
\item Syst�me peu intrusif dans le code (Eliminer ou �viter au maximum les tests de droits d'acc�s dans le code des modules)
\item Niveau de granularit� � l'action ex�cut�e
\item Faciliter l'�volution (modification des droits, des profils,...)
\item Syst�me l�ger et performant (si possible sans acc�s � la base de donn�es)
\item Syst�me s�r.
\end{itemize}
\vspace{0.3cm}

Les fonctionnalit�es propos�es en standard (sans code sp�cifique dans un module) :\\
\begin{itemize}
\item D�finition des sections et modules accessibles par profil.
\item Autorisation d'ex�cution au niveau de l'action d'un module par profil
\item Tout acc�s ou action non d�fini est interdit
\item Possibilit� de tests plus sp�cifiques dans un module (ex: champ affich� selon droit pr�cis) par utilisation de l'API de droits dans le module.
\item API simple de tests de droits.
\end{itemize}


\subsubsection{Principe des droits}

Chaque action (consultation, cr�ation, modification,...) de tous les modules d'\obm n�cessite une autorisation pour �tre ex�cut�e.
La d�finition des actions des modules pr�cise donc un ``droit �l�mentaire'' (ex: droit de lecture) n�cessaire pour son ex�cution.\\

Les droits d'un utilisateur sur un module (ou permission sur un module) sont constitu�s de l'ensemble des droits �l�mentaires qu'il poss�de sur ce module (ex: lecture + �criture sur le module \company).
Les droits �l�mentaires �tant d�finis comme des champs de bits, la permission sur un module est une combinaison logique des droits �l�mentaires.\\

Les permissions ne sont pas associ�es directement aux utilisateurs, mais � des profils utilisateurs. Chaque utilisateur d'\obm est associ� � un profil.


\subsubsection{Droits �l�mentaires}

Droits �l�mentaires d�finis par \obm :\\

\begin{tabular}{|p{3cm}|p{2cm}|p{2cm}|p{5cm}|}
\hline
\textbf{Droit} & \textbf{bit} & \textbf{Valeur} & \textbf{Description} \\
\hline
\$cright\_read & 0 & 1 & Lecture donn�es simples \\ 
\hline
\$cright\_own & 1 & 2 & Ecriture donn�es personnelles \\ 
\hline
\$cright\_write & 2 & 4 & Ecriture donn�es simples \\ 
\hline
\$cright\_read\_admin & 3 & 8 & Lecture donn�es admin \\ 
\hline
\$cright\_write\_admin & 4 & 16 & Ecriture donn�es admin \\ 
\hline
\end{tabular}

\subsubsection{Permissions : combinaisons de droits �l�mentaires}

Toute combinaison de droits �l�mentaires est une permission.
\obm par commodit� d�finit les permissions des profils par d�faut :\\

\begin{tabular}{|p{3cm}|p{2.5cm}|p{2.2cm}|p{5cm}|}
\hline
\textbf{Permission} & \textbf{Droits (bits)} & \textbf{Valeur (hex)} & \textbf{Description} \\
\hline
\$perm\_reader & 0 & 01 & Voir donn�es simples \\ 
\hline
\$perm\_user & 0+1 & 03 & Voir donn�es simples + �criture donn�es personnelles \\ 
\hline
\$perm\_editor & 0+2 & 05 & Lecture + �criture donn�es simples \\ 
\hline
\$perm\_admin & 0+2+3+4 & 1D & Tout faire \\ 
\hline
\end{tabular}


\subsubsection{Droits sur un module}

Chaque action d'un module pr�cise un droit �l�mentaire d'ex�cution (voir \ref{actions}).
Exemple pour l'action search du module \user qui n�cessite le droit \$cright\_read :\\

\shadowbox{
\begin{minipage}{13cm}
\begin{verbatim}
// Search
  $actions["user"]["search"] = array (
    'Url'      => "$path/user/user_index.php?action=search",
    'Right'    => $cright_read,
    'Condition'=> array ('None') 
                                  );
\end{verbatim}
\end{minipage}
}

Si le droit \$cright\_read est inclus dans la permission de l'utilisateur (de son profil) sur ce module, il pourra ex�cuter l'action. 


\subsubsection{D�finition des profils}

Un profil est un ensemble de :
\begin{itemize}
\item Permissions sur des sections et des modules.
\item Propri�t�s
\end{itemize}

\begin{tabular}{|p{5cm}|p{8cm}|}
\hline
\textbf{Caract�ristique} & \textbf{Contenu} \\
\hline
section (permission) & Tableau des sections. Une valeur � 1 indique onglet affich� pour le profil (l'entr�e 'default' d�finit la valeur pour les sections non list�es \\ 
\hline
module (permission) & Tableau des modules. D�finition des droits � chaque module. L'entr�e 'default' d�finit la valeur pour les modules non list�s \\ 
\hline
level & Niveau du profil. Un administrateur ou utilisateur ne peut pas cr�er / modifier / supprimer un utilisateur de privil�ge plus �lev� (niveau 0 = privil�ge maximum).\\ 
\hline
level\_managepeers & Capacit� du profil � g�rer des utilisateurs de m�me niveau de profil. Si cet indicateur est � true (ou 1), il est possible de g�rer des utilisateurs de m�me niveau, sinon non \\ 
\hline
properties[admin\_realm] & Granularit� d'administration (et de possibilit� de mise � jour) pour un administrateur. Valeurs possibles : 'user', 'delegation', 'domain'.\\
\hline
\end{tabular}
\vspace{0.3cm}

Une permission par d�faut peut �tre attribu�e � un profil pour les sections et les modules.
Exemple de d�finition de profil :\\

\shadowbox{
\begin{minipage}{13cm}
\begin{verbatim}
$profiles['editor] = array (
  'section' => array (
    'default' => 0,
    'com' => 1,
    'prod' => 1,
    'user' => 1),
  'module' => array (
    'default' => $perm_editor),
  'level' => 3
);
$profiles['admin'] = array (
  'section' => array (
    'default' => 1,
    'com' => 1,
    'prod' => 1,
    'user' => 1),
  'module' => array (
    'default' => $perm_admin),
  'properties' => array (
    'admin_realm' => array ('user', 'delegation', 'domain')),
  'level' => 1,
  'level_managepeers' => 1
);
\end{verbatim}
\end{minipage}
}

Un utilisateur du profil \textbf{editor} pourra acc�der aux sections ``com'', ``prod'' et ``user'' (et uniquement celles-ci) et aura les permissions ``\$perm\_editor'' sur tous les modules.\\

\paragraph{Personnalisation des profils}

Les profils par d�faut peuvent �tre modifi�s ou supprim�s et de nouveaux profils peuvent �tre d�finis dans le fichier de configuration.


\subsubsection{Impl�mentation interne d'\obm}

\paragraph{D�finition du droit d'acc�s � une section}

La d�finition du droit d'acc�s n�cessaire pour acc�der (et voir) une section est pr�cis�e dans la d�finition des sections dans le fichier de configuration (voir section \ref{cgp_show_section}).\\

La configuration est interpr�t�e dans le \fichier{obminclude/global\_pref.inc} et renseigne le hashage \variable{\$sections} utilis� par \obm.


\paragraph{D�finition du droit d'acc�s � un module}

La d�finition du droit d'acc�s � un module est pr�cis�e dans la d�finition des modules dans \fichier{obminclude/global\_pref.inc}.
Un droit d'acc�s unitaire est requis (l'utilisateur doit avoir le droit de lecture \variable{\$cright\_read} sur le module \settings pour y acc�der dans l'exemple suivant).\\

\shadowbox{
\begin{minipage}{13cm}
\begin{verbatim}
  if ($cgp_show["module"]["contact"]) {
    $modules["contact"] = array(
                               'Name'=> $l_module_contact,
                               'Ico' => "$ico_contact",
		               'Url' => "$path/contact/contact_index.php",
			       'Right'=> $cright_read);
  }
\end{verbatim}
\end{minipage}
}


\subsubsection{Limitations actuelles du syst�me de droits}

\begin{itemize}
\item Les droits sont donn�s sur les actions sans distinction de particularit� des donn�es trait�es. Il n'est pas possible par exemple de donner le droit de modification de groupes en fonction des caract�ristiques du groupe � modifier (ex: pas les groupes syst�me,...). Le droit donn� est : modification de groupe OUI ou Non. Ces cas sont donc v�rifi�s au moment de l'ex�cution de l'action.

\item L'ajout de profil n'est pas automatiquement pris en compte dans le module \user (pour permettre l'affectation graphique du profil aux utilisateurs, l'apparition dans les s�lections...). Cela n�cessite du code dans le module.
\end{itemize}
\section{Calendar}

Le \calendar \obm  est un agenda partag� permettant d'inserer, modifier, consulter ou supprimer des �venements pour un ou plusieurs utilisateurs simultan�.

\clearpage

\subsection{Organisation de la base de donn�es}

Le \calendar utilise trois tables :
\begin{itemize}
 \item CalendarCategory
 \item CalendarEvent
 \item CalendarSegment
\end{itemize}

\subsubsection{CalendarCategory}
Cette table est utilis�e pour sotcker les cat�gories des �v�nements.
\begin{description}
 \item[calendarcategory\_id] int(8) : C'est la cl� primaire de la table.
 \item[calendarcategory\_timeupdate] timestamp(14) : Donne au format Unix la date de derni�re mise � jour du tuple.
 \item[calendarcategory\_timecreate] timestamp(14) : Donne au format Unix la date de cr�ation du tuple.
 \item[calendarcategory\_userupdate] int(8) : Id de l'utilisateur qui � op�r� la derni�re mise � jour.
 \item[calendarcategory\_usercreate] int(8) : Id de l'utilisateur qui � cr�� le tuple.
 \item[calendarcategory\_label] varchar(128) : Label de la cat�gorie. 
\end{description}

\subsubsection{CalendarEvent}
Cette table stocke toute la description d'un �venement ainsi que ces caract�ristiques.
\begin{description}
 \item[calendarevent\_id] int(8) : C'est la cl� primaire de la table.
 \item[calendarevent\_timeupdate] timestamp(14) : Donne au format Unix la date de derni�re mise � jour du tuple.
 \item[calendarevent\_timecreate] timestamp(14) : Donne au format Unix la date de cr�ation du tuple.
 \item[calendarevent\_userupdate] int(8) : Id de l'utilisateur qui � op�r� la derni�re mise � jour.
 \item[calendarevent\_usercreate] int(8) : Id de l'utilisateur qui � cr�� le tuple.
 \item[calendarevent\_title] varchar(255) : Titre de l'�venement(description courte).
 \item[calendarevent\_description] text  : Description de l'�v�nement
 \item[calendarevent\_category\_id] int(8) : Cl� �trang�re pointant vers CalendarCategory indiquant la cat�gorie de l'�venement.
 \item[calendarevent\_priority] int(2) : Priotit� de l'�v�nement. Les differentes valeurs possibles sont <1> pour Basse, <2> pour Normal, <3> pour Haute.
 \item[calendarevent\_privacy] int(2) : Indique si l'�v�nement est un �v�nement priv� ou non. Un �v�nement priv� ne peux pas etre vue en vue multi-utilisateurs.
 \item[calendarevent\_length] varchar(14)  : Dur�e de l'�v�nement en seconde.
 \item[calendarevent\_repeatkind] varchar(20) : Type de r�p�tition de l'�venement. Les differentes valeurs sont <none> pour indiquer l'absence de r�p�tition, <weekly> pour indiquer une r�p�tion qui aurait lieu toute les semaines, <monthlybydate> pour une r�p�tion qui aurait lieu tout les mois a une date donn� (le 4 de chaque mois par exemple), <monthlybyday> pour une r�p�tition qui aurait lieu tout les mois � un jour donn�(exemple tout les seconds lundi du mois) et enfin <yearly> pour une r�p�tition annuelle.
 \item[calendarevent\_repeatdays] varchar(7) : Pour le cas d'une r�p�tition h�bdomadaire, ce champs determine les jours de la semaine auxquels ont lieu l'�venement. La valeur de ce champ est compos�e de 7 valeurs binaire(une pour chaque jour de la semaine). Un 1 signifira que le jour en question l'�venement aura lieu, le 0 que non (exemple : si la semaine commence Lundi : 1100100 signifiera que l'�venement a lieu les lundi, mardi, et vendredi).
 \item calendarevent\_endrepeat varchar(12) : Date de fin de r�p�tition.
\end{description}

\subsubsection{CalendarEvent}
Cette table stocke toutes les occurrences d'un �v�nement.
\begin{description}
 \item[calendarsegment\_eventid] int(8) : C'est la cl� primaire de la table.
 \item[calendarsegment\_customerid] int(8) : Cl� �trang�re pointant vers UserObm ou vers la table de groupe(non impl�ment� actuellement) indiquant l'utilisateur ou le groupe concern� par cette occurrence de l'�venement.
 \item[calendarsegment\_date] varchar(12) : Date de l'occurence.
 \item[calendarsegment\_flag] varchar(5) : Ce champ permet de determiner si la date est la date de debut ou de fin de l'occurence. Ainsi chaque occurence aura deux tuples dans cette table, un possedant le flag <Begin>(d�but), l'autre le flag <End>(fin).
 \item[calendarsegment\_type] varchar(5) : Determine si l'occurence affect un groupe (<group>) ou un utilisateur(<user>).
 \item[calendarsegment\_state] char(1) : Etat de l'occurence, ce champ peux prendre trois valeurs, <R> pour Refus�, <W> pour en attente, et <A> pour accept�.
\end{description} 

\clearpage


\subsection{Actions, droits, et fonctions}
Dans cette section seront list�es et d�crites les actions du module, ainsi que les droits qui leurs sont appliqu�s et l'intitul� des fonctions execut�es. 

\subsubsection{Actions}

Voici la liste des action du modul \calendar, ainsi qu'une description sommaire de chacune d'entre elles.\\
\begin{tabular}{|c|p{9cm}|}
 \hline
 \textbf{Intitul�} & \textbf{Description} \\
 \hline
 \hline
 <index> & Action par d�faut, elle affiche suivant le cas le formulaire pr�sentant les �venements en attente ou le planning de la semaine en cours pour l'utilisateur courant. \\ 
 \hline
 <decision> & Cette action modify l'�tat de l'occurence d'un �venement en fonction du choix de l'utilisateur sur le formulaire pr�sentant les �venements en attente. \\ 
 \hline
  <view\_year> & Insere un �v�nement suite � l'affichage du formulaire du gestion de conflit. \\ 
 \hline
  <view\_month> & Affiche le planning au format du mois. \\
 \hline
  <view\_week> & Affiche le planning au format de la semaine. \\
 \hline
  <view\_day> & Affiche le planning au format du jour. \\ 
 \hline
 <new> & Cette action affiche le formulaire d'ajout d'un nouvel �venement. \\
 \hline
 <insert> & Cette action suivant les cas ins�re un �v�nement saisie par l'utilisateur, annule l'insertion et r�-affiche le formulaire d'ajout d'�v�nement en cas d'erreur, ou affiche le forumaire de gestion de conflit.  \\ 
 \hline
 <insert\_conflict> & Insere un �v�nement suite � l'affichage du formulaire du gestion de conflit. \\ 
 \hline
  <detail\_consult> & Affiche les d�tails d'un �v�nement ainsi que le forumulaire de changement de d�cision. \\
 \hline
  <check\_delete> & Affiche le formulaire de confiramtion de suppression. \\
 \hline
  <delete> & Supprime un �venement ainsi que toutes ses occurrences. \\
 \hline
  <detailupdate> & Affiche le formulaire de mise � jour des d�tails d'un �venements. \\
 \hline
  <update> & Suivant les cas met � jour un �v�nement et eventuellement ses occurences, annule la mise � jour et r�-affiche le formulaire de mise � jour d'�v�nement en cas d'erreur. \\
 \hline
  <update\_decision> & Mise � jour de l'etat d'une occurence suite � un changement de l'utilisateur sur le forumulaire de changement de d�cision. \\
 \hline
\end{tabular}

\clearpage

\subsubsection{Droits}

Ci-dessous le tableau pr�sentant les droits necessaires � l'execution de chaque action ainsi que l'intitul� de ce droit. \\
\begin{tabular}{|c|c|l|}
 \hline
 \textbf{Action} & \textbf{Code du droit} & \textbf{Intitul� du droit}\\
 \hline
 \hline
  <index> & agenda\_read & Droits d'Utilisateur en lecture. \\ 
 \hline
 <decision> & agenda\_write & Droits d'Utilisateur en ecriture. \\ 
 \hline
  <view\_year> & agenda\_read & Droits d'Utilisateur en lecture. \\ 
 \hline
  <view\_month> & agenda\_read & Droits d'Utilisateur en lecture. \\
 \hline
  <view\_week> & agenda\_read & Droits d'Utilisateur en lecture. \\
 \hline
 <new> & agenda\_write & Droits d'Utilisateur en ecriture. \\
 \hline
 <insert> & agenda\_write & Droits d'Utilisateur en ecriture. \\ 
 \hline
 <insert\_conflict> & agenda\_write & Droits d'Utilisateur en ecriture. \\ 
 \hline
  <view\_day> & agenda\_read & Droits d'Utilisateur en lecture. \\ 
 \hline
  <detail\_consult> & agenda\_read & Droits d'Utilisateur en lecture. \\
 \hline
  <check\_delete> & agenda\_admin\_write & Droits d'Administrateur en ecriture. \\
 \hline
  <delete> & agenda\_admin\_write & Droits d'Administrateur en ecriture. \\
 \hline
  <detailupdate> &  agenda\_write & Droits d'Utilisateur en ecriture. \\
 \hline
  <update> & agenda\_write & Droits d'Utilisateur en ecriture. \\
 \hline
  <update\_decision> & agenda\_write & Droits d'Utilisateur en ecriture. \\
 \hline
\end{tabular}
\subsubsection{Fonctions}
 Ci-dessous le tableau pr�sentant la ou les fonctions appel�es pour chacune des actions. \\
\begin{tabular}{|c|c|}
 \hline
 \textbf{Action} & \textbf{Fonction(s)}\\
 \hline
 \hline
  <index> & run\_query\_waiting\_events() \\
          & display\_warn\_msg() \\  
          & html\_waiting\_events() \\ 
          & ou \\ 
 	  & run\_query\_waiting\_events() \\
	  & run\_query\_week\_event\_list() \\
	  & run\_query\_get\_user\_name() \\
	  & run\_query\_userobm() \\
	  & dis\_week\_planning() \\
	  & localizeDate() \\
	  & store\_users() \\
	  & store\_events() \\
	  & html\_planning\_bar() \\
 \hline
  <decision> & run\_query\_insert\_decision() \\ 
             & display\_ok\_msg \\
   	     & run\_query\_waiting\_events() \\
  	     & run\_query\_week\_event\_list() \\
	     & run\_query\_get\_user\_name() \\
	     & run\_query\_userobm() \\
	     & dis\_week\_planning() \\
	     & localizeDate() \\
	     & store\_users() \\
	     & store\_events() \\
	     & html\_planning\_bar() \\
	     & ou \\
	     & display\_warn\_msg() \\
	     & html\_dis\_conflict() \\
 \hline
  <view\_year> & run\_query\_year\_event\_list() \\
 	       & run\_query\_get\_user\_name() \\
	       & run\_query\_userobm() \\
	       & dis\_year\_planning() \\
	       & localizeDate() \\
	       & store\_users() \\
	       & store\_daily\_events() \\
	       & html\_planning\_bar() \\
 \hline
  <view\_month> & run\_query\_month\_event\_list() \\
 	        & run\_query\_get\_user\_name() \\
	        & run\_query\_userobm() \\
	        & dis\_month\_planning() \\
	        & localizeDate() \\
	        & store\_users() \\
	        & store\_daily\_events() \\
	        & html\_planning\_bar() \\
 \hline
 \end{tabular}

\begin{tabular}{|c|c|p{3cm}|}
 \hline
 \textbf{Action} & \textbf{Fonction(s)} \\
 \hline
 \hline
  <view\_week> & run\_query\_week\_event\_list() \\
	       & run\_query\_get\_user\_name() \\
	       & run\_query\_userobm() \\
	       & dis\_week\_planning() \\
	       & localizeDate() \\
	       & store\_users() \\
	       & store\_events() \\
	       & html\_planning\_bar() \\   	    
 \hline
  <view\_day> & run\_query\_day\_event\_list() \\ 
	      & run\_query\_get\_user\_name() \\
	      & run\_query\_userobm() \\
	      & dis\_day\_planning() \\
	      & localizeDate() \\
	      & store\_users() \\
	      & store\_events() \\
	      & html\_planning\_bar() \\  
 \hline
 <insert> & check\_data\_form() \\ 
          & run\_query\_week\_event\_list() \\
          & run\_query\_get\_user\_name() \\
          & run\_query\_userobm() \\
          & display\_ok\_msg() \\
          & dis\_week\_planning() \\
	  & localizeDate() \\
	  & store\_users() \\
	  & store\_events() \\
	  & html\_planning\_bar() \\
	  & ou \\	  
	  & check\_data\_form() \\ 
          & display\_warn\_msg() \\
          & html\_dis\_conflict() \\
	  & ou \\	
	  & check\_data\_form() \\ 
          & display\_err\_msg() \\
	  & run\_query\_userobm() \\
          & run\_query\_get\_eventcategories() \\
          & run\_query\_userobm() \\
          & dis\_event\_form() \\
	  & ou \\	
	  & check\_data\_form() \\ 
          & display\_warn\_msg() \\
	  & run\_query\_userobm() \\
          & run\_query\_get\_eventcategories() \\
          & run\_query\_userobm() \\
          & dis\_event\_form() \\

 \hline
\end{tabular}

\begin{tabular}{|c|c|p{3cm}|}
 \hline
 \textbf{Action} & \textbf{Fonction(s)} \\
 \hline
 \hline
 <insert\_conflict> & run\_query\_manage\_conflict()\\ 
                    & run\_query\_week\_event\_list() \\
                    & run\_query\_get\_user\_name() \\
                    & run\_query\_userobm() \\
                    & display\_ok\_msg() \\
                    & dis\_week\_planning() \\
            	    & localizeDate() \\
	            & store\_users() \\
	            & store\_events() \\
	            & html\_planning\_bar() \\
 \hline 
 <new> & run\_query\_userobm() \\
       & run\_query\_get\_eventcategories() \\
       & run\_query\_userobm() \\
       & dis\_event\_form() \\
 \hline
  <detail\_consult> & run\_query\_detail() \\
  		    & run\_query\_event\_customers() \\
		    & display\_record\_info() \\
		    & html\_calendar\_consult() \\
 \hline
  <check\_delete> & html\_dis\_delete() \\
 \hline
  <delete>  & run\_query\_delete() \\
  	    & run\_query\_waiting\_events() \\
	    & run\_query\_week\_event\_list() \\
	    & run\_query\_get\_user\_name() \\
	    & run\_query\_userobm() \\
	    & dis\_week\_planning() \\
	    & localizeDate() \\
	    & store\_users() \\
	    & store\_events() \\
	    & html\_planning\_bar() \\
 \hline
  <detailupdate> & run\_query\_userobm() \\
  		 & run\_query\_get\_eventcategories() \\
  		 & run\_query\_detail() \\
  		 & run\_query\_event\_customers\_array() \\
  		 & display\_record\_info() \\
  		 & dis\_event\_form() \\
 \hline
\end{tabular}

\begin{tabular}{|c|c|p{3cm}|}
 \hline
 \textbf{Action} & \textbf{Fonction(s)} \\
 \hline
 \hline
  <update> & check\_data\_form() \\
           & run\_query\_modify\_event() \\
           & display\_ok\_msg \\
           & run\_query\_week\_event\_list() \\
           & run\_query\_get\_user\_name() \\
           & run\_query\_userobm() \\
           & display\_ok\_msg() \\
           & dis\_week\_planning() \\
           & localizeDate() \\
	   & store\_users() \\
	   & store\_events() \\
	   & html\_planning\_bar() \\
           & ou \\
           & check\_data\_form() \\
           & run\_query\_modify\_event() \\
           & display\_warn\_msg( \\
           & html\_dis\_conflict() \\
           & ou \\
	   & check\_data\_form() \\
	   & run\_query\_modify\_event() \\
           & display\_error\_msg() \\
           & html\_dis\_conflict() \\
           & run\_query\_week\_event\_list() \\
           & run\_query\_get\_user\_name() \\
           & run\_query\_userobm() \\
           & display\_ok\_msg() \\
           & dis\_week\_planning() \\
           & localizeDate() \\
	   & store\_users() \\
	   & store\_events() \\
	   & html\_planning\_bar() \\	   
           & ou \\
	   & check\_data\_form() \\
           & display\_warn\_msg() \\
           & run\_query\_userobm() \\
           & run\_query\_get\_eventcategories() \\
           & dis\_event\_form() \\
 \hline
\end{tabular}

\begin{tabular}{|c|c|p{3cm}|}
 \hline
 \textbf{Action} & \textbf{Fonction(s)} \\
 \hline
 \hline 
  <update\_decision> & run\_query\_change\_decision() \\
                     & display\_ok\_msg \\
                     & run\_query\_week\_event\_list() \\
                     & run\_query\_get\_user\_name() \\
                     & run\_query\_userobm() \\
                     & display\_ok\_msg() \\
                     & dis\_week\_planning() \\
            	     & localizeDate() \\
	             & store\_users() \\
	             & store\_events() \\
	             & html\_planning\_bar() \\
		     & ou \\
		     & run\_query\_change\_decision() \\
		     & display\_warn\_msg() \\
		     & html\_dis\_conflict() \\
 \hline
\end{tabular}
\subsection{Fonctionalit�}
 Dans cette partie seront abord�es les fonctionnalit�s du module \calendar. Ces fonctionnalit�s seront �tudi�s d'un point de vue technique et non utilisateur. Il va donc plus etre sujet du fonctionnement que de l'utilit� et de l'utilisation du module.

\subsubsection{Affichage}
 Nous passerons assez rapidement sur l'affichage des details d'un �venement qui est une fonctionnalit� assez basique(recherche des �l�ments en base, puis affichage de ces derniers de mani�re structur�).
 Nous nous interresseront plus au differentes vues du planning.

 Nous alons dans un premier temps traiter les vues quotidiennes et hebdomadaire.
 Ces deux vues seront trait�s ensemble car leurs fonctionnements est assez similaires. Nous �piloguerons sur les vues mensuelles et annuelles qui divergent l�g�rement ces deux vues.
 Les etapes de la production de cette affichage sont les suivantes : 
 \begin{itemize}
  \item{R�cup�ration des informations en base de donn�es}
  \item{Traitement des donn�es et construction du mod�le}
  \item{Traitement du mod�le et construction de la vue}
  \item{Affichage de la vue}
 \end{itemize}
 Donc pour la r�cup�ration des donn�es en base, trois fonctions sont utilis�es � savoir :
 \begin{description}
  \item{\fonction{run\_query\_(day ou week)\_event\_list}} : Cette fonction permet de r�cup�rer les donn�es propres aux �venements(description, titre... mais aussi les dates) ayant lieu le jour ou la semaine pass� en param�tre et concernant les utilisateurs pass�s en param�tre.
  \item{\fonction{run\_query\_get\_user\_name}} : Cette fonction r�cup�re en base de donn�es les noms des utilisateurs dont les id sont pass� en param�tre.
  \item{\fonction{run\_query\_userobm}} : Cette fonction r�cup�re l'ensemble des utilisateurs d'\obm(utilis� pour la constitution de la planning bar).
 \end{description}
 Une fois les donn�es r�cup�r�es, elles sont donc trait�s afin de constituer le mod�le.
 En premier lieu les informations sur les utilisateurs r�cup�r�s par la fonction \fonction{run\_query\_get\_user\_name} sont stock�es dans un tableau,\variable{calendar\_user}, ou chaque utilisateur est associ� � une couleur et a une classe \css et une � une image, et ce dans la fonction \fonction{store\_users}.
 Ensuite est constitu� le mod�le. Celui-ci est construit dans la fonction \fonction{store\_events}. Cette fonction renvoit deux tableaux, le premier,\variable{event\_data}, index� sur les id des evenements, stocke l'ensemble des donn�es relative a un evenement ainsi que le nombre total de ces occurences sur le jour ou la semaine. Le second,\variable{current\_events}, est un tableau ind�x� sur le temps, contient pour chaque unit� de temps(l'unit� de temps est definit dans le fichier \fichier{global.inc}) la ou les id des evenements ayant lieu � ce moment l�.
 \\\textbf{Exemple:}\\
 Prenons la vue quotidienne pour les utilisateurs U1 et U2.
 Voici la liste des �venements de ces deux utilisateur pour le jour J1 :\\
 \begin{tabular}{|c|c|c|c|}
  \hline
   \textbf{Evenements} & \textbf{Heures} & \textbf{U1} & \textbf{U2} \\
  \hline
  \hline
   E1 & H1->H3 & X & \\
   E2 & H2->H4 & & X \\
   E3 & H6->H7 & X & X \\
   E4 & H3->H4 & X &  \\
  \hline
 \end{tabular}  
\\Donc le tableau de donn�es sera de la forme :\\
 \begin{tabular}{|c|c|c|}
  \hline
   \textbf{Id} & \textbf{Information} & \textbf{Occurence} \\
  \hline
  \hline
   E1 & I1 & 1 \\
  \hline
   E2 & I2 & 1 \\
  \hline
   E3 & I3 & 2 \\
  \hline
   E4 & I4 & 1 \\
  \hline
 \end{tabular} 
\\L'�venement E3 � 2 occurences, car il y en a une pour l'utilisateur U1 et une pour l'utilisateur U2.
Le tableau contenant le mod�le proprement dit sera de la forme suivante :\\
 \begin{tabular}{|c|c|c|}
  \hline
   \textbf{Heures} & \textbf{U1} & \textbf{U2} \\
  \hline
  \hline
   H1 & E1 & \\
  \hline
   H2 & E1 & E2 \\
  \hline
   H3 & E1,E4 & E2 \\
  \hline
   H4 & E4 & E2 \\
  \hline
   H6 & E3 & E3 \\
  \hline
   H7 & E3 & E3 \\
  \hline
 \end{tabular} 
 \\\textbf{Fin exemple}\\
 Une fois le mod�le constitu� il est trait� dans la fonction \fonction{dis\_planning\_(day ou week)}.
 Pour ce faire, on entrera dans une structure de controle qui bouclera sur les unit� de temps. A chaque it�ration on v�rifie pour 
 chaque utilisateur si il y a un �vemenement � l'unit� de temps en cours dans le tableau \variable{current\_events}.
 Si oui alors :
 \begin{itemize}
  \item{On test si les informations de cet �venement on d�j� �tait affich� pour cet utilisateur. Si oui on continue l'iteration sur les unit�s de temps}
  \itemize{Si non, on ins�re dans la vue les informations propre � l'evenement d�t�ct� et dans \variable{event\_data} on diminue le nombre d'occurence de l'evenement de 1.}
  \item{On determine la taille de la case contenant les informations en comptant le nombre de boucle qui reponde � la condition : intersection des �venement � l'heure T et des �venements a l'heure T+1 n'est pas nulle. Cette condition permet en fait de determin� la plage horraire sur laquelle se deroule l'evenement et eventuellement les �venements qui sont en conflits avec celui ci.}
  \item{On reprend l'iteration sur les unit�s de temps}
 \end{itemize}
Enfin on affiche la vue.
 \\\textbf{Exemple:}\\
Reprenons l'exemple ci-dessus :
On boucle donc sur les unit�s de temps :


H1 : Un �venement pour l'utilisateur U1 dans le tableau.\\
Dans \variable{event\_data} le nombre d'occurence de E1 > 0.\\
On ins�re donc dans la vue les information I1 et on decrement le nombre d'occurence de E1.\\
On boucle sur la \variable{current\_events}:\\
\begin{itemize}
 \item{H2 : Dans \variable{event\_data} le nombre d'occurence de E1 = 0.Intersection de \variable{current\_events[H1][U1]} et \variable{current\_events[H2][U1]} = E1. On continue donc a boucler.}
 \item{H3 : Dans \variable{event\_data} le nombre d'occurence de E1=0 et E4 > 0.On ins�re donc dans la vue les information I4 et on decrement le nombre d'occurence de E4. Intersection de \variable{current\_events[H3][U1]} et \variable{current\_events[H4][U1]} = E4. On continue donc a boucler.}
 \item{H4 : Dans \variable{event\_data} le nombre d'occurence de E4=0.Intersection de \variable{current\_events[H4][U1]} et \variable{current\_events[H5][U1]} = NULL. On arrete donc de boucler. La taille de la case sera donc de 4.}
On reprend l'iteration sur les heures :\\
\end{itemize}
H2: Un �venement pour l'utilisateur U1 et U2 dans le tableau.\\
Dans \variable{event\_data} le nombre d'occurence de E1 = 0, donc on ne s'en occupe pas, par contre pour E2 il est sup�rieur � 0.\\
On ins�re donc dans la vue les information I2 et on decrement le nombre d'occurence de E2.\\
On boucle sur la \variable{current\_events}:\\
\begin{itemize}
 \item{H3 : Dans \variable{event\_data} le nombre d'occurence de E2 = 0.Intersection de \variable{current\_events[H3][U2]} et \variable{current\_events[H4][U2]} = E2. On continue donc a boucler.}
 \item{H4 : Dans \variable{event\_data} le nombre d'occurence de E2 = 0.Intersection de \variable{current\_events[H4][U2]} et \variable{current\_events[H5][U2]} = E2. On arrete donc de boucler. La taille de la case sera donc de 3.}
\end{itemize} 
H3 : Deux �venements pour l'utilisateur U1 et un pour l'utilisateur U2 dans le tableau.\\
Dans \variable{event\_data} le nombre d'occurence de E1, E4 et E2 sont �gaux � 0, donc on ne s'en occupe pas.\\
H4 : Un �venement pour l'utilisateur U1 et U2 dans le tableau.\\
Dans \variable{event\_data} le nombre d'occurence de E4 et E2 sont �gaux � 0, donc on ne s'en occupe pas.\\
H5 : Pas d'�venements.
H6 : Un �venement pour l'utilisateur U1 et U2 dans le tableau.\\
Dans \variable{event\_data} le nombre d'occurence de E3 > 0, on ins�re donc dans la vue les information I3 pour l'utilisateur U1 et on decrement le nombre d'occurence de E3. Le nombre d'occurence de E3 > 0 On ins�re donc dans la vue les information I3 pour l'utilisateur U2 et on decrement le nombre d'occurence de E3.\\
On boucle sur la \variable{current\_events}:\\
\begin{itemize}
 \item{H7 : Dans \variable{event\_data} le nombre d'occurence de E3 = 0.Intersection de \variable{current\_events[H6][U2]} et \variable{current\_events[H7][U2]} = E2. On continue donc a boucler.}
 \item{On arrive � la fin du tableau \variable{current\_events} on arrete donc la boucle.La taille de la case sera donc de 2}
\end{itemize} 
On � termin� la boucle sur les unit�s de temps. On affiche donc la vue qui ressemblera a cela :\\
\begin{center}
 \begin{tabular}{|c|c|c|}
  \hline
   \textbf{Heure} & \textbf{U1} & \textbf{U2} \\
  \hline
  \hline
  H1 & I1 & \\
  \cline{1-1}
  \cline{3-3}
  H2 & I4 & I2 \\
  \cline{1-1} 
  H3 &  &  \\
  \cline{1-1} 
  H4 &  &  \\
  \hline 
  H5 &  &  \\
  \hline
  H6 & I3 & I3 \\
  \cline{1-1} 
  H7 & I3 & I3 \\
  \hline
 \end{tabular}  
\end{center}

\textbf{Fin exemple}\\
En ce qui concerne les vues annuelles et mensuelles, la difference se situe au niveau de la constitution du tableau \variable{current\_events}. Ce tableau est index� sur les jours et non sur les unit�s de temps ( la fonction utilis� pour ce faire est la fonction \fonction{store\_daily\_events}). Ceci est du au fait que la vue pour ces affichages l� est faites par jour et non par unit� de temps. A part cela le fonctionnement est strictement le m�me.

\subsubsection{Insertion}


% Documentation technique d'OBM : module Time manager
% ALIACOM Bastien Continsouzas
% $Id

\section{Time}

Le \timemanager \obm  est un module permettant.

\clearpage

\subsection{Organisation de la base de donn�es}

Le \timemanager utilise une table :
\begin{itemize}
 \item{TimeTask}
\end{itemize}

\subsubsection{TimeTask}
Cette table stocke toutes les informations concernant une t�che qui a �t� saisie par l'utilisateur..
\begin{description}
 \item{timetask\_id} int(8) : C'est la cl� primaire de la table.
 \item{timetask\_timeupdate} timestamp(14) : Donne au format Unix la date de derni�re mise � jour du tuple.
 \item{timetask\_timecreate} timestamp(14) : Donne au format Unix la date de cr�ation du tuple.
 \item{timetask\_userupdate} int(8) : Id de l'utilisateur qui � op�r� la derni�re mise � jour.
 \item{timetask\_usercreate} int(8) : Id de l'utilisateur qui � cr�� le tuple.
 \item{timetask\_user\_id} int(8) : Cl� �trang�re pointant vers UserObm indiquant l'utilisateur qui a effectu� cette t�che.
 \item{timetask\_date} timestamp(14) : Date � laquelle a eu lieu cette t�che.
 \item{timetask\_projecttask\_id} int(8) : Cl� �trang�re pointant vers ProjectTask indiquant la t�che (et par ce biais le projet) ----
 \item{timetask\_length} int(2) : Dur�e de la t�che en fractions de journ�es.
 \item{timetask\_tasktype\_id} int(8) : Cl� �trang�re pointant vers TaskType indiquant le type de le t�che effectu�e.
 \item{timetask\_label} varchar(255) : Commentaires saisis par l'utilisateur pour cette t�che.
 \item{timetask\_status} varchar(1) : Etat de validation de cette t�che.
\end{description}

\clearpage


\subsection{Actions, droits, et fonctions}
Dans cette section seront list�es et d�crites les actions du module, ainsi que les droits qui leurs sont appliqu�s et l'intitul� des fonctions execut�es. 

\subsubsection{Actions}

Voici la liste des action du module \timemanager, ainsi qu'une description sommaire de chacune d'entre elles.\\
\begin{tabular}{|c|p{9cm}|}
 \hline
 \textbf{Intitul�} & \textbf{Description} \\
 \hline
 \hline
 <index> & Action par d�faut, elle affiche une vue de la semaine, le formulaire permettant la saisie d'une t�che et la liste d�j� saisies pour la semaine. \\ 
 \hline
 <viewmonth> & Cette action affiche une vue de l'�tat de la saisie pour un mois. Elle permet de visualiser les parties du mois pour lesquelles la gestion des temps a �t� remplie.\\ 
 \hline
  <globalview> & Cette action affiche un �cran qui permet de visualiser, pour un mois, l'�tat de remplissage de la gestion des temps pour tous les utilisateurs. Il est possible de valider le mois pour les personnes qui ont enti�rement rempli le mois. \\ 
 \hline
  <insert> & Ajoute une t�che saisie dans la base de donn�es. \\
 \hline
  <validate> & Valide les t�ches saisies sur un mois par un utilisateur. Cette action que seul un administrateur peut effectuer emp�che ensuite toute modification sur les t�ches du mois concern�. \\
 \hline
  <unvalidate> & Annule la validation d'un mois. \\ 
 \hline
 <stats> & Cette action affiche l'�cran de statistiques. Il permet de visualiser les informations concernant la r�partition du temps de travail entre les diff�rents projets et les diff�rents types de t�ches. . \\
 \hline
 <delete> & Supprime une t�che.  \\ 
 \hline
 <detailupdate> & Affiche un popup permettant de modifier les informations concernant une t�che pr�c�demment saisies. \\ 
 \hline
\end{tabular}

\clearpage

\subsubsection{Droits}

Ci-dessous le tableau pr�sentant les droits necessaires � l'execution de chaque action ainsi que l'intitul� de ce droit. \\
\begin{tabular}{|c|c|l|}
 \hline
 \textbf{Action} & \textbf{Code du droit} & \textbf{Intitul� du droit}\\
 \hline
 \hline
  <index> & time\_read & Droits d'Utilisateur en lecture. \\ 
 \hline
  <viewmonth> & time\_read & Droits d'Utilisateur en lecture. \\ 
 \hline
  <globalview> & time\_admin\_read & Droits d'Administrateur en lecture. \\ 
 \hline
  <insert> & time\_write & Droits d'Utilisateur en �criture. \\
 \hline
  <validate> & time\_admin\_write & Droits d'Administrateur en �criture. \\
 \hline
  <unvalidate> & time\_admin\_write & Droits d'Administrateur en �criture. \\
 \hline
  <stats> & time\_read & Droits d'Utilisateur en lecture. \\ 
 \hline
  <delete> & time\_write & Droits d'Utilisateur en �criture. \\ 
 \hline
  <detailupdate> & time\_write & Droits d'Utilisateur en �criture. \\ 
 \hline
  <display> & project\_read & Droits d'Utilisateur en ecriture. \\
 \hline
  <dispref\_display> & project\_read & Droits d'Utilisateur en ecriture. \\
 \hline
  <dispref\_level> & project\_read & Droits d'Utilisateur en ecriture. \\
 \hline
\end{tabular}

\subsubsection{Fonctions}
 Ci-dessous le tableau pr�sentant la ou les fonctions appel�es pour chacune des actions. \\
\begin{tabular}{|c|c|}
 \hline
 \textbf{Action} & \textbf{Fonction(s)}\\
 \hline
 \hline
  <index> & dis\_time\_links() \\
          & dis\_time\_index() \\  
          & dis\_time\_list() \\ 
 	  & dis\_time\_search\_form() \\
 \hline
  <viewmonth> & dis\_time\_links() \\ 
              & dis\_time\_index() \\
   	      & dis\_time\_search\_form() \\
 \hline
  <globalview> & dis\_time\_links() \\ 
               & dis\_time\_index() \\
 \hline
  <insert> & dis\_time\_links() \\
 	   & run\_query\_insert() \\
	   & run\_query\_validate() \\
	   & dis\_time\_index() \\
	   & dis\_time\_list() \\
	   & dis\_time\_search\_form() \\
 \hline
 \end{tabular}

\begin{tabular}{|c|c|p{3cm}|}
 \hline
 \textbf{Action} & \textbf{Fonction(s)} \\
 \hline
 \hline
  <validate> & run\_query\_adminvalidate() \\
	     & dis\_time\_links() \\
	     & dis\_time\_index() \\
 \hline
  <unvalidate> & run\_query\_adminunvalidate() \\
	     & dis\_time\_links() \\
	     & dis\_time\_index() \\
 \hline
  <stats> & run\_query\_stat\_project() \\ 
	  & run\_query\_stat\_taskttype() \\
          & dis\_time\_links() \\ 
          & dis\_time\_statsuser() \\
   	  & dis\_time\_search\_form() \\
 \hline
  <delete> & dis\_time\_links() \\
 	   & run\_query\_delete() \\
	   & run\_query\_validate() \\
	   & dis\_time\_index() \\
	   & dis\_time\_list() \\
	   & dis\_time\_search\_form() \\
 \hline
\end{tabular}

\begin{tabular}{|c|c|p{3cm}|}
 \hline
 \textbf{Action} & \textbf{Fonction(s)} \\
 \hline
 \hline
  <detailupdate> & run\_query\_tasktype()\\ 
                 & run\_query\_project() \\
                 & run\_query\_projecttask() \\
                 & first\_day\_week() \\
                 & get\_this\_week() \\
                 & ou \\
	         & run\_query\_update() \\
	         & run\_query\_validate() \\
 \hline
  <display> & run\_query\_display\_pref() \\
            & dis\_time\_display\_pref() \\
 \hline
  <dispref\_display> & run\_query\_display\_pref\_update() \\
                     & run\_query\_display\_pref() \\
                     & dis\_time\_display\_pref() \\
 \hline
  <dispref\_level> & run\_query\_display\_pref\_level\_update() \\
                   & run\_query\_display\_pref() \\
                   & dis\_time\_display\_pref() \\
 \hline 
\end{tabular}

\subsection{Fonctionalit�}
 Dans cette partie seront abord�es les fonctionnalit�s du module \project. Ces fonctionnalit�s seront �tudi�s d'un point de vue technique et non utilisateur. Il va donc plus etre sujet du fonctionnement que de l'utilit� et de l'utilisation du module.

\subsubsection{Vue Hebdomadaire}

 Nous alons dans un premier temps traiter les vues quotidiennes et hebdomadaire.

 Les etapes de la production de cet affichage sont les suivantes : 
 \begin{itemize}
  \item{R�cup�ration des informations sur le remplissage de la semaine}
  \item{R�cup�ration des informations n�cessaires � la cr�ation du formulaire de saisie}
  \item{Affichage de la vue}
 \end{itemize}

 Donc pour la r�cup�ration des donn�es sur les t�ches en base, trois fonctions sont utilis�es � savoir :
 \begin{description}
  \item{\fonction{run\_query\_task\_one\_week}} : Cette fonction permet de r�cup�rer l'�tat de remplissage des diff�rents jours de la semaine. 
  \item{\fonction{run\_query\_search}} : Cette fonction r�cup�re en base de donn�es la liste des t�ches d�j� saisies pour la semaine.
  \item{\fonction{run\_query\_valid\_search}} : Cette fonction r�cup�re l'�tat de validation des diff�rents jours de la semaine. 
 \end{description}

 Donc pour la r�cup�ration des donn�es sur les t�ches que l'utilisateur est susceptible de r�aliser, trois fonctions sont utilis�es � savoir :
 \begin{description}
  \item{\fonction{run\_query\_tasktype}} : Cette fonction permet de r�cup�rer la liste des types de t�ches que l'utilisateur est susceptible d'effectuer (c'est � dire les t�ches non factur�es et celles qui sont ratach�es � un projet auquel l'utilisateur prend part)
  \item{\fonction{run\_query\_project}} : Cette fonction r�cup�re en base de donn�es la liste de tous les projets auquels participe l'utilisateur.
  \item{\fonction{run\_query\_projecttask}} : Cette fonction r�cup�re de la liste des sous-t�ches qui composent les projets qui ont �t� r�cup�r�s par la fonction pr�c�dente.
 \end{description}

 A partir des informations g�n�r�es, on appelle les fonctions :
 \begin{description}
  \item{\fonction{dis\_time\_links}} : qui affiche les liens vers les semaines pr�c�dentes et suivantes.
  \item{\fonction{dis\_time\_index}} : qui affiche une vue synth�tique de la semaine : cet affichage met en �vidence l'�tat de remplissage des diff�rents jours et permet de faire ressortir de fa�on visible celles qui restent � compl�ter.
  \item{\fonction{dis\_form\_addtask}} : qui affiche le formumlaire de saisie si il reste des journ�es � remplir (cf formulaire de saisie des t�ches)
  \item{\fonction{dis\_time\_list\_result}} : qui affiche la liste des t�ches d�j� saisies pour la semaine. Il s'agit d'une instance de la classe OBM\_DISPLAY ; l'affichage de cette liste est donc param�trable depuis l'�cran affichage.
  \item{\fonction{dis\_time\_search\_form}} : qui affiche, pour l'administrateur seulement, une boite de s�lection des utilisateurs. Cela lui permet d'intervenir a la place de n'importe quel utilisateur.
 \end{description}

\subsubsection{Vue mensuelle}

 Donc pour la r�cup�ration des donn�es sur les t�ches qui ont d�j� �t� saisies pour le mois en cours :
 \begin{description}
  \item{\fonction{run\_query\_task\_one\_month}} : Cette fonction permet de r�cup�rer les informations sur le remplissage de la gestion des temps pour le mois en cours.
 \end{description}

 A partir des informations r�cup�r�es, on appelle les fonctions :
 \begin{description}
  \item{\fonction{dis\_time\_links}} : qui affiche les liens vers les mois pr�c�dents et suivants.
  \item{\fonction{dis\_time\_index}} : qui affiche une vue synth�tique du mois : cet affichage met en �vidence l'�tat de remplissage des diff�rents jours et permet de faire ressortir de fa�on visible celles qui restent � compl�ter. Depuis cet �cran, on peut acc�der aux vues hebdomadaires des diff�rentes semains du mois.
  \item{\fonction{dis\_time\_search\_form}} : qui affiche, pour l'administrateur seulement, une boite de s�lection des utilisateurs. Cela lui permet d'intervenir a la place de n'importe quel utilisateur.
 \end{description}

\subsubsection{Formulaire de saisie des t�ches}

Afin de faciliter la saisie, l'action sur les listes cat�gorie ou projet modifie le contenu de ces m�mes listes et de la liste de sous-t�ches.
Deux �l�ments sont n�cessaires pour permettre ce comportement :
\begin{itemize}
\item{l'envoi d'informations sur les projets dans la page web}
\item{la modification du formulair � l'aide de scrips JavaScript}
\end{itemize}

On va donc commencer par g�nerer deux tableaux de donn�s qui seront inclus dans la source de la page html : un pour les projets qui pourront apparaitre dans la liste d�roulantes, un pour les sous-t�ches qui composent ces m�mes projets.

 Ces deux tableaux seront exploit�s par les scripts qui seront appel�s quand l'utilisateur agira sur la liste des cat�gories ou sur celle des projets :
 \begin{itemize}
  \item{fill\_tasktype} : met � jour la valeur s�lectionn�e pour la rendre coh�rente avec le projet s�lectionn�e.
  \item{fill\_projectall} : remplit la liste des projets avec tous les projets auquels l'utilisateur prend part.
  \item{fill\_project} : remplit cette m�me liste mais avec seulement les projets qui correspondent au type de t�che s�lectionn�.
  \item{fill\_projecttask} : remplit la liste des sous-t�ches en fonction du projet s�lectionn�.
  \item{select\_default} : s�lectionne les valeurs ad�quates lors du chargement de l'�cran permettant la modification d'un t�che.
 \end{itemize}

\subsubsection{Insertion/modification d'une t�che}

 La modification des informations dans la base de donn�es se fait en deux �tapes
 \begin{itemize}
  \item{mise � jour de la base de donn�es} : \fonction{run\_query\_insert} ajoute cette t�che dans la base de donn�es ou \fonction{run\_query\_update} mets � jour la t�che modifi�e.
  \item{validation des journ�es} : \fonction{run\_query\_validate} marque comme complets les jours que cette modification a fini de remplire ou, au contraire, d�tecte si une modification a remis en cause le fait qu'une journ�e soit valid�e.
 \end{itemize}

\subsubsection{Validation d'un mois}

 Cette fonctionnalit� n'est disponible que pour l'administrateur. Il peut :
 \begin{itemize}
  \item{compl�ter un mois incomplet}
  \item{valider un mois} : \fonction{run\_query\_adminvalidate} marque comme valid� un mois. Il ne sera alors plus possible de modifier les t�ches qui ont �t� valid�es.
  \item{annuler la validation} : \fonction{run\_query\_adminunvalidate} annule cette validation. C'est la seule solution pour pouvoir corriger � nouveau des t�ches pr�cedemment v�rouill�es.
 \end{itemize}

% Documentation technique d'OBM : module Project
% ALIACOM Bastien Continsouzas
% $Id


\clearpage
\section{Project}

Le module \project \obm.

\subsection{Organisation de la base de donn�es}

Le \project utilise 4 tables :
\begin{itemize}
 \item Project
 \item ProjectTask
 \item ProjectUser
 \item ProjectStat
\end{itemize}

\subsection{Project}
Table principale des informations d'un projet.\\

\begin{tabular}{|p{3cm}|c|p{5.4cm}|p{2.6cm}|}
\hline
\textbf{Champs} & \textbf{Type} & \textbf{Description} & \textbf{Commentaire} \\
\hline
\_id & int 8 & Identifiant & Cl� primaire \\
\hline
\_timeupdate & timestamp 14 & Date de mise � jour & \\
\hline
\_timecreate & timestamp 14 & Date de cr�ation & \\
\hline
\_userupdate & int 8 & Id du modificateur & \\
\hline
\_usercreate & int 8 & Id du createur & \\
\hline
\_name & varchar 128 & Nom du projet & \\
\hline
\_tasktype\_id & int 8 & Type de t�che du projet & \\
\hline
\_company\_id & int 8 & Soci�t� contractante & NULL (projets internes)\\
\hline
\_deal\_id & int 8 & Affaire d'origine & NULL (projets internes)\\
\hline
\_soldtime & int 8 & Dur�e (jours) vendue & \\
\hline
\_archive & int 2 & Indicateur d'archivage & (1 = 0ui)\\
\hline
\_comment & text (64k) & Commentaire &\\
\hline
\end{tabular}


\subsection{ProjectTask}
Table des informations des t�ches des projets.\\

\begin{tabular}{|p{3cm}|c|p{5.4cm}|p{2.6cm}|}
\hline
\textbf{Champs} & \textbf{Type} & \textbf{Description} & \textbf{Commentaire} \\
\hline
\_id & int 8 & Identifiant & Cl� primaire \\
\hline
\_project\_id & int 8 & Projet de la t�che & \\
\hline
\_timeupdate & timestamp 14 & Date de mise � jour & \\
\hline
\_timecreate & timestamp 14 & Date de cr�ation & \\
\hline
\_userupdate & int 8 & Id du modificateur & \\
\hline
\_usercreate & int 8 & Id du createur & \\
\hline
\_label & varchar 128 & Label de la t�che & \\
\hline
\_parenttask\_id & int 8 & T�che m�re (ProjectTask) & 0 si racine \\
\hline
\_rank & int 8 & Ordre d'affichage de la t�che & \\
\hline
\end{tabular}


\subsection{ProjectUser}
Table d'affectation (avec informations associ�es) d'utilisateurs aux t�ches d'un projet.

\begin{tabular}{|p{3cm}|c|p{5.4cm}|p{2.6cm}|}
\hline
\textbf{Champs} & \textbf{Type} & \textbf{Description} & \textbf{Commentaire} \\
\hline
\_id & int 8 & Identifiant & Cl� primaire \\
\hline
\_project\_id & int 8 & Projet & \\
\hline
\_user\_id & int 8 & Utilisateur & \\
\hline
\_projecttask\_id & int 8 & T�che & \\
\hline
\_timeupdate & timestamp 14 & Date de mise � jour & \\
\hline
\_timecreate & timestamp 14 & Date de cr�ation & \\
\hline
\_userupdate & int 8 & Id du modificateur & \\
\hline
\_usercreate & int 8 & Id du createur & \\
\hline
\_projectedtime & int 8 & Dur�e pr�vue & \\
\hline
\_missingtime & int 8 & Dur�e restante estim�e & Par Chef P \\
\hline
\_validity & timestamp 14 & Date d'estimation d'avancement & \\
\hline
\_soldprice & int 8 & Prix de vente de l'utilisateur & \\
\hline
\_manager & int 1 & Statut de l'utilisateur dans le projet & 1 = CP \\
\hline
\end{tabular}

\subsubsection{Remarques}

\paragraph{projectuser\_validity} : Date de la derni�re modification de l'avancement qui permet de v�rifier la cr�dibilit� des informations disponibles sur l'avancement du projet (non impl�ment� pour l'instant).
\paragraph{projectuser\_soldprice} : Tarif journalier de cet utilisateur sur cette t�che. Permet une analyse des co�ts plus d�taill�e pour le projet (non impl�ment� pour l'instant).
\paragraph{projectuser\_manager} : Indique si la personne est chef de projet. Influe sur les droits de modification.\\


ProjectUser est en fait la table de liaison entre T�ches et Membres d'un projet.
L'ajout d'un membre au projet ins�re un tuple dans cette table avec une t�che nulle. Ainsi, la \textbf{liste des membres d'un projet} est directement tir�e de cette table.


\subsection{ProjectStat}
Table d'historique des informations relatives � l'avancement des projet.\\

\begin{tabular}{|p{3cm}|c|p{5.4cm}|p{2.6cm}|}
\hline
\textbf{Champs} & \textbf{Type} & \textbf{Description} & \textbf{Commentaire} \\
\hline
\_project\_id & int 8 & Projet & \\
\hline
\_usercreate & int 8 & Id du cr�ateur & \\
\hline
\_date & timestamp 14 & Date d'estimation d'avancement & \\
\hline
\_useddays & int 8 & Dur�e pass�e (jours) sur le projet & \\
\hline
\_remainingdays & int 8 & Dur�e restante estim�e & Par Chef P \\
\hline
\end{tabular}


\clearpage


\subsection{Actions, droits, et fonctions}
Dans cette section seront list�es et d�crites les actions du module, ainsi que les droits qui leurs sont appliqu�s et l'intitul� des fonctions execut�es. 

\subsubsection{Actions}

Voici la liste des action du module \project, ainsi qu'une description sommaire de chacune d'entre elles.\\

\begin{tabular}{|l|p{9cm}|}
 \hline
 \textbf{Intitul�} & \textbf{Description} \\
 \hline
 \hline
  index & (D�faut) formulaire de recherche de projets. \\ 
 \hline
  search & R�sultat de recherche de projets. \\
 \hline
  new & Formulaire de cr�ation d'un projet. \\
 \hline
  detailconsult & Fiche d�tail d'un projet. \\
 \hline
  detailupdate & Formulaire de modification d'un projet. \\
 \hline
  insert & Insertion d'un projet. \\
 \hline
  update & Mise � jour du projet. \\
 \hline
  check\_delete & V�rification avant suppression du projet. \\
 \hline
  delete & Suppression du projet. \\
 \hline
  task & Liste des t�ches d�finies et formulaire de nouvelle t�che. \\
 \hline
  task\_add & Ajout d'une t�che au projet. \\
 \hline
  task\_del & Suppression de t�ches. \\
 \hline
  member & Liste des participants au projet. \\
 \hline
  sel\_member & Appel externe au module \user pour l'ajout de participants).\\
 \hline
  member\_add & Ajout d'un participant au projet. \\
 \hline
  member\_del & Suppression de participants au projet. \\
 \hline
  member\_update & Mise � jour d'un participant (statut, chef de projet). \\
 \hline
  allocate & Formulaire d'affectation des participants aux t�ches. \\
 \hline
  allocate\_update & Mise � jour de l'affectations des participants. \\
 \hline
  progress & Formulaire de progresion d'un projet. \\
 \hline
  progress\_update & Mise � jour de la progression du projet. \\
 \hline
  display\_pref & Ecran de modification des pr�f�rences d'affichage. \\
 \hline
  dispref\_display & Modifie l'affichage d'un �l�ment. \\
 \hline
  dispref\_level & Modifie l'ordre d'affichage d'un �l�ment. \\
 \hline
\end{tabular}

\clearpage

\subsubsection{Droits}

Ci-dessous le tableau pr�sentant les droits necessaires � l'execution de chaque action ainsi que l'intitul� de ce droit. \\
\begin{tabular}{|c|c|l|}
 \hline
 \textbf{Action} & \textbf{Code du droit} & \textbf{Intitul� du droit}\\
 \hline
 \hline
  <index> & project\_read & Droits d'Utilisateur en lecture. \\ 
 \hline
  <search> & project\_read & Droits d'Utilisateur en lecture. \\ 
 \hline
  <new> & project\_write & Droits d'Utilisateur en �criture. \\ 
 \hline
  <detailconsult> & project\_read & Droits d'Utilisateur en �criture. \\ 
 \hline
  <detailupdate> & project\_write & Droits d'Utilisateur en �criture. \\
 \hline
  <insert> & project\_write & Droits d'Utilisateur en �criture. \\
 \hline
  <update> & project\_write & Droits d'Utilisateur en �criture. \\
 \hline
  <check\_delete> & project\_write & Droits d'Utilisateur en �criture. \\
 \hline
  <delete> & project\_write & Droits d'Utilisateur en �criture. \\
 \hline
  <task> & project\_write & Droits d'Utilisateur en �criture. \\
 \hline
  <task\_add> & project\_write & Droits d'Utilisateur en �criture. \\
 \hline
  <task\_del> & project\_write & Droits d'Utilisateur en �criture. \\
 \hline
  <member> & project\_write & Droits d'Utilisateur en �criture. \\
 \hline
  <sel\_member> & project\_write & Droits d'Utilisateur en �criture. \\
 \hline
  <member\_add> & project\_write & Droits d'Utilisateur en �criture. \\
 \hline
  <member\_del> & project\_write & Droits d'Utilisateur en �criture. \\
 \hline
  <member\_update> & project\_write & Droits d'Utilisateur en �criture. \\
 \hline
  <allocate> & project\_write & Droits d'Utilisateur en �criture. \\ 
 \hline
  <allocate\_update> & project\_write & Droits d'Utilisateur en �criture. \\ 
 \hline
  <progress> & project\_write & Droits d'Utilisateur en �criture. \\
 \hline
  <progress\_update> &  project\_write & Droits d'Utilisateur en �criture. \\
 \hline
  <display> & project\_read & Droits d'Utilisateur en lecture. \\
 \hline
  <dispref\_display> & project\_read & Droits d'Utilisateur en lecture. \\
 \hline
  <dispref\_level> & project\_read & Droits d'Utilisateur en lecture. \\
 \hline
\end{tabular}


\subsubsection{Fonctions}
 Ci-dessous le tableau pr�sentant la ou les fonctions appel�es pour chacune des actions. \\
\begin{tabular}{|c|c|}
 \hline
 \textbf{Action} & \textbf{Fonction(s)}\\
 \hline
 \hline
  <index> & dis\_project\_search\_form() \\
 \hline
  <search> & dis\_project\_search\_form() \\
           & dis\_project\_search\_list() \\
 \hline
  <new> & run\_query\_projecttype() \\ 
        & html\_project\_form() \\
 \hline
  <init> & run\_query\_detail() \\ 
         & html\_project\_init\_form() \\
 \hline
  <task> & run\_query\_projectname() \\
 	       & run\_query\_tasks() \\
	       & html\_project\_taskadd\_form() \\
	       & html\_project\_tasklist() \\
 \hline
  <member> & run\_query\_projectname() \\
 	         & run\_query\_tasks() \\
	         & run\_query\_members() \\
	         & html\_project\_member\_form() \\
 \hline
  <alloacte> & run\_query\_projectname() \\
 	     & run\_query\_tasks() \\
	     & run\_query\_members() \\
	     & run\_query\_allocation() \\
             & html\_project\_allocate\_form() \\
 \hline
  <allocate\_update> & run\_query\_projectupdate() \\
             	    & run\_query\_detail() \\ 
 	     	    & run\_query\_tasks() \\
	     	    & run\_query\_members() \\
	     	    & run\_query\_allocation() \\
             	    & html\_project\_consult() \\
 \hline
 \end{tabular}

\begin{tabular}{|c|c|p{3cm}|}
 \hline
 \textbf{Action} & \textbf{Fonction(s)} \\
 \hline
 \hline
  <detailconsult> & run\_query\_create() \\
                  & run\_query\_detail() \\ 
                  & html\_project\_consult() \\
                  & ou \\
                  & run\_query\_insert() \\
                  & run\_query\_detail() \\ 
                  & html\_project\_consult() \\
                  & ou \\
                  & run\_query\_detail() \\ 
 	          & run\_query\_tasks() \\
	          & run\_query\_members() \\
	          & run\_query\_allocate() \\
                  & html\_project\_consult() \\
 \hline
  <detailupdate> & run\_query\_detail() \\ 
                 & html\_project\_init\_form() \\
                 & ou \\
                 & run\_query\_projecttype() \\
                 & html\_project\_form() \\
 \hline
 <d\_update> & run\_query\_update() \\ 
             & run\_query\_detail() \\ 
 	     & run\_query\_tasks() \\
	     & run\_query\_members() \\
	     & run\_query\_allocate() \\
             & html\_project\_consult() \\
 \hline
  <progress> & run\_query\_projectname() \\
 	     & run\_query\_tasks() \\
	     & run\_query\_members() \\
	     & run\_query\_allocate() \\
             & html\_project\_advance() \\
 \hline
 <progress\_update> & run\_query\_progress() \\ 
             	  & run\_query\_statlog() \\ 
             	  & run\_query\_detail() \\ 
 	     	  & run\_query\_tasks() \\
	     	  & run\_query\_members() \\
	     	  & run\_query\_allocation() \\
             	  & html\_project\_consult() \\
 \hline
\end{tabular}

\begin{tabular}{|c|c|p{3cm}|}
 \hline
 \textbf{Action} & \textbf{Fonction(s)} \\
 \hline
 \hline
 <task\_add> & run\_query\_tasklist\_insert()\\ 
             & run\_projectname() \\
             & run\_query\_tasks() \\
	     & html\_project\_taskadd\_form() \\
	     & html\_project\_tasklist() \\
 \hline 
 <task\_del> & run\_query\_tasklist\_delete()\\ 
             & run\_projectname() \\
             & run\_query\_tasks() \\
	     & html\_project\_taskadd\_form() \\
	     & html\_project\_tasklist() \\
 \hline
 <member\_add> & run\_query\_projectname() \\
               & run\_query\_memberlist\_insert()\\ 
 	       & run\_query\_tasks() \\
	       & run\_query\_members() \\
	       & html\_project\_member\_form() \\
 \hline 
 <member\_del> & run\_query\_projectname() \\
               & run\_query\_memberlist\_delete()\\ 
 	       & run\_query\_tasks() \\
	       & run\_query\_members() \\
	       & html\_project\_member\_form() \\
 \hline 
 <member\_change> & run\_query\_projectname() \\
                  & run\_query\_memberstatus\_change()\\ 
 	          & run\_query\_tasks() \\
	          & run\_query\_members() \\
	          & html\_project\_member\_form() \\
 \hline
  <display> & run\_query\_display\_pref() \\
            & dis\_project\_display\_pref() \\
 \hline
  <dispref\_display> & run\_query\_display\_pref\_update() \\
                     & run\_query\_display\_pref() \\
                     & dis\_project\_display\_pref() \\
 \hline
  <dispref\_level> & run\_query\_display\_pref\_level\_update() \\
                   & run\_query\_display\_pref() \\
                   & dis\_project\_display\_pref() \\
 \hline
\end{tabular}

\subsection{Fonctionalit�}
 Dans cette partie seront abord�es les fonctionnalit�s du module \project. Ces fonctionnalit�s seront �tudi�s d'un point de vue technique et non utilisateur. Il va donc plus �tre sujet du fonctionnement que de l'utilit� et de l'utilisation du module.

\subsubsection{Cr�ation du projet}

 Deux cas de figure sont possibles :
 \begin{description}
  \item{Projet reli� � une affaire} : on modifie juste les informations relatives au projet dans la table Deal avec \fonction{run\_query\_create}
  \item{Projet interne} : on cr�e une nouvelle entr�e dans la table Deal avec \fonction{run\_query\_insert}
 \end{description}

\subsubsection{Affichage du projet}

 Il se fait en deux temps
 \begin{itemize}
  \item{R�cup�ration des informations dans la table} 
  \item{Affichage des diff�rentes informations}
 \end{itemize}

 On r�cup�re donc les donn�es suivantes
 \begin{itemize}
  \item{\fonction{run\_query\_detail}} : informations qui caract�risent le projet (nom, soci�t�, type de projet, temps vendu/pr�vu)
  \item{\fonction{run\_query\_tasks}} : liste des sous-t�ches du projet et les informations sur leur arborescence.
  \item{\fonction{run\_query\_members}} : liste des participants au projet et leur statut (chef de projet/participant)
  \item{\fonction{run\_query\_allocation}} : liste des affectations de personnes au sous-t�ches du projet.
 \end{itemize}

 On affiche ces informations avec \fonction{html\_project\_consult}

\subsubsection{Modifications}

 Les informations susceptibles d'�voluer sont :
 \begin{itemize}
  \item{informations g�n�rales} : \fonction{run\_query\_update} modifie le nom, le temps vendu/pr�vu
  \item{liste des sous-t�ches} : \fonction{run\_query\_tasklist\_insert} ajoute une sous-t�che et \fonction{run\_query\_tasklist\_delete} en enl�ve.
  \item{liste des participants} : \fonction{run\_query\_memberlist\_insert} ajoute un (ou plusieurs)  participant, \fonction{run\_query\_memberlist\_delete} en enl�ve et \fonction{run\_query\_memberstatus\_change} modifie le statut (chef de projet/participant).
  \item{liste des affectations} : \fonction{run\_query\_projectupdate} met � jour la liste des affectations.
  \item{liste des affectations} : \fonction{run\_query\_progress} met � jour l'�tat d'avancement des participants.
 \end{itemize}
% Documentation technique d'OBM : module Facture
% ALIACOM Pierre Baudracco
% $Id$


\clearpage
\section{Facture}

Le module \invoice \obm.

\subsection{Organisation de la base de donn�es}

Le module \invoice utilise 3 tables :
\begin{itemize}
 \item Invoice
 \item InvoiceStatus
 \item InvoiceTerm
\end{itemize}

\subsection{Invoice}
Table principale des informations d'une facture.\\

\begin{tabular}{|p{3cm}|c|p{5.4cm}|p{2.6cm}|}
\hline
\textbf{Champs} & \textbf{Type} & \textbf{Description} & \textbf{Commentaire} \\
\hline
\_id & int 8 & Identifiant & Cl� primaire \\
\hline
\_timeupdate & timestamp 14 & Date de mise � jour & \\
\hline
\_timecreate & timestamp 14 & Date de cr�ation & \\
\hline
\_userupdate & int 8 & Id du modificateur & \\
\hline
\_usercreate & int 8 & Id du createur & \\
\hline
\_company\_id & int 8 & Soci�t� client ou forunisseur & \\
\hline
\_deal\_id & int 8 & Affaire concern�e & \\
\hline
\_project\_id & int 8 & Projet concern� & \\
\hline
\_number & varchar 10 & Num�ro de la facture & \\
\hline
\_label & varchar 128 & Label & \\
\hline
\_amount\_ht & double 10,2 & montant Hors taxe &\\
\hline
\_amount\_ttc & double 10,2 & montant TTC &\\
\hline
\_status\_id & int 4 & Etat & \\
\hline
\_date & date & Jour de facturation & pr�vu ou r�el \\
\hline
\_inout & char 1 & Type client ou fournisseur & (+ client, - fournisseur) \\
\hline
\_archive & char 1 & Indicateur d'archivage & (1 = 0ui)\\
\hline
\_comment & text (64k) & Commentaire &\\
\hline
\end{tabular}


\subsection{InvoiceStatus}
Table des informations des �tats des factures.\\

\begin{tabular}{|p{3cm}|c|p{5.4cm}|p{2.6cm}|}
\hline
\textbf{Champs} & \textbf{Type} & \textbf{Description} & \textbf{Commentaire} \\
\hline
\_id & int 8 & Identifiant & Cl� primaire \\
\hline
\_payment & int 1 & Indicateur facture doit avoir des paiements & (1 = Oui) \\
\hline
\_archive & char 1 & Indicateur facture de cet Etat est archivable & (1 = 0ui)\\
\hline
\_label & varchar 24 & Label de l'�tat & \\
\hline
\end{tabular}


\subsubsection{Valeurs par d�faut}

Il est possible de g�rer ses propres �tats de factures mais des �tats par d�faut sont d�finis :\\

\begin{tabular}{|p{3cm}|c|c|p{6cm}|}
\hline
\textbf{Etat} & \textbf{Paiement} & \textbf{Archivable} & \textbf{Commentaire} \\
\hline
A cr�er &  &  & \\
\hline
Envoy�e & X  &  & \\
\hline
Pay�e en partie & X &  & \\
\hline
Litige & X &  & \\
\hline
Annul�e &  & X & \\
\hline
Pertes et profits &  & X & \\
\hline
Pay�e & X & X & \\
\hline
\end{tabular}

\subsubsection{Remarques}

\paragraph{L'ordre d'affichage} des �tats est : ``Non archivable, Pas de paiements'', ce qui donne l'ordre du tableau ci-dessus.


\subsection{ProjectTerm}
Table des consitions de r�glement ou d�lais de paiements.

\begin{tabular}{|p{3cm}|c|p{5.4cm}|p{2.6cm}|}
\hline
\textbf{Champs} & \textbf{Type} & \textbf{Description} & \textbf{Commentaire} \\
\hline
\_id & int 8 & Identifiant & Cl� primaire \\
\hline
\_project\_id & int 8 & Projet & \\
\hline
\_user\_id & int 8 & Utilisateur & \\
\hline
\_projecttask\_id & int 8 & T�che & \\
\hline
\_timeupdate & timestamp 14 & Date de mise � jour & \\
\hline
\_timecreate & timestamp 14 & Date de cr�ation & \\
\hline
\_userupdate & int 8 & Id du modificateur & \\
\hline
\_usercreate & int 8 & Id du createur & \\
\hline
\_projectedtime & int 8 & Dur�e pr�vue & \\
\hline
\_missingtime & int 8 & Dur�e restante estim�e & Par Chef P \\
\hline
\_validity & timestamp 14 & Date d'estimation d'avancement & \\
\hline
\_soldprice & int 8 & Prix de vente de l'utilisateur & \\
\hline
\_manager & int 1 & Statut de l'utilisateur dans le projet (si t�che est nulle) & 1 = CP \\
\hline
\end{tabular}



\subsection{Actions et droits}

Voici la liste des actions du module \project, avec le droit d'acc�s requis ainsi qu'une description sommaire de chacune d'entre elles.\\

\begin{tabular}{|l|c|p{9.5cm}|}
 \hline
 \textbf{Intitul�} & \textbf{Droit} & \textbf{Description} \\
 \hline
 \hline
  index & read & (D�faut) formulaire de recherche de factures. \\ 
 \hline
  search & read & R�sultat de recherche de factures. \\
 \hline
  new & write & Formulaire de cr�ation d'une facture. \\
 \hline
  detailconsult & read & Fiche d�tail d'une facture. \\
 \hline
  detailupdate & write & Formulaire de modification d'une facture. \\
 \hline
  duplicate & write & Formulaire de cr�ation � partir d'une facture existante. \\
 \hline
  insert & write & Insertion d'une facture. \\
 \hline
  update & write & Mise � jour d'une facture. \\
 \hline
  check\_delete & write & V�rification avant suppression d'une facture. \\
 \hline
  delete & write & Suppression d'une facture projet. \\
 \hline
  display & read & Ecran de modification des pr�f�rences d'affichage. \\
 \hline
  dispref\_display & read & Modifie l'affichage d'un �l�ment. \\
 \hline
  dispref\_level & read & Modifie l'ordre d'affichage d'un �l�ment. \\
 \hline
  document\_add & write & Ajout de liens vers des documents. \\
 \hline
\end{tabular}

\clearpage
\section{Etendre OBM}
% Documentation technique d'OBM : Etendre OBM : Ajouter un champ
% ALIACOM Pierre Baudracco
% $Id$


%%\clearpage
\subsection{Ajouter un champ dans un module}

Diff�rentes �tapes sont n�cessaires pour ajouter un champ dans un module. Selon l'utilisation du nouveau champ, certaines �tapes peuvent �tres ignor�es (ex: inclusion dans moteur de recherche, affichage dans r�sultat,...).\\

\begin{itemize}
 \item Modifier les scripts de cr�ation de base de donn�es pour inclure le nouveau champ.
   \begin{verbatim}
emacs scripts/0.8/create_obmdb_0.8.mysql.sql
emacs scripts/0.8/create_obmdb_0.8.pgsql.sql
   \end{verbatim}
 \item Modifier (ou cr�er) les scripts de mise � jour de base de donn�es pour inclure le nouveau champ.
   \begin{verbatim}
emacs scripts/0.8/update_0.8.1-0.8.2.mysql.sql
emacs scripts/0.8/update_0.8.1-0.8.2.pgsql.sql
   \end{verbatim}
 \item (module\_index.php) R�colter le param�tre issu du formulaire (get\_param\_module)\\
 \item Si champ s�lection d'une liste d�roulante :
    \begin{itemize}
     \item[*] Cr�er la table associ�e (scripts cr�ation et mise � jour bases de donn�es)\\
     \item[*] (module\_index.php) Appeler la liste de valeurs run\_query\_fieldname() pour la passer aux fonctions formulaire (mise � jour voire recherche).\\
     \item[*] (module\_display.inc) Modifier la d�finition des fonctions formulaire afin d'inclure la liste de valeur (html\_module\_form(), html\_module\_search\_form()).\\
     \item[*] (module\_query.php) Impl�menter les fonctions run\_query\_fieldname(), run\_query\_fieldname\_insert(), run\_query\_fieldname\_update(), run\_query\_fieldname\_delete(), run\_query\_fieldname\_links().
      \begin{verbatim}
function run_query_companynafcode()
function run_query_nafcode_insert()
function run_query_nafcode_update()
function run_query_nafcode_delete()
function run_query_nafcode_links()
      \end{verbatim}
     \item[*] (module\_display.inc) Impl�menter les fonctions de gestion et visualisation : html\_module\_field\_form(), dis\_field\_links().
      \begin{verbatim}
function html_company_nafcode_form()
function dis_nafcode_links()
      \end{verbatim}
     \item[*] (module\_index.php) D�finir les actions avec les droits associ�s (get\_module\_action()) et les impl�menter dans le branchement global du module (if \$action ==...) avec la traduction des messages (insert\_ok, insert\_error,...).\\
     \item[*] (module\_js.inc) Impl�menter les actions Javascript de v�rification des formulaire de gestion (check\_field\_upd(), check\_field\_new(), check\_field\_checkdel()).\\
    \end{itemize}
 \item (lang/*/module.inc) Cr�er les variables de langues (au moins pour le fran�ais et l'anglais).\\
 \item Si le champ peut �tre masqu�, indiquer l'option dans obminclude/global.inc et obm\_conf.inc et mettre la doc � jour (manuel\_obm/technique/t\_site.tex).\\
 \item (module\_display.inc) Prendre en compte le nouveau champ dans les fonctions d'affichage (html\_module\_form(), html\_module\_consult(), dis\_module\_warn\_insert()).\\
 \item Si le champ peut �tre recherch� l'impl�menter dans les fonctions html\_module\_search\_form(), html\_module\_search\_list() pour propagation du r�sultat, et dans la fonction de recherche run\_query\_search().\\
 \item (module\_query.inc) Prendre en compte le nouveau champ dans les fonctions base de donn�es : run\_query\_detail(), run\_query\_insert(), run\_query\_update(), dis\_module\_warn\_insert()).\\
 \item Si le champ doit figur� dans les r�sultats de recherche, il faut l'ajouter au module correspondant dans les pr�f�rences par d�faut (scripts/0.8/obmdb\_default\_values\_0.8.sql).
\end{itemize}


%\input{user/u_manuelr.tex}

\end{document}
